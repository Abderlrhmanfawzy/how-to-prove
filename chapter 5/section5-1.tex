%%%%%%%%%%%%%%%%%%%%%%%%%%%%%%%%%%%%%%%%%
% Author: Sibi <sibi@psibi.in>
%%%%%%%%%%%%%%%%%%%%%%%%%%%%%%%%%%%%%%%%%
\documentclass{article}
\usepackage{graphicx}
\usepackage{verbatim}
\usepackage{amsmath}
\usepackage{amsfonts}
\usepackage{amssymb}
\usepackage{tabularx}
\usepackage{mathtools}
\setlength\parskip{\baselineskip}
\begin{document}
\title{Chapter 4 (Section 4.6)}
\author{Sibi}
\date{\today}
\maketitle

% See here: http://tex.stackexchange.com/a/43009/69223
\DeclarePairedDelimiter\abs{\lvert}{\rvert}%
\DeclarePairedDelimiter\norm{\lVert}{\rVert}%

% Swap the definition of \abs* and \norm*, so that \abs
% and \norm resizes the size of the brackets, and the 
% starred version does not.
\makeatletter
\let\oldabs\abs
\def\abs{\@ifstar{\oldabs}{\oldabs*}}
%
\let\oldnorm\norm
\def\norm{\@ifstar{\oldnorm}{\oldnorm*}}
\makeatother
\newpage

\section{Problem 1}

\subsection{Solution (a)}
Yes

\subsection{Solution (b)}
No

\subsection{Solution (c)}
Yes

\section{Problem 2}
\subsection{Solution (a)}
No

\subsection{Solution (b)}
$f$ is not a function because $(haskell,a) \in f$ and $(haskell,l) \in
f$.

g is a function.

\subsection{Solution (c)}
$R = \{(J,F), (M,J), (S,M), (F,J)\}$

R is a function.

\end{document}
