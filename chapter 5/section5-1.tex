%%%%%%%%%%%%%%%%%%%%%%%%%%%%%%%%%%%%%%%%%
% Author: Sibi <sibi@psibi.in>
%%%%%%%%%%%%%%%%%%%%%%%%%%%%%%%%%%%%%%%%%
\documentclass{article}
\usepackage{graphicx}
\usepackage{verbatim}
\usepackage{amsmath}
\usepackage{amsfonts}
\usepackage{amssymb}
\usepackage{tabularx}
\usepackage{mathtools}
\setlength\parskip{\baselineskip}
\begin{document}
\title{Chapter 4 (Section 4.6)}
\author{Sibi}
\date{\today}
\maketitle

% See here: http://tex.stackexchange.com/a/43009/69223
\DeclarePairedDelimiter\abs{\lvert}{\rvert}%
\DeclarePairedDelimiter\norm{\lVert}{\rVert}%

% Swap the definition of \abs* and \norm*, so that \abs
% and \norm resizes the size of the brackets, and the 
% starred version does not.
\makeatletter
\let\oldabs\abs
\def\abs{\@ifstar{\oldabs}{\oldabs*}}
%
\let\oldnorm\norm
\def\norm{\@ifstar{\oldnorm}{\oldnorm*}}
\makeatother
\newpage

\section{Problem 1}

\subsection{Solution (a)}
Yes

\subsection{Solution (b)}
No

\subsection{Solution (c)}
Yes

\section{Problem 2}
\subsection{Solution (a)}
No

\subsection{Solution (b)}
$f$ is not a function because $(haskell,a) \in f$ and $(haskell,l) \in
f$.

g is a function.

\subsection{Solution (c)}
$R = \{(J,F), (M,J), (S,M), (F,J)\}$

R is a function.

\section{Solution 3}
\subsection{Solution (a)}
\begin{align*}
  f(a) = b \\
  f(b) = b \\
  f(c) = a
\end{align*}

\subsection{Solution (b)}
$f(2) = 2^2 - 4 = 0$

\subsection{Solution (c)}
$f(\pi) = 3$ \\
$f(-\pi) = -4$

\section{Solution 4}
\subsection{Solution (a)}
$H(Italy) = Rome$

\subsection{Solution (b)}
\begin{align*}
  A = \{1,2,3\} \\
  B = \mathbb{P}(A) \\
  F(X) = A \setminus B \\
  F({1,3}) = \{1,2,3\} \setminus \{1,3\} \\
           = \{2\}
\end{align*}

\subsection{Solution (c)}
$f(2) = (3,1)$

\section{Solution 5}
$L \circ H = \{(n_1, n_2) \in N \times N \mid \exists c \in C((n_1,c)
\in H \land (c,n_2) \in L)\}$ \\
$L \circ H (n) = n$ \\

$H \circ L = \{(c_1, c_2) \in C \times C \mid \exists n \in N((c_1, n)
\in L \land (n, c_2) \in H)\}$ \\
$H \circ L = \{(c_1,c_2) \in C \times C \mid c_2 is the
capital of the country in which c_1 is present. \}$

\section{Solution 6}
\subsection{Solution (a)}
\begin{align*}
  (f \circ g)(x) = f(g(x)) \\
  = f(2x - 1) \\
  = \frac{1}{(2x - 1)^2 + 2} \\
  = \frac{1}{4x^2 + 1 - 4x + 2} \\
  \frac{1}{4x^2 - 4x + 3} 
\end{align*}
\subsection{Solution (b)}
\begin{align*}
  (g \circ f)(x) = g(f(x)) \\
  g(\frac{1}{x^2 + 2}) \\
  \frac{2}{x^2 + 2} - 1 \\
  \frac{-x^2}{x^2 + 2}
\end{align*}

\section{Solution 7}
\subsection{Solution (a)}
Let $c$ be an arbitrary element in $C$. We must show there is a unique
$b \in B$ such that $(c,b) \in f \mid C$.

Existence proof \\
From $C \subseteq A$, it follows that $c \in A$. From $f : A \to B $,
it follows that $f(c) = b$. Since $b \in B$, it follows that $(c,b)
\in C \times B$. So, $(c,b) \in f \cap (C \times B)$.

Uniqueness proof \\
Suppose $(c,b_1) \in f \mid C$ and $(c,b_2) \in f \mid C$. So,
$(c,b_1) \in f$ and $(c,b_2) \in f$. It follows that $f(c) = b_1$ and
$f(c) = b_2$. From the definition of function, it follows that there
can be only one $b \in B$ such that $(c,b) \in f$. Thus, it follows
that $b_1 = b_2$.

Proof of $f(c) = (f \mid c)(c)$ \\
We know that $c \in C$ and $f : A \to B$. From $f$, it follows that
$f(c) = b$. So,
\begin{align*}
  f(c) = b \iff (c,b) \in f \\
  \iff (c,b) \in f \land (c,b) \in C \times B \\
  \iff (c,b) \in f \mid c \\
  \iff (f \mid c)(c) = b 
\end{align*}

\subsection{Solution (b)}
($\Rightarrow$) Suppose $g = f \mid c$. Let $(c,b) \in g$. Since $g = f
\mid c$, it follows that $(c,b) \in f$. So, $g \subseteq f$.
($\Leftarrow$) Suppose $g \subseteq f$. Let $(c,b) \in g$. It follows
that $(c,b) \in f$. And since $c \in C$ and $b \in B$, it follows that
$(c,b) \in C \times B$. So, $(c,b) \in f \cap (C \times B)$. So, $g
\subseteq f \cap (C \times B)$. Let $(c,b) \in f \mid C$. It follows
that $c \in C$ and $b \in B$ and $(c,b) \in f$. We know that $g: C \to
B$. Since $c \in C$, there exists an element in $B$ which is equal to
$g(c)$. Since $g \subseteq f$, it follows that $g(c) = b$. So, $(c,b)
\in g$. Therefore, $f \mid c \subseteq g$.

\subsection{Solution (c)}
From $(b)$, it follows that $g = h\mid Z iff g \subseteq h$. Let
$(a,b)$ be arbitrary element in $g$. It follows that $a \in Z$ and $b
\in R$. Since $Z \subseteq R$, it follows that $a \in R$. From $(a,b)
\in g$, we know that $g(a) = b = 2a + 3$. From the definition of $h$,
it follows that $h(a) = 2a + 3$. So, $(a,2a + 3) \in h$. Since $b = 2a
+ 3$, $(a,b) \in h$. Since $(a,b)$ is arbitrary, it follows that $g
\subseteq h$.

\end{document}
