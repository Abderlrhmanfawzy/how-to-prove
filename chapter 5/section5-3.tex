%%%%%%%%%%%%%%%%%%%%%%%%%%%%%%%%%%%%%%%%%
% Author: Sibi <sibi@psibi.in>
%%%%%%%%%%%%%%%%%%%%%%%%%%%%%%%%%%%%%%%%%
\documentclass{article}
\usepackage{graphicx}
\usepackage{verbatim}
\usepackage{amsmath}
\usepackage{amsfonts}
\usepackage{amssymb}
\usepackage{tabularx}
\usepackage{mathtools}
\newcommand{\BigO}[1]{\ensuremath{\operatorname{O}\bigl(#1\bigr)}}
\setlength\parskip{\baselineskip}
\begin{document}
\title{Chapter 5 (Section 5.3)}
\author{Sibi}
\date{\today}
\maketitle

% See here: http://tex.stackexchange.com/a/43009/69223
\DeclarePairedDelimiter\abs{\lvert}{\rvert}%
\DeclarePairedDelimiter\norm{\lVert}{\rVert}%

% Swap the definition of \abs* and \norm*, so that \abs
% and \norm resizes the size of the brackets, and the 
% starred version does not.
\makeatletter
\let\oldabs\abs
\def\abs{\@ifstar{\oldabs}{\oldabs*}}
%
\let\oldnorm\norm
\def\norm{\@ifstar{\oldnorm}{\oldnorm*}}
\makeatother
\newpage

\section{Solution 1}
$R^{-1}(p) = $ the person immediately right to $p.$

\section{Solution 2}
$F^{-1}(X) = \text{unique } Y \text{ such that} F(Y) = X$.
So, \begin{align*}
      F(Y) = X \\
      A \setminus Y = X \\
      A \setminus X = Y \\ \\
      F^{-1}(X) = A \setminus X
    \end{align*}

\section{Solution 3}
We will try to find a function $g: R \to R$ such that $f \circ g =
i_\mathbb{R}$ and $g \circ f = i_\mathbb{R}$.

We are hoping to find $g = f^{-1}$. So, $f(x) = \text{unique }y$ such
that $f^{-1}(y) = x$. So, $\frac{2x + 5}{3} = y$. Solving it, we get
$x = \frac{3y-5}{2}$. So, $g(x) = \frac{3x-5}{2}$.

Let's check for $f \circ g = id$.
\begin{align*}
  f(g(x)) \\
  = f(\frac{3x-5}{2}) \\
  = \frac{3x - 5 + 5}{3} \\
  = x \\
  = id_{\mathbb{R}}(x)
\end{align*}

Let's check for $g \circ f = id $
\begin{align*}
  g(f(x)) \\
  = g(\frac{2x + 5}{3}) \\
  = \frac{2x + 5 - 5}{2} \\
  = x \\
  = id_{\mathbb{R}}(x)
\end{align*}

From theorem 5.3.4, we can conclude that $f$ is one-to-one and onto.

\section{Solution 4}
We will try to find a function $g: R \to R$ such that $f \circ g =
i_\mathbb{R}$ and $g \circ f = i_\mathbb{R}$.

We are hoping to find $g = f^{-1}$. So, $f(x) = \text{unique }y$ such
that $f^{-1}(y) = x$. So, $2x^3 - 3 = y$. Solving it, we get
$x = (\frac{3+y}{2})^{1/3}$. So, $g(x) = (\frac{3+x}{2})^{1/3}$.

Let's check for $f \circ g = id$.
\begin{align*}
  f \circ g(x) \\
  = f((\frac{3+x}{2})^{1/3}) \\
  = 3 + x - 3 \\
  = x \\
  = id_R \\
\end{align*}

Similarly, $g \circ f = id_R$

From theorem 5.3.4, we can conclude that $f$ is one-to-one and onto.

\section{Solution 5}
We will try to find a function $g: R^{+} \to R$ such that $f \circ g =
i_\mathbb{R^{+}}$ and $g \circ f = i_\mathbb{R}$.

We are hoping to find $g = f^{-1}$. So, $f(x) = \text{unique }y$ such
that $f^{-1}(y) = x$.So,
\begin{align*}
  10^{2-x} = y \\
  \log_{10}10^{2-x} = \log_{10}y \\
  2 - x = log_{10}y \\
  x = 2 - log y
\end{align*}

So, $g(x) = 2 - log(x)$. Let's verify $f \circ g = id_{\mathbb{R^{+}}}$

\begin{align*}
  f \circ g(x) \\
  = f(2 - log x) \\
  = 10^{2 + log x - 2} \\
  = 10^{log x} \\
  = x \\
  = id_{\mathbb{R^{+}}}
\end{align*}

Also, let's check $g \circ f = id_R$
\begin{align*}
  g \circ f(x) \\
  = g(10^{2 - x}) \\
  = 2 - log_{10} 10^{2-x}\\
  = 2 - (2 - x) \\
  = x \\
\end{align*}

From theorem 5.3.4, we can conclude that $f$ is one-to-one and onto.

\section{Solution 6}
\subsection{Solution 6(a)}
We will try to find a $g: B \to A$ such that $f \circ g = i_A$ and $g
\circ f = i_A$.

$f(x) = $ unique $y$ such that $f^{-1}(y) = x$.

So, $\frac{3x}{x-2} = y$ \\
Solving it, we get $f^{-1}(y) = \frac{2y}{y-3}$

No $g$ is undefined at $x = 3$. So, let the set $B$ be $R \setminus {3}$.

Now, let's verify $f \circ g = i_B$
\begin{align*}
  f \circ g(b) \\
  = f(g(b)) \\
  = f(\frac{2b}{b-3}) \\
  = \frac{6b}{b} \\
  = b \\
  = i_B(b)
\end{align*}

Also, $g \circ f = i_A$
\begin{align*}
  g \circ f(a) \\
  = g(f(a)) \\
  = g(\frac{3a}{a-2}) \\
  = \frac{6a}{a}
  = a \\
  = i_A(a)
\end{align*}

\subsection{Solution 6(b)}
From theorem $5.3.4$, it follows that $f^{-1}(x) = \frac{2x}{x-3}$

\section{Solution 7}
\subsection{Solution 7(a)}
Let $a,b$ be arbitrary element in $R$.

$(\Rightarrow)$ Suppose $(a,b) \in f$. It follows that
$b = a + \frac{7}{5}$. So, $(a,a+\frac{7}{5}) \in f$. Applying $a$ to
$f_2 \circ f_1$, we get $a + \frac{7}{5}$. So, $(a, a+\frac{7}{5}) \in
f_2 \circ f_1$. So, $f \subseteq f_2 \circ f_1$.
$(\Leftarrow)$ Similar to above.

\subsection{Solution 7(b)}
Applying $a$ to $f^{-1}$, we get $5a - 7$. Solving for inverse
functions, we get:
\begin{align*}
  f_1^{-1}(x) = x - 7 \\
  f_2^{-1}(x) = 5x
\end{align*}

Solving them:
\begin{align*}
  f_1^{-1}(f_2^{-1}(a)) \\
  = f_1^{-1}(51) \\
  = 5a - 7
\end{align*}

So, yes they are same!

\section{Solution 8}
\subsection{Solution (a)}
Let $b$ be arbitrary element of $B$. Let $a = f^{-1}(b) \in A$. Then
$(b,a) \in f^{-1}$. Thus,
\begin{align*}
  f \circ f^{-a}(b) \\
  = f(f^{-1}(b)) \\
  = f(a) \\
  = b \\
  = i_B(b)
\end{align*}

Since $b$ was arbitrary, we can conclude that $f \circ f^{-1} = i_B$.

\subsection{Solution (b)}
We know that,
\begin{align*}
  f^{-1} \circ f = i_A \\
  f^{-1} \circ f \circ f^{-1} = i_A \circ f^{-1} 
\end{align*}

Let $b$ be arbitrary element in $B$. Suppose $f(a) = b$. Then $(a,b)
\in f$ and $(b,a) \in f^{-1}$. So,
\begin{align*}
  f^{-1} \circ (f \circ f^{-1})(b) = i_A \circ f^{-1}(b) \\
  f^{-1}(b) = i_A(a) \\
  f^{-1}(b) = a \\
  f \circ f^{-1}(b) = f(a) \tag{Apply f} \\
  f \circ f^{-a} (b) = b \\
  f \circ f^{-a} (b) = i_B(b)
\end{align*}
Since $b$ was arbitrary, we can conclude that $f \circ f^{-1} = i_B$.

\section{Solution 9}
Suppose $g: B \to A$ and $f \circ g = i_B$. Let $b$ be arbitrary
element in $B$. Then $f \circ g(b) = i_B(b)$. Since $g$ is a function
$\exists a \in A$ such that $g(b) = a$. So, $f(a) = b$. Since $b$ was
arbitrary, we can conclude that $f$ is onto.

\section{Solution 10}
Suppose $f: A \to B, g: B \to A, g \circ f = i_A$ and $f \circ g =
i_B$. Let $(b,a)$ be arbitrary element of $B \times A$ such that
$(b,a) \in g$. Then $f \circ g(b) = f(a)$. From $f \circ g = i_B$, it
follows that $f(a) = b$. So, $(a,b) \in f^{-1}$. So, $g \subseteq
f^{-1}$. Similarly, do the other case with $g \circ f = i_A$ and show
$f^{-1} \subseteq g$.

\section{Solution 11}
\subsection{Solution (a)}
Suppose $f:A \to B$ and $g:B \to A$. Suppose $f$ is one to one and $f
\circ g = i_B$. From theorem $5.3.3$, it follows that $f$ is onto. Now
from $5.3.4$, it follows that $g \circ f = i_A$. Now from theorem
$5.3.5$, we can conclude that $g = f^{-1}$.

\subsection{Solution (b)}
Suppose $f:A \to B$ and $g:B \to A$. Suppose $f$ is onto and $g \circ
f = i_A$. From theorem, $5.3.3$, it follows that $f$ is one to one.
From $5.3.4$, it follows that $f \circ g = i_B$. From $5.3.5$, it
follows that $g = f^{-1}$

\subsection{Solution (c)}
Suppose $f \circ g = i_B$ and $g \circ f \neq i_A$.

\subsubsection{Solution (i)}
From theorem $5.3.3$, it follows that $f$ is onto.

\subsubsection{Solution (ii)}
We know that $g \circ f \neq i_A$. So, $g \neq f^{-1}$. Applying
contrapositive law to $(a)$, we get $g \neq f^{-1} \implies f \circ g
\neq i_B \lor f \text{ is not one to one}$. We know that $f \circ g =
i_B$. So $f$ is not one to one.

Rest of the proofs are based on the above technique.
\end{document}
