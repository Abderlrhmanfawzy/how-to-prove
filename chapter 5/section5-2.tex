%%%%%%%%%%%%%%%%%%%%%%%%%%%%%%%%%%%%%%%%%
% Author: Sibi <sibi@psibi.in>
%%%%%%%%%%%%%%%%%%%%%%%%%%%%%%%%%%%%%%%%%
\documentclass{article}
\usepackage{graphicx}
\usepackage{verbatim}
\usepackage{amsmath}
\usepackage{amsfonts}
\usepackage{amssymb}
\usepackage{tabularx}
\usepackage{mathtools}
\newcommand{\BigO}[1]{\ensuremath{\operatorname{O}\bigl(#1\bigr)}}
\setlength\parskip{\baselineskip}
\begin{document}
\title{Chapter 5 (Section 5.2)}
\author{Sibi}
\date{\today}
\maketitle

% See here: http://tex.stackexchange.com/a/43009/69223
\DeclarePairedDelimiter\abs{\lvert}{\rvert}%
\DeclarePairedDelimiter\norm{\lVert}{\rVert}%

% Swap the definition of \abs* and \norm*, so that \abs
% and \norm resizes the size of the brackets, and the 
% starred version does not.
\makeatletter
\let\oldabs\abs
\def\abs{\@ifstar{\oldabs}{\oldabs*}}
%
\let\oldnorm\norm
\def\norm{\@ifstar{\oldnorm}{\oldnorm*}}
\makeatother
\newpage

\section{Solution 1}
One-to-one: (c) \\
Onto: (a)

\section{Solution 2}
One to one: (c) \\
Onto: (b)-function g, (c)

\section{Solution 3}
One to one: No one \\
Onto: (a), (b), (c)

\section{Solution 4}
One to one: (a), (b), (c) \\
Onto: (b), (c)

\section{Solution 5}
\subsection{Solution 5 (a)}
Let $a_1$ and $a_2$ be arbitrary element in $A$. Suppose $f(a_1) =
f(a_2)$. It follows that $(a_1 + 1)(a_2 - 1) = (a_2 + 1)(a_1 - 1)$.
Solving it, we get $a_1 = a_2$. Since $a_1$ and $a_2$ are arbitrary we
can conclude that $f$ is one to one.

Let $a_1$ be arbitrary element in $A$. Let $a_2 = a_1 + 1 / a_1 - 1$.
Then it follows that $f(a_2) = f(a_1)$. So, $\forall a_1 \in A \exists
a_2 \in A(f(a_2) = f(a_1))$.

\subsection{Solution 5 (b)}
$(\Rightarrow)$ Suppose $(a_1, a_2) \in f \circ f$. Then it follows
$\exists a_3 \in A (a_1, a_2) \in f$ and $(a_3,a_2) \in f$. We know
that $f(x) = \frac{x + 1}{x - 1}$. So, $a_3 = \frac{a_1 + 1}{a_1 -
  1}$. Solving it, we get $a_1 = \frac{a_3 + 1}{a_1 - 1}$. Similarly
from $(a_3, a_2)$ we get, $a_2 = \frac{a_3 + 1}{a_1 - 1}$. So, $a_1 =
a_2$. Since $a_1 = a_2$, it follows that $(a_1, a_2) \in i_A$. So, $f
\circ f \subseteq i_A$.

$(\Leftarrow)$ Suppose $(a_1, a_2) \in i_A$. Then it follows that $a_1
= a_2$. Since $a_1 \in A$, it follows that $(a_1, \frac{a_1 + 1}{a_1 -
  1}) \in f$. Deducing, we get $(\frac{a_1 + 1}{a_1 - 1}, a_1) \in f$.
So, $(a_1, a_2) \in f \circ f$. Since $a_1$ and $a_2$ are arbitrary,
it follows that $i_A \subseteq f \circ f$.

\section{Solution 6}
\subsection{Solution 6 (a)}
$f(2) = \{y \in R \mid y^2 < 2\}$

\subsection{Solution 6 (b)}
\subsubsection{Proof for one-to-one}
Let $r_1, r_2$ be arbitrary element in $R$ such that $f(r_1) =
f(r_2)$. So, $\{y \in R \mid y^2 < r_1\} = \{y \in R \mid y^2 < r_2\}$
. Now those two sets can be equivalent only if $r_1 = r_2$
\subsubsection{Proof that it is not onto}
Suppose $a = \{1,2\}$. such that $f(r) = a$ for some $r$. It follows
that $\{y \in R \mid y^2 < r\} = \{1,2\}$. But for any $x \in R$,
$f(x) \neq \{1,2\}$.

\section{Solution 7}
\subsection{Solution 7(a)}
$\{1,2,3,4\}$

\subsection{Solution 7(b)}
\subsubsection{Proof that it is not one to one}
Suppose $b_1 = \{\{1,2\}, \{3,4\}\}$ and $b_2 = \{\{1,2,3\}, \{4\}\}$.
It follows that $f(b_1) = f(b_2)$. But $b_1 \neq b_2$. So, $f$ is not
one to one.

\subsubsection{Proof for onto}
Let $a$ be an arbitrary element in $A$. Suppose $b = \{\{a\}\}$. Then
$f(b) = \cup{b} = a$. Since $a$ was arbitrary, we can conclude that
$f$ is onto.

\section{Solution 8}
\subsection{Solution 8 (a)}
Suppose $g \circ f$ is onto. Then it follows that $\forall c \in C
\exists a \in A(g \circ f(a) = c)$. Let $c_1$ be arbitrary element in
$C$. From universal instantiation, it follows that $\exists a \in A(g
\circ f(a) = c_1)$. So there is $\exists a \in A$ such that $g \circ
f(a) = c_1$. Since $f$ is a function, let $f(a) = b$ for some $b \in
B$. Then it follows that $g(b) = c_1$.

\subsection{Solution 8 (b)}
Suppose $g \circ f$ is one to one. Let $a_1, a_2$ be arbitrary element
of $A$. Suppose $f(a_1) = f(a_2)$. We know that $g \circ f(a_1) = g
\circ f(a_2)$, So, $a_1 = a_2$. Since $a_1$ and $a_2$ are arbitrary,
we can conclude that $f$ is one-to-one.

\end{document}
