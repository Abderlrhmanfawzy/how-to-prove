%%%%%%%%%%%%%%%%%%%%%%%%%%%%%%%%%%%%%%%%%
% Author: Sibi <sibi@psibi.in>
%%%%%%%%%%%%%%%%%%%%%%%%%%%%%%%%%%%%%%%%%
\documentclass{article}
\usepackage{graphicx}
\usepackage{verbatim}
\usepackage{amsmath}
\usepackage{amsfonts}
\usepackage{amssymb}
\usepackage{tabularx}
\usepackage{mathtools}
\newcommand{\BigO}[1]{\ensuremath{\operatorname{O}\bigl(#1\bigr)}}
\setlength\parskip{\baselineskip}
\begin{document}
\title{Chapter 5 (Section 5.2)}
\author{Sibi}
\date{\today}
\maketitle

% See here: http://tex.stackexchange.com/a/43009/69223
\DeclarePairedDelimiter\abs{\lvert}{\rvert}%
\DeclarePairedDelimiter\norm{\lVert}{\rVert}%

% Swap the definition of \abs* and \norm*, so that \abs
% and \norm resizes the size of the brackets, and the 
% starred version does not.
\makeatletter
\let\oldabs\abs
\def\abs{\@ifstar{\oldabs}{\oldabs*}}
%
\let\oldnorm\norm
\def\norm{\@ifstar{\oldnorm}{\oldnorm*}}
\makeatother
\newpage

\section{Solution 1}
One-to-one: (c) \\
Onto: (a)

\section{Solution 2}
One to one: (c) \\
Onto: (b)-function g, (c)

\section{Solution 3}
One to one: No one \\
Onto: (a), (b), (c)

\section{Solution 4}
One to one: (a), (b), (c) \\
Onto: (b), (c)

\section{Solution 5}
\subsection{Solution 5 (a)}
Let $a_1$ and $a_2$ be arbitrary element in $A$. Suppose $f(a_1) =
f(a_2)$. It follows that $(a_1 + 1)(a_2 - 1) = (a_2 + 1)(a_1 - 1)$.
Solving it, we get $a_1 = a_2$. Since $a_1$ and $a_2$ are arbitrary we
can conclude that $f$ is one to one.

Let $a_1$ be arbitrary element in $A$. Let $a_2 = a_1 + 1 / a_1 - 1$.
Then it follows that $f(a_2) = f(a_1)$. So, $\forall a_1 \in A \exists
a_2 \in A(f(a_2) = f(a_1))$.
\end{document}
