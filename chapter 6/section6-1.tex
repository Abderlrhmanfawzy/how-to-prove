%%%%%%%%%%%%%%%%%%%%%%%%%%%%%%%%%%%%%%%%%
% Author: Sibi <sibi@psibi.in>
%%%%%%%%%%%%%%%%%%%%%%%%%%%%%%%%%%%%%%%%%
\documentclass{article}
\usepackage{graphicx}
\usepackage{verbatim}
\usepackage{amsmath}
\usepackage{amsfonts}
\usepackage{amssymb}
\usepackage{tabularx}
\usepackage{mathtools}
\newcommand{\BigO}[1]{\ensuremath{\operatorname{O}\bigl(#1\bigr)}}
\setlength\parskip{\baselineskip}
\begin{document}
\title{Chapter 6 (Section 6.1)}
\author{Sibi}
\date{\today}
\maketitle

% See here: http://tex.stackexchange.com/a/43009/69223
\DeclarePairedDelimiter\abs{\lvert}{\rvert}%
\DeclarePairedDelimiter\norm{\lVert}{\rVert}%

% Swap the definition of \abs* and \norm*, so that \abs
% and \norm resizes the size of the brackets, and the 
% starred version does not.
\makeatletter
\let\oldabs\abs
\def\abs{\@ifstar{\oldabs}{\oldabs*}}
%
\let\oldnorm\norm
\def\norm{\@ifstar{\oldnorm}{\oldnorm*}}
\makeatother
\newpage

\section{Solution 1}
By mathematical induction, \\
Base case. When $n = 0$, $0 = \frac{n(n+1)}{2} = 0$.
Induction step. Let $n$ be an arbitrary element and suppose $0 + 1 + 2
+ ... + n = \frac{n(n+1)}{2}$. Then,
\begin{align*}
  0 + 1 + 2 + ... + n + (n + 1) = \frac{n(n+1)}{2} + (n + 1) \\
  = (n+1)(\frac{n}{2} + 1) \\
  = \frac{(n+1)(n+2)}{2}
\end{align*}

\section{Solution 2}
By mathematical induction,

Base case. When $n = 0$, both sides of the equation becomes $0$.

Induction step. Let $n$ be arbitrary and suppose $0^2 + 1^2 + 2^2 +
... + n^2 = \frac{n(n+1)(2n+1)}{6}$. Then,
\begin{align*}
  0^2 + 1^2 + 2^2 + ... + n^2 + (n+1)^2 = \frac{n(n+1)(n+2)}{6} + (n +
  1)^2 \\
  = (n+1)(\frac{n(2n + 1)}{6} + (n + 1)) \\
  = (n+1)(\frac{2n^2 + n + 6n + 6}{6}) \\
  = \frac{(n+1)(n+2)(n+3)}{6} \\
\end{align*}
\section{Solution 3}
By mathematical induction,

Base case. When $n=0$, both sides of the equation becomes $0$.

Induction step. Let $n$ be an arbitrary element in $N$. Suppose $0^3 +
1^3 + 2^3 + ... + n^3 = [\frac{n(n+1)}{2}]^2$. Then,

\begin{align*}
  0^3 + 1^3 + 2^3 + ... + n^3 + (n+1)^3= [\frac{n(n+1)}{2}]^2 + (n+1)^3 \\
  = (n+1)^2(\frac{n^2 + 4n + 4}{4}) \\
  = \frac{(n+1)^2(n+2)^2}{4} \\
  = [\frac{(n+1)(n+2)}{2}]^2 
\end{align*}

\section{Solution 4}
By mathematical induction,

Base case. When $n=1$, both sides of the equation becomes 1.

Induction step. Let $n$ be an arbitrary element and suppose $1+3+5+...
+ (2n-1) = n^2$. Then,

\begin{align*}
  1 + 3 + 5 + ... + (2n - 1) = n^2 \\
  1 + 3 + 5 + ... + (2n - 1) + (n+1)^2 = n^2 + (n+1)^2 \\
  1 + 3 + 5 + ... + (2n - 1) + n^2 + 1 + 2n = n^2 + (n+1)^2 \\
  1 + 3 + 5 + ... + (2n + 1) = (n+1)^2.
\end{align*}

\section{Solution 5}
By mathematical induction,

Base case. When $n=0$, both sides of the equation becomes $0$.

Inductive step. Let $n$ be an arbitrary element and suppose $0.1 + 1.2
+ 2.3 + ... + n(n+1) = \frac{n(n+1)(n+2)}{3}$. Then,

\begin{align*}
  0.1 + 1.2 + 2.3 + ... + n(n+1) + (n+1)(n+2) = \frac{n(n+1)(n+2)}{3}
  + (n+1)(n+2) \\
  = (n+1)(n+2)(\frac{n}{3} + 1) \\
  = \frac{(n+1)(n+2)(n+3)}{3}
\end{align*}
\section{Solution 6}
Guess: $\frac{n(n+1)(n+2)(n+3)}{4}$

By mathematical induction,

Base case. When $n=0$, both sides of the equation becomes $0$.

Induction step. Let $n$ be an arbitrary element. Suppose $0.1.2 +
1.2.3 + 2.3.4 + ... + n(n+1)(n+2) = \frac{n(n+1)(n+2)(n+3)}{4}$. Then,

\begin{align*}
  0.1.2 + 1.2.3 + 2.3.4 + ... + n(n+1)(n+2) + (n+1)(n+2)(n+3) =
  \frac{n(n+1)(n+2)(n+3)}{4} +  (n+1)(n+2)(n+3) \\
  = (n+1)(n+2)(n+3)(\frac{n}{4} + 1) \\
  = \frac{(n+1)(n+2)(n+3)(n+4)}{4}
\end{align*}

\end{document}
