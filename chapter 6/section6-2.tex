%%%%%%%%%%%%%%%%%%%%%%%%%%%%%%%%%%%%%%%%%
% Author: Sibi <sibi@psibi.in>
%%%%%%%%%%%%%%%%%%%%%%%%%%%%%%%%%%%%%%%%%
\documentclass{article}
\usepackage{graphicx}
\usepackage{verbatim}
\usepackage{amsmath}
\usepackage{amsfonts}
\usepackage{amssymb}
\usepackage{tabularx}
\usepackage{mathtools}
\newcommand{\BigO}[1]{\ensuremath{\operatorname{O}\bigl(#1\bigr)}}
\setlength\parskip{\baselineskip}
\begin{document}
\title{Chapter 6 (Section 6.2)}
\author{Sibi}
\date{\today}
\maketitle

% See here: http://tex.stackexchange.com/a/43009/69223
\DeclarePairedDelimiter\abs{\lvert}{\rvert}%
\DeclarePairedDelimiter\norm{\lVert}{\rVert}%

% Swap the definition of \abs* and \norm*, so that \abs
% and \norm resizes the size of the brackets, and the 
% starred version does not.
\makeatletter
\let\oldabs\abs
\def\abs{\@ifstar{\oldabs}{\oldabs*}}
%
\let\oldnorm\norm
\def\norm{\@ifstar{\oldnorm}{\oldnorm*}}
\makeatother
\newpage

\section{Solution 1}


$R' = R \cap (A' \times A')$
$A' = A \setminus \{a\}$

\subsection{Solution (a)}
Reflexive.

Let $b$ be an arbitrary element on $A'$. Since $A' \subseteq A$, it
follows that $(b,b) \in R$ since $R$ is reflexive. Also,
$(b,b) \in A' \times A'$. So, $(b,b) \in R \cap (A' \times A')$. Since
$b$ is arbitrary we can conclude that $R$ is reflexive.

Transitive.

Let $b,c,d$ be arbitrary element on $A'$ such that $(b,c) \in R'$ and
$(c,d) \in R'$. Since $R$ is transitive, it follows that
$(b,d) \in R$. Also, $(b,d) \in A' \times A'$. So,
$(b,d) \in R \cap (A' \times A')$.

Anti-symmetric.

Let $x,y$ be arbitrary element on $A'$ such that $xR'y$ and $yR'x$.
Since $R$ is anti-symmetric, it follows that $x=y$.

\end{document}
