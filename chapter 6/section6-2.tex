%%%%%%%%%%%%%%%%%%%%%%%%%%%%%%%%%%%%%%%%%
% Author: Sibi <sibi@psibi.in>
%%%%%%%%%%%%%%%%%%%%%%%%%%%%%%%%%%%%%%%%%
\documentclass{article}
\usepackage{graphicx}
\usepackage{verbatim}
\usepackage{amsmath}
\usepackage{amsfonts}
\usepackage{amssymb}
\usepackage{tabularx}
\usepackage{mathtools}
\newcommand{\BigO}[1]{\ensuremath{\operatorname{O}\bigl(#1\bigr)}}
\setlength\parskip{\baselineskip}
\begin{document}
\title{Chapter 6 (Section 6.2)}
\author{Sibi}
\date{\today}
\maketitle

% See here: http://tex.stackexchange.com/a/43009/69223
\DeclarePairedDelimiter\abs{\lvert}{\rvert}%
\DeclarePairedDelimiter\norm{\lVert}{\rVert}%

% Swap the definition of \abs* and \norm*, so that \abs
% and \norm resizes the size of the brackets, and the 
% starred version does not.
\makeatletter
\let\oldabs\abs
\def\abs{\@ifstar{\oldabs}{\oldabs*}}
%
\let\oldnorm\norm
\def\norm{\@ifstar{\oldnorm}{\oldnorm*}}
\makeatother
\newpage

\section{Solution 1}


$R' = R \cap (A' \times A')$
$A' = A \setminus \{a\}$

\subsection{Solution (a)}
Reflexive.

Let $b$ be an arbitrary element on $A'$. Since $A' \subseteq A$, it
follows that $(b,b) \in R$ since $R$ is reflexive. Also,
$(b,b) \in A' \times A'$. So, $(b,b) \in R \cap (A' \times A')$. Since
$b$ is arbitrary we can conclude that $R$ is reflexive.

Transitive.

Let $b,c,d$ be arbitrary element on $A'$ such that $(b,c) \in R'$ and
$(c,d) \in R'$. Since $R$ is transitive, it follows that
$(b,d) \in R$. Also, $(b,d) \in A' \times A'$. So,
$(b,d) \in R \cap (A' \times A')$.

Anti-symmetric.

Let $x,y$ be arbitrary element on $A'$ such that $xR'y$ and $yR'x$.
Since $R$ is anti-symmetric, it follows that $x=y$.

\subsection{Solution (b)}
Reflexive.

Let $x$ be an arbitrary element in $A$. Then either $x \in A'$ or
$x \in \{a\}$. Let us consider the cases:

Case 1. $x \in A'$ Since $T'$ is a total order in $A'$, it follows
that $(x,x) \in T$.
Case 2. $x \in \{a\}$. Then $x = a$. It follows that
$(a,a) \in \{a\} \times A$. So, $(x,x) \in T$.

Transitive.

Let $x,y$ and $z$ be arbitrary element in $A$ such that $(x,y) \in T$
and $(y,z) \in T$. Let us consider the cases.

Case 1. $x \neq a$. Then $x \in A'$. Since
$T = T' \cup (\{a\} \times A)$, it follows that $y \in A'$ and
$z \in A'$. Since $T'$ is transitive, it follows that $(x,z) \in T$.
Case 2. $x = a$ Since $z \in A$, it follows that
$(a,z) \in \{a\} \times A$. So, $(x,z) \in T$.

Anti-symmetric

Let $x,y$ be arbitrary element on $A$ such that $xTy$ and $yTx$. Let
us consider the cases:
Case 1. $x \neq a$. Then $x \in A'$, $y \in A'$ since $(x,y) \in T'
\cup ({a} \times A)$ and $x \in A'$. So, $(x,y) \in T'$. Similarly,
$(y,x) \in T'$. Since $T'$ is a total order on $A'$, it follows that
$x = y$.
Case 2. $x = a$. So, $(y,a) \in T$. Therefore $(y,a) \in T' \cup \{a\}
\times A$. Now $T'$ is a relation on $A'$ which doesn't have $a$. So,
$(y,a) \notin T'$. So, $(y,a) \in \{a\} \times A$. Now, clearly $y =
a$. So, $x = y$.

$\forall x \in A \forall y \in A (xTy \lor yTx)$

Let $x,y$ be arbitrary element on $A$. Let us consider the cases:
Case 1. $x = a$. Since $y \in A$, it follows that $(a,y) \in \{a\}
\times A$. So, $xTy \lor yTx$.
Case 2. $x \neq a$. Now if $y = a$, then $xRa$ will contradict the $R$
minimality of $A$. If $y \neq a$, then $y \in A'$. So, $(x,y) \in R'$.
Since $R' \subseteq T' \subseteq T$, it follows that $(x,y) \in T$.

\section{Solution 3}
We will show by induction that for every natural number $n >= 1$,
every subset of $A$ with $n$ elements has a smallest element.

Base case. When $n =1$, then the single element present in it is the
smallest element.

Induction step. Suppose $n >= 1$ and suppose that every subset of $A$
has a smallest element. Now let $B$ be arbitrary subset of $A$ with
$n+1$ elements. Let $b$ be an arbitrary element of $B$ and let
$B' = B \setminus \{b\}$ with n elements. By inductive hypothesis,
$B'$ has a smallest element $c \in B'$. Let us consider the cases.

Case 1. $bRc$ We know that $\forall x \in B'(cRx)$ and $B' = B
\setminus \{b\}$.
Now from $bRc$, it is clear that $b$ is the smallest element. (Note
that $R$ is transitive).

Case 2. $cRb$ Then it follows that $\forall x \in B(cRx)$. So, $c$ is
the smallest element.

\section{Solution 4}
\subsection{Solution (a)}
We will show by mathematical induction that $\forall n >= 1 \forall B
\subseteq A (\forall x \in A \forall y \in A(xRy \lor yRx) \implies
\exists z \in B \forall y \in B((z,y) \in R \circ R))$.

Base case. When $n=1$, then $B = \{b\}$ for some $b \in A$. Since
$(b,b) \in R \circ R$, base case holds.

Induction step. Suppose $n >= 1$ and suppose that every subset of $A$
has some elements $z \in B$ such that $\forall y \in B((z,y) \in R
\circ R)$. Now let $C$ be arbitrary subset of $A$ with $n+1$ elements.
Now let $c$ be some element in $C$ and let $C' = C \setminus \{c\}$.
By inductive hypothesis, $\exists z \in C' \forall y \in C'((z,y) \in
R \circ R)$.
Case 1. $(z,c) \in R \circ R$. Then $\forall y \in C((z,y) \in R \circ
R)$.
Case 2. $(z,c) \notin R \circ R$. Then we will try to prove that
$\forall y \in C((c,y) \in R \circ R)$. Let $y$ be arbitrary element
in $C$. If $y = c$, then since $R$ is reflexive, $(c,y) \in R \circ
R$. Now, if $y \neq c$, then $y \in C'$. So, $(z,y) \in R \circ R$.
This means $\exists x \in A such that (z,x) \in R$ and $(x,y) \in R$.
Now if $(z,x) \in R$and $(x,c) \in R$, then $(z,c) \in R \circ R$
contrary to assumption. So, $(x,c) \notin R$. From the hypothesis on
$R$, $(c,x) \in R$. From $(c,x) \in R$ and $(x,y) \in R$, it follows
that $(c,y) \in R \circ R$.

\subsection{Solution (b)}
\begin{align*}
  A = \text{Set of all contestants} \\
  R = \{(x,y) \in A \times A \mid x beats y \} 
\end{align*}
Now $R$ satisfies this: $\forall x \in A \forall y \in A(xRy \lor
yRx)$. Now from $(a)$, it follows that $B \subseteq A$, $\exists x \in
B(\forall y \in B((x,y) \in R \circ R))$. Let $B=A$, then $(x,y) \in R
\circ R$. So there is at least one successful player.

\section{Solution 5}
By mathematical induction,

Base case. When $n=1$, $F_1 = F_o.F_1.F_2$. Both the sides of the
equation becomes $5$ and so the base case holds.

Induction step. Let $n$ be arbitrary element and suppose $F_n =
2^{(2^n)} + 1$. We have to prove that $F_{n+1} = F_0.F_1.F_2 ... F_n +
2$. By inductive hypothesis,
\begin{align*}
  F_n = F_0.F_1.F_2 ... F_{n-1} + 2 \\
  F_0.F_1.F_2 ... F_{n-1}.F_n = (F_n - 2)F_n \\
  F_0.F_1.F_2 ... F_{n-1}.F_n + 2 = (F_n - 2)F_n + 2 \\
  = (F_n)^2 - 2F_n + 2 \\
  = (2^{(2^n)} + 1)^2 - 2(2^{2^n} + 1) + 2 \\
  = 2^{(2^n)} + 1 \\
  = F_n + 1
\end{align*}

\section{Solution 6}
By mathematical induction,

Base case. When $n=1$, $|a_1| = |a_1|$ so base case holds.

Induction step. Let $n$ be arbitrary element and suppose $a_1 + a_2 +
... + a_n <= |a_1| + |a_2| + ... + |a_n|$. Let $a_{n+1}$ be arbitrary
real number. Let us consider the cases.

Case 1. $a_{n+1}$ is positive: From our inductive hypothesis, it
follows that $a_1 + a_2 + ... + a_n + a_{n+1} <= |a_1| + |a_2| + ... +
|a_n| + |a_{n+1}|$.

Case 2. $a_{n+1}$ is negative: Then it follows that
$a_1 + a_2 + ... + a_n + a_{n+1} < |a_1| + |a_2| + ... + |a_n| +
|a_{n+1}|$. So,
$a_1 + a_2 + ... + a_n + a_{n+1} <= |a_1| + |a_2| + ... + |a_n| +
|a_{n+1}|$.

\end{document}
