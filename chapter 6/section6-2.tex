%%%%%%%%%%%%%%%%%%%%%%%%%%%%%%%%%%%%%%%%%
% Author: Sibi <sibi@psibi.in>
%%%%%%%%%%%%%%%%%%%%%%%%%%%%%%%%%%%%%%%%%
\documentclass{article}
\usepackage{graphicx}
\usepackage{verbatim}
\usepackage{amsmath}
\usepackage{amsfonts}
\usepackage{amssymb}
\usepackage{tabularx}
\usepackage{mathtools}
\newcommand{\BigO}[1]{\ensuremath{\operatorname{O}\bigl(#1\bigr)}}
\setlength\parskip{\baselineskip}
\begin{document}
\title{Chapter 6 (Section 6.2)}
\author{Sibi}
\date{\today}
\maketitle

% See here: http://tex.stackexchange.com/a/43009/69223
\DeclarePairedDelimiter\abs{\lvert}{\rvert}%
\DeclarePairedDelimiter\norm{\lVert}{\rVert}%

% Swap the definition of \abs* and \norm*, so that \abs
% and \norm resizes the size of the brackets, and the 
% starred version does not.
\makeatletter
\let\oldabs\abs
\def\abs{\@ifstar{\oldabs}{\oldabs*}}
%
\let\oldnorm\norm
\def\norm{\@ifstar{\oldnorm}{\oldnorm*}}
\makeatother
\newpage

\section{Solution 1}


$R' = R \cap (A' \times A')$
$A' = A \setminus \{a\}$

\subsection{Solution (a)}
Reflexive.

Let $b$ be an arbitrary element on $A'$. Since $A' \subseteq A$, it
follows that $(b,b) \in R$ since $R$ is reflexive. Also,
$(b,b) \in A' \times A'$. So, $(b,b) \in R \cap (A' \times A')$. Since
$b$ is arbitrary we can conclude that $R$ is reflexive.

Transitive.

Let $b,c,d$ be arbitrary element on $A'$ such that $(b,c) \in R'$ and
$(c,d) \in R'$. Since $R$ is transitive, it follows that
$(b,d) \in R$. Also, $(b,d) \in A' \times A'$. So,
$(b,d) \in R \cap (A' \times A')$.

Anti-symmetric.

Let $x,y$ be arbitrary element on $A'$ such that $xR'y$ and $yR'x$.
Since $R$ is anti-symmetric, it follows that $x=y$.

\subsection{Solution (b)}
Reflexive.

Let $x$ be an arbitrary element in $A$. Then either $x \in A'$ or
$x \in \{a\}$. Let us consider the cases:

Case 1. $x \in A'$ Since $T'$ is a total order in $A'$, it follows
that $(x,x) \in T$.
Case 2. $x \in \{a\}$. Then $x = a$. It follows that
$(a,a) \in \{a\} \times A$. So, $(x,x) \in T$.

Transitive.

Let $x,y$ and $z$ be arbitrary element in $A$ such that $(x,y) \in T$
and $(y,z) \in T$. Let us consider the cases.

Case 1. $x \neq a$. Then $x \in A'$. Since
$T = T' \cup (\{a\} \times A)$, it follows that $y \in A'$ and
$z \in A'$. Since $T'$ is transitive, it follows that $(x,z) \in T$.
Case 2. $x = a$ Since $z \in A$, it follows that
$(a,z) \in \{a\} \times A$. So, $(x,z) \in T$.

Anti-symmetric

Let $x,y$ be arbitrary element on $A$ such that $xTy$ and $yTx$. Let
us consider the cases:
Case 1. $x \neq a$. Then $x \in A'$, $y \in A'$ since $(x,y) \in T'
\cup ({a} \times A)$ and $x \in A'$. So, $(x,y) \in T'$. Similarly,
$(y,x) \in T'$. Since $T'$ is a total order on $A'$, it follows that
$x = y$.
Case 2. $x = a$. So, $(y,a) \in T$. Therefore $(y,a) \in T' \cup \{a\}
\times A$. Now $T'$ is a relation on $A'$ which doesn't have $a$. So,
$(y,a) \notin T'$. So, $(y,a) \in \{a\} \times A$. Now, clearly $y =
a$. So, $x = y$.

$\forall x \in A \forall y \in A (xTy \lor yTx)$

Let $x,y$ be arbitrary element on $A$. Let us consider the cases:
Case 1. $x = a$. Since $y \in A$, it follows that $(a,y) \in \{a\}
\times A$. So, $xTy \lor yTx$.
Case 2. $x \neq a$. Now if $y = a$, then $xRa$ will contradict the $R$
minimality of $A$. If $y \neq a$, then $y \in A'$. So, $(x,y) \in R'$.
Since $R' \subseteq T' \subseteq T$, it follows that $(x,y) \in T$.

\section{Solution 3}
We will show by induction that for every natural number $n >= 1$,
every subset of $A$ with $n$ elements has a smallest element.

Base case. When $n =1$, then the single element present in it is the
smallest element.

Induction step. Suppose $n >= 1$ and suppose that every subset of $A$
has a smallest element. Now let $B$ be arbitrary subset of $A$ with
$n+1$ elements. Let $b$ be an arbitrary element of $B$ and let
$B' = B \setminus \{b\}$ with n elements. By inductive hypothesis,
$B'$ has a smallest element $c \in B'$. Let us consider the cases.

Case 1. $bRc$ We know that $\forall x \in B'(cRx)$ and $B' = B
\setminus \{b\}$.
Now from $bRc$, it is clear that $b$ is the smallest element. (Note
that $R$ is transitive).

Case 2. $cRb$ Then it follows that $\forall x \in B(cRx)$. So, $c$ is
the smallest element.

\section{Solution 4}
\subsection{Solution (a)}
We will show by mathematical induction that $\forall n >= 1 \forall B
\subseteq A (\forall x \in A \forall y \in A(xRy \lor yRx) \implies
\exists z \in B \forall y \in B((z,y) \in R \circ R))$.

Base case. When $n=1$, then $B = \{b\}$ for some $b \in A$. Since
$(b,b) \in R \circ R$, base case holds.

Induction step. Suppose $n >= 1$ and suppose that every subset of $A$
has some elements $z \in B$ such that $\forall y \in B((z,y) \in R
\circ R)$. Now let $C$ be arbitrary subset of $A$ with $n+1$ elements.
Now let $c$ be some element in $C$ and let $C' = C \setminus \{c\}$.
By inductive hypothesis, $\exists z \in C' \forall y \in C'((z,y) \in
R \circ R)$.
Case 1. $(z,c) \in R \circ R$. Then $\forall y \in C((z,y) \in R \circ
R)$.
Case 2. $(z,c) \notin R \circ R$. Then we will try to prove that
$\forall y \in C((c,y) \in R \circ R)$. Let $y$ be arbitrary element
in $C$. If $y = c$, then since $R$ is reflexive, $(c,y) \in R \circ
R$. Now, if $y \neq c$, then $y \in C'$. So, $(z,y) \in R \circ R$.
This means $\exists x \in A such that (z,x) \in R$ and $(x,y) \in R$.
Now if $(z,x) \in R$and $(x,c) \in R$, then $(z,c) \in R \circ R$
contrary to assumption. So, $(x,c) \notin R$. From the hypothesis on
$R$, $(c,x) \in R$. From $(c,x) \in R$ and $(x,y) \in R$, it follows
that $(c,y) \in R \circ R$.

\subsection{Solution (b)}
\begin{align*}
  A = \text{Set of all contestants} \\
  R = \{(x,y) \in A \times A \mid x beats y \} 
\end{align*}
Now $R$ satisfies this: $\forall x \in A \forall y \in A(xRy \lor
yRx)$. Now from $(a)$, it follows that $B \subseteq A$, $\exists x \in
B(\forall y \in B((x,y) \in R \circ R))$. Let $B=A$, then $(x,y) \in R
\circ R$. So there is at least one successful player.

\section{Solution 5}
By mathematical induction,

Base case. When $n=1$, $F_1 = F_o.F_1.F_2$. Both the sides of the
equation becomes $5$ and so the base case holds.

Induction step. Let $n$ be arbitrary element and suppose $F_n =
2^{(2^n)} + 1$. We have to prove that $F_{n+1} = F_0.F_1.F_2 ... F_n +
2$. By inductive hypothesis,
\begin{align*}
  F_n = F_0.F_1.F_2 ... F_{n-1} + 2 \\
  F_0.F_1.F_2 ... F_{n-1}.F_n = (F_n - 2)F_n \\
  F_0.F_1.F_2 ... F_{n-1}.F_n + 2 = (F_n - 2)F_n + 2 \\
  = (F_n)^2 - 2F_n + 2 \\
  = (2^{(2^n)} + 1)^2 - 2(2^{2^n} + 1) + 2 \\
  = 2^{(2^n)} + 1 \\
  = F_n + 1
\end{align*}

\section{Solution 6}
By mathematical induction,

Base case. When $n=1$, $|a_1| = |a_1|$ so base case holds.

Induction step. Let $n$ be arbitrary element and suppose $a_1 + a_2 +
... + a_n <= |a_1| + |a_2| + ... + |a_n|$. Let $a_{n+1}$ be arbitrary
real number. Let us consider the cases.

Case 1. $a_{n+1}$ is positive: From our inductive hypothesis, it
follows that $a_1 + a_2 + ... + a_n + a_{n+1} <= |a_1| + |a_2| + ... +
|a_n| + |a_{n+1}|$.

Case 2. $a_{n+1}$ is negative: Then it follows that
$a_1 + a_2 + ... + a_n + a_{n+1} < |a_1| + |a_2| + ... + |a_n| +
|a_{n+1}|$. So,
$a_1 + a_2 + ... + a_n + a_{n+1} <= |a_1| + |a_2| + ... + |a_n| +
|a_{n+1}|$.

\section{Solution 7}
\subsection{Solution (a)}
Let $a$ and $b$ be arbitrary positive real numbers. We know that
$(a-b)^2 >= 0$. Then,
\begin{align*}
  (a-b)^2 >= 0 \\
  a^2 + b^2 >= 2ab \\
  \frac{a}{b} + \frac{b}{a} >= 2
\end{align*}

QED.

\subsection{Solution (b)}
Suppose $a,b$ and $c$ be arbitrary real number and suppose $0 < a <= b
<= c$. Then, $(c-b)(c-a) >= 0$ since $c >= b$ and $c >= a$. Then,
\begin{align*}
  c(c-b)-a(c-b) >= 0 \\
  c(c-b) + ab >= ac \\
  \frac{ab + c(c-b)}{ac} >= 1 \\
  \frac{b}{c} + \frac{c-b}{a} >= 1 \\
  \frac{b}{c} + \frac{c}{a} - \frac{b}{a} >= 1
\end{align*}
QED.

\subsection{Solution (c)}
By mathematical induction,

Base case. When $n = 2$, from 7(a), it follows that $\frac{a_1}{a_2} +
\frac{a_2}{a_1} >= 2$. So base case holds.

Induction step. Let $n >= 2$ be arbitrary element and suppose
$\frac{a_1}{a_2} + \frac{a_2}{a_3} + \frac{a_3}{a_4} + ... +
\frac{a_{n-1}}{a_n} + \frac{a_n}{a_1} >= n$. We have to prove
$\frac{a_1}{a_2} + \frac{a_2}{a_3} + \frac{a_3}{a_4} + ... +
\frac{a_{n-1}}{a_n} + \frac{a_n}{a_{n+1}} + \frac{a_{n+1}}{a_1} >= n +
1$. From $7(b)$, it follows that $\frac{a_n}{a_{n+1}} +
\frac{a_{n+1}}{a_1} - \frac{a_n}{a_1} >= 1$.
From inductive hypothesis, it follows that:
$\frac{a_1}{a_2} + \frac{a_2}{a_3} + \frac{a_3}{a_4} + ... +
\frac{a_{n-1}}{a_n} + \frac{a_n}{a_{n+1}} +
\frac{a_{n+1}}{a_1} >= n + 1$.

QED.

\section{Solution 8}
\subsection{Solution (a)}
Let $a$ and $b$ be arbitrary positive real number. We know that
$(a-b)^2 >= 0$. So,
\begin{align*}
  (a-b)^2 >= 0 \\
  a^2 + b^2 >= 2ab \\
  a^2 + b^2 + 2ab >= 4ab \\
  (a+b)^2 >= 4ab \\
  \frac{(a+b)}{2} >= \sqrt{ab}
\end{align*}
QED.

\subsection{Solution (b)}
By mathematical induction on $n$,

Base case. When $n=1$, From 8(a), base case holds.

Induction step. Suppose $n>=1$ and the arithmetic geometric mean
inequality holds for lists of length $2^n$. Now let $a_1,a_2,...,
a_{a^{n+1}}$ be a list of $2^{n+1}$ positive real numbers. Let
$m_1 = \frac{a_1 + a_2 + ... + a_{2^n}}{2^n} and m_2 = \frac{a_{2^n +
    1} + a_{2^n + 2} + ... + a_{2^{n+1}}}{2^n}$.

So, $a_1 + a_2 + ... + a_{2^n} = m_12^n$ and similarly $a_{2^n + 1} +
a_{2^n + 2} + ... + a_{2^{n+1}} = m_22^n$. Also, by inductive
hypothesis, we know that $m_1 >= \sqrt[2^n]{a_1a_2...a_{2^n}}$ and
$m_2 >= \sqrt[2^n]{a_{2^n + 1}a_{2^n + 2}...a_{2^{n+1}} }$. Therefore,
\begin{align*}
  \frac{a_1 + a_2 + ... + a_{2^n + 1}}{2^{n+1}} \\
  = \frac{m_12^n + m_22^n}{2^{n+1}} \\
  = \frac{m_1 + m_2}{2} \\
  >= \sqrt{m_1m_2}   (\text{from Solution (a)}) \\
  >= \sqrt{\sqrt[2^n]{a_1a_2...a_{2^n}}\sqrt[2^n]{a_{2^n + 1}a_{2^n+2}...a_{2^{n+1}}}} \\
  >= \sqrt[2^{n+1}]{a_1a_2...a_{a^{n+1}}}
\end{align*}

\section{Solution (c)}
By mathematical induction on $n$,

Base case. When $n=n_0$, then base case holds because of inductive
hypothesis.

Induction step. Suppose $n >= n_0$ and there are positive real numbers
$a_1, a_2, ..., a_n$ such that
\begin{align*}
  \frac{a_1 + a_2 + ... + a_n}{n} < \sqrt[n]{a_1a_2...a_n}
\end{align*}
Let $m = \frac{a_1 + a_2 + ... + a_n}{n}$ and $a_{n+1} = m$. Then we
have $m < \sqrt[n]{a_1a_2...a_n}$, so $m^n < a_1a_2...a_n$. So,
$m^{n+1} < a_1a_2...a_na_{n+1}$. So, $m <
\sqrt[n+1]{a_1a_2...a_{n+1}}$. Now,
\begin{align*}
  \frac{a_1 + a_2 + ... + a_{n+1}}{n+1} \\
  = \frac{mn + m}{n+1} \\
  = m
\end{align*}
So, $\frac{a_1 + a_2 + ... + a_{n+1}}{n+1} <
\sqrt[n+1]{a_1a_2...a_{n+1}}$

\subsection{Solution (d)}
$(b)$ and $(c)$ give contradiction for a list of length $n_0 < 2^n$
where $n >= 1$. So the inequality must hold.

\section{Solution 9}
Let $\frac{1}{a_n} = k_n$. From arithmetic geometric mean inequality,

\begin{align*}
  \frac{k_1 + k_2 + ... + k_n }{n} <= \sqrt[n]{k_1K_2...K_n} \\
  \frac{n}{k_1 + k_2 + ... + k_n } <= (k_1k_2...k_n)^{-1/n} \\
  <= (a_1a_2...a_n)^{1/n} \\
\end{align*}
QED.

\section{Solution 10}
By mathematical induction on $n$,

Base case. When $n=0$, $P(A)$ has 1 element which is the empty set.

Induction step. Let $n$ be an arbitrary element and suppose if $A$ has
$n$ elements then $P(A)$ has $2^n$ elements. Let $A$ has $n+1$
elements. Then $P(A)$ can be divided into two sets $A_1$ and $A_2$.

$A_1$ = Set which doesn't have $(n+1)$th element in it.
$A_2$ = Set which has $(n+1)$th element in it.

By inductive hypothesis, $A_1$ has $2^n$ elements. Now $A_2 = A_1
\times \{x\}$ where $x$ is the $(n+1)$th element.

So, $2^n + 2^n = 2^{n+1}$ elements.

\section{Solution 11}
By mathematical induction,
Base case. When $n=0$, $P_2(A)$ has zero elements, so base case holds.

Induction step. Let $n$ be an arbitrary element and suppose if $A$ has
$n$ elements then $P_2(A)$ has $n(n-1)/2$ elements. Suppose $A$ has
$n+1$ elements. Let us pick an arbitrary element $a$ from $A$ such
that $A' = A \setminus \{a\}$. Now $A'$ has $n$ elements and $P_2(A')$
has $n(n-1)/2$ elements. There are two kinds of subsets of $A$ those
that contain $a$ as an element and those that don't. The subset hat
don't contain $a$ are just the subsets of $A'$ and by inductive
hypothesis there are $n(n-1)/2$ of them. Now the subset of $A$ that
contain $A$ and have a length of two are $n$. Summing them, we have
$n(n+1)/2$ elements as required.

\section{Solution 12}
By mathematical induction,

Base case. When $n=1$, the remaining area is a trapezoidal.

Induction step. Let $n$ be an arbitrary element and suppose if an
equilateral triangle is cut into $4^n$ congruent equilateral triangles
and one corner is removed, then the remaining area is covered by
trapezoidal tiles. Suppose equilateral triangle is cut into $4^{n+1}$
congruent triangles with $4^n$ triangles in each of them. Now you can
view this as $4$ congruent equilateral triangles. Now since one corner
has been removed, one triangle out of the $4$ has remaining area of it
is covered by trapezoidal tiles. Now for the remaining three triangle,
cut a trapezoidal area such that a corner is removed from each of
them. Then again applying inductive hypothesis on each of them, we get
trapezoidal area in each of them.

\section{Solution 13}
By mathematical induction on $n$,

Base case. When $n = 1$, the chord divides the circle into $2$.
Calculating $\frac{n^2 + n + 2}{2}$, we know that base case holds.

Induction step. Let $n$ be an arbitrary element such that $n$ chords
cut the circle into $\frac{n^2 + n + 2}{2}$ regions. Now suppose
$(n+1)$ chords are drawn in the circle. Now for $n$ chords, we have
$\frac{n^2 + n + 2}{2}$ regions. And for the next chord, we have
$(n+1)$ regions.
So,
\begin{align*}
  \frac{n^2 + n + 2}{2} + (n+1) \\
  = \frac{n^2 + 3n + 4}{2} \\
  = \frac{(n+1)^2 + (n+1) + 2}{2}
\end{align*}

\section{Solution 15}
You cannot make any assumptions about $n$. $n \in A$ doesn't imply
$(n+1) \in A$.


\end{document}
