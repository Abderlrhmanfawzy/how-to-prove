%%%%%%%%%%%%%%%%%%%%%%%%%%%%%%%%%%%%%%%%%
% Author: Sibi <sibi@psibi.in>
%%%%%%%%%%%%%%%%%%%%%%%%%%%%%%%%%%%%%%%%%
\documentclass{article}
\usepackage{graphicx}
\usepackage{verbatim}
\usepackage{amsmath}
\usepackage{amsfonts}
\usepackage{amssymb}
\usepackage{tabularx}
\setlength\parskip{\baselineskip}
\begin{document}
\title{Chapter 3 (Section 3.4)}
\author{Sibi}
\date{\today}
\maketitle
\newpage

\section{Problem 1}

( $\Rightarrow$ ) Suppose $\forall x ( P(x) \land Q(x))$. Let y be
arbitrary. Then since $\forall x (P(x) \land Q(x))$, $P(y) \land Q(y)$
and so in particular $P(y)$. Since y is arbitrary, this shows that
$\forall x P(x)$. Similarly, $\forall x Q(x)$ for arbitrary y. Thus,
$\forall x P(x) \land \forall x Q(x)$.

( $\Leftarrow$ ) Suppose $\forall x P(x) \land \forall x Q(x)$. Let y
be arbitrary. Then since $\forall x P(x)$, $P(y)$ and similarly since
$\forall x Q(x)$, $Q(y)$. Thus $P(y) \land Q(y)$ and since y is
arbitrary, it follows that $\forall x(P(x) \land Q(x))$.

\section{Problem 2}

Suppose $A \subseteq B \land A \subseteq C$. Let x be an arbitrary
element in $A$. From $x \in A$ and $A \subseteq B$, it follows that $x
\in B$. Similarly, from $x \in A$ and $A \subseteq C$, it follows that
$x \in C$. Therefore, $x \in (B \cap C)$. Since x is arbitrary, we can
conclude that $A \subseteq B \cap C$.

\section{Problem 3}

Suppose $A \subseteq B$. Let $C$ be an arbitrary set and x be an
arbitrary element in $C \setminus B$. It follows that $x \in C$ and $x
\notin B$. From $x \notin B$ and $A \subseteq B$, it follows that $x
\notin A$. Therefore $x \in C$ and $x \notin A$. Since x is arbitrary,
$C/B \subseteq C/A$.

\section{Problem 4}

Suppose $A \subseteq B \cap A \nsubseteq C$. Let x be an arbitrary
element in A. From $x \in A$ and $A \subseteq B$, it follows that $x
\in B$. Let there be some $y \in A$ and from $A \nsubseteq C$, it
follows that $y \notin C$. Therefore, $x \in B$ and $y \notin C$.
Since x is arbitrary, $y \in B$. Therefore, $y \nsubseteq C$.
\end{document}

