%%%%%%%%%%%%%%%%%%%%%%%%%%%%%%%%%%%%%%%%%
% Author: Sibi <sibi@psibi.in>
%%%%%%%%%%%%%%%%%%%%%%%%%%%%%%%%%%%%%%%%%
\documentclass{article}
\usepackage{graphicx}
\usepackage{verbatim}
\usepackage{amsmath}
\usepackage{amsfonts}
\usepackage{amssymb}
\usepackage{tabularx}
\setlength\parskip{\baselineskip}
\begin{document}
\title{Chapter 3 (Section 3.4)}
\author{Sibi}
\date{\today}
\maketitle
\newpage

\section{Problem 1}

( $\Rightarrow$ ) Suppose $\forall x ( P(x) \land Q(x))$. Let y be
arbitrary. Then since $\forall x (P(x) \land Q(x))$, $P(y) \land Q(y)$
and so in particular $P(y)$. Since y is arbitrary, this shows that
$\forall x P(x)$. Similarly, $\forall x Q(x)$ for arbitrary y. Thus,
$\forall x P(x) \land \forall x Q(x)$.

( $\Leftarrow$ ) Suppose $\forall x P(x) \land \forall x Q(x)$. Let y
be arbitrary. Then since $\forall x P(x)$, $P(y)$ and similarly since
$\forall x Q(x)$, $Q(y)$. Thus $P(y) \land Q(y)$ and since y is
arbitrary, it follows that $\forall x(P(x) \land Q(x))$.

\section{Problem 2}

Suppose $A \subseteq B \land A \subseteq C$. Let x be an arbitrary
element in $A$. From $x \in A$ and $A \subseteq B$, it follows that $x
\in B$. Similarly, from $x \in A$ and $A \subseteq C$, it follows that
$x \in C$. Therefore, $x \in (B \cap C)$. Since x is arbitrary, we can
conclude that $A \subseteq B \cap C$.

\section{Problem 3}

Suppose $A \subseteq B$. Let $C$ be an arbitrary set and x be an
arbitrary element in $C \setminus B$. It follows that $x \in C$ and $x
\notin B$. From $x \notin B$ and $A \subseteq B$, it follows that $x
\notin A$. Therefore $x \in C$ and $x \notin A$. Since x is arbitrary,
$C/B \subseteq C/A$.

\section{Problem 4}

Suppose $A \subseteq B \cap A \nsubseteq C$. Let x be an arbitrary
element in A. From $x \in A$ and $A \subseteq B$, it follows that $x
\in B$. Let there be some $y \in A$ and from $A \nsubseteq C$, it
follows that $y \notin C$. Therefore, $x \in B$ and $y \notin C$.
Since x is arbitrary, $y \in B$. Therefore, $y \nsubseteq C$.

\section{Problem 5}

Suppose $A \subseteq B \setminus C$ and $A \neq \emptyset$. Let x be
some element in A. From $x \in A$ and $A \subseteq B \setminus C$, it
follows that $x \in B$ and $x \notin C$. Therefore $x \in B \land x
\notin C$, so $B \nsubseteq C$.

\section{Problem 6}

Let x be arbitrary, then
\begin{align*}
  x \in A \setminus (B \cap C) \iff x \in A \land x \notin (B \cap C) \\
                               \iff x \in A \land \neg (x \in B \land
                               C) \\
                               \iff x \in A \land \neg(x \in B \land x
                               \in C) \\
                               \iff x \in A \land (x \notin B \lor x
                               \notin C) \\
                               \iff (x \in A \land x \notin B) \lor (x
                               \in A \land x \notin C) \\
                               \iff x \in (A \setminus B) \lor x \in
                               (A \setminus C) \\
                               \iff x \in (A \setminus B) \cup (A
                               \setminus C)
\end{align*}

Thus, $\forall x (x \in A \setminus (B \cap C)) \iff x \in (A
\setminus B) \cup (A \setminus C)$, so $A \setminus (B \cap C) = (A
\setminus B) \cup (A \setminus C)$.

\section{Problem 7}

( $\Rightarrow$ ) Let A and B be arbitrary set. Suppose $x \in P(A
\cap B)$. Then by definition,  $x \subseteq A \cap B$. Let y be an
arbitrary element such that $y \in x$. From $y \in x$ and $x \subseteq
A \cap B$, it follows that $y \in A \cap B$. Similarly, $x \in P(B)$.
Therefore $x \in (P(A) \cap P(B))$.

( $\Leftarrow$ ) Suppose $x \in P(A) \cap P(B)$. Then $x \in P(A)$ and
$x \in P(B)$, so $x \subseteq A$ and $x \subseteq B$. Let y be an
arbitrary element in x, then $y \in A$ and $y \in B$. Therefore, $y
\in A \cap B$. Since $y \in x$ and $y \in A \cap B$, it follows that
$x \subseteq A \cap B$. Thus, $x \in P(A \cap B)$.

\section{Problem 8}

( $\Rightarrow$ ) Suppose $A \subseteq B$. Let x be an arbitrary
element in $A$. It follows that $x \in A$ and $x \in B$, therefore $x
\subseteq P(A)$ and $x \subseteq P(B)$. Let y be an arbitrary element
in x. From, $y \in x$ and $x \subseteq P(A)$, it follows that $y \in
P(A)$. Similarly $y \in P(B)$. Since y is arbitrary, $P(A) \subseteq
P(B)$.

( $\Leftarrow$ ) Let x be an arbitrary element in $P(A)$. Suppose
$P(A) \subseteq P(B)$. It follows that $x \in P(A)$ and $x \in P(B)$,
therefore $x \subseteq A$ and $x \subseteq B$. From here, we can
conclude that $A \subseteq B$.

\section{Problem 9}

Suppose x and y are odd integers. Let k be some number such that $x =
2k + 1$. Similarly, let j be some number such that $y = 2j + 1$.
Multiplying x and y, we get $2(2kj + k + j) + 1$. Since $2kj + k + j$
is an integer, we can conclude that $xy$ is an odd number.

\section{Problem 10}

( $\Rightarrow$ ) Let n be an arbitrary number in $ \mathbb{Z}$. We will prove
the contrapositive. Suppose n is a odd number. Then there exists some
k such that $n = 2k + 1$. Multiplying n with $n^2$ we get $2(2k^3 +
6k^2 + 3k) + 1$. Since $2k^3 + 6k^2 + 3k + 1$ is an integer, it
follows that $n^3$ is an odd number.

( $\Leftarrow$ ) Let n be an arbitrary number in $ \mathbb{Z} $. Suppose n is
even. Then there is an some number k such that $n = 2k$. Multiplying n
by $n^2$, we get $n(n^2) = 2k(2k)^2 = 2k(4k^2) = 2(4k^3)$. Since
$4k^3$ is an integer, it follows that $n^3$ is an even number.

\section{Problem 11}

Solution(a)

You cannot introduce the same integer $k$ in both cases.

Solution(b)

\begin{align*}
  n = 0 \land m = 1 \\
  n^2 - m^2 = 0 - 1 = -1 \\
  n + m = 0 + 1 \\
\end{align*}

\section{Problem 12}

( $\Rightarrow$ ) Let x be an arbitrary element in R. Let there be
some $y$ in $\mathbb{R}$. We will prove the contrapositive. Suppose $x
= 1$, then $x + y = 1 + y$ and $xy = y$. So, $x + y \neq xy$.
Therefore, if $x + y = xy$, then $x \neq 1$.

( $\Leftarrow$ ) Let x be an arbitrary element in $R$. Suppose $x \neq
1$. Let $y = x / x - 1$. Summing it with $x$, $x+y = xy$. Therefore if
$ x \neq 1$, then $x + y = xy$.

\section{Problem 13}

( $\Rightarrow$ ) Let $z = 1$. Let x be an arbitrary element in
$\mathbb{R^+}$. Let $y$ be an element in $\mathbb{R}$. Suppose $y - x
= y / x$. We will prove the contradiction. Let $x = z$, it follows
that $x = 1$. Putting it in $y - x$, we get $y -x = y - 1$. Similarly,
putting it in $y - x$, we get $y - x = y - 1$. Similarly, putting it
in $y / x$, we get $y / x = y$. Therefore $y - x \neq y/x$. But it
contradicts the fact that $y - x = y / x$, so $x \neq z$. Therefore,
if $y -x = y/x$, then $x \neq z$.

( $\Leftarrow$ ) Let $z = 1$. Let $x$ be an arbitrary element in
$\mathbb{R^+}$. Let $y = x^2 / x - 1$. Suppose $x \neq z$. Solving $ y
- x = \frac{x^2}{x - 1} = \frac{x}{x - 1} * \frac{x}{x} =
\frac{y}{x}$. So, $y - x = \frac{y}{a}$. Therefore if $x \neq z$, then
$y - x = \frac{y}{x}$.

\section{Problem 14}

Let x be an arbitrary element in $\cup \{ A \setminus B | A \in F \}$.
It follows that there exists some $A$ in $F$ such that $x \in A
\setminus B$. So, $x \in A$ and $x \notin B$. From $x \in A$ and $x
\notin B$, it follows that $A \nsubseteq B$. Since $A \nsubseteq B$,
it follows that $A \notin P(B)$. So $A \in F$ and $A \notin P(B)$, $A
\in F \setminus P(B)$. Since $x \in A$, it follows that $x \in \cup(F
\setminus P(B))$. Since x is arbitrary, we can conclude that $\cup \{
A \setminus B | A \in F \} \subseteq \cup (F \setminus P(B))$.

\section{Problem 15}

Let $A$ be an arbitrary element in $F$ and $B$ be some element in $G$.
Suppose $\forall A \in F \exists B \in G (A \cap B = \emptyset)$. Let
$x$ be an arbitrary element in $A$. From $x \in A$ and $A \in F$, it
follows that $x \in \cup F$. Let us assume that $x \in \cap G$. Then
for all element in $G$, x is present in it. But this contradicts the
fact that there is some element $B in G$ such that $x \notin B$.
Therefore $x \notin \cap G$, So $\cup F$ and $\cap G$ are disjoint
sets.

\section{Problem 16}

( $\Leftarrow$ ) Let $x$ be an arbitrary element in $A$. Since $A \in
P(A)$, then by definition of $\cup P(A)$, $x \in \cup P(A)$.

( $\Rightarrow$ ) Let $x$ be an arbitrary element in $\cup P(A)$. Then
by definition, there exists some element $B$ in $P(A)$ such that $x
\in B$. Since $B \in P(A)$, it follows that $B \subseteq A$. From $x
\in B$ and $B \subseteq A$, it follows that $x \in A$. Since x is
arbitrary, $\cup P(A) \subseteq A$.

\section{Problem 17}

Solution (a)

( $\Leftarrow$ ) Let $x$ be an arbitrary element in $\cup(F \cap G)$.
Then there exists some element $A$ in $F \cap G$ such that $x \in A$.
Since $A \in F$ and $A \in G$, from $x \in A$, we can conclude that $x
\in \cup F$ and $x \in \cup G$. Therefore, $x \in (\cup F) \cap (\cup
G)$. Since $x$ is arbitrary, we can conclude that $(F \cap G)
\subseteq (\cup F) \cap (\cup G)$.

Solution (b)

Both $A$ cannot be same.

Solution (c)

\begin{align*}
  F = \{\{1\}\} \\
  G = \{\{1\}, \{2\}\} \\
  \cup (F \cap G) = \emptyset \\
  \cup (F) \cap \cup (G) = \{1\} \\
\end{align*}

\section{Problem 18}

( $\Rightarrow$ ) Suppose $(\cup F) \cap (\cup G) \subseteq \cup (F
\cap G)$. Let $x$ be an arbitrary element in $\forall A \in F \forall
B \in G (A \cap B)$, so $x \in A$ and $x \in B$. Suppose $A \in F$ and
$x \in A$, it follows that $x \in \cup F$. Similarly, $x \in \cup G$,
therefore $x \in (\cup F) \cap (\cup G)$. From, $x \in (\cup F) \cap
(\cup G)$ and $(\cup F) \cap (\cup G) \subseteq \cup (F \cap G)$ it
follows that $x \in \cup (F \cap G).$ Since $x$ is arbitrary $\forall
A \in F \forall B \in G (A \cap B) \subseteq \cup (F \cap G)$.

( $\Leftarrow$ ) Suppose $\forall A \in F \forall B \in G (A \cap B)
\subseteq \cup (F \cap G)$. Let $x$ be an arbitrary element in $(\cup
F) \cap (\cup G)$. Therefore $x \in \cup F$, it follows that there
exists a set $f \in F$ such that $x \in f$. Similarly there exists a
set $g \in G$ such that $x \in g$. Since $\forall A \in F \forall B \in G (A \cap B)
\subseteq \cup (F \cap G)$, so $(f \cap g) \subseteq \cup (F \cap G)$,
it follows $x \in (F \cap G)$. Since $x$ is arbitrary, we can conclude
that $(\cup F) \cap (\cup G) \subseteq \cup(F \cap G)$.

\section{Problem 19}

( $\Rightarrow$ ) Suppose $(\cup F) \cap (\cup G) = \emptyset$. Let
$A$ be an arbitrary element in $F$ and $B$ be an arbitrary element in
$G$. Let $x$ be an arbitrary element in $A$. From $x \in A$ and $A \in
F$, it follows that $x \in \cup F$. Also, from $x \in \cup F$ and
$(\cup F) \cap (\cup G)$, it follows that $x \notin \cup G$, so $x
\notin B$. Since $x$ is arbitrary and $x \in A$ and $x \notin B$ it
follows that $A \cap B = \emptyset$. Therefore if $(\cup F) \cap (\cup
G) = \emptyset$, then $\forall A \in F \forall B \in G (A \cap B =
\emptyset)$.

( $\Leftarrow$ ) Suppose $\forall A \in F \forall B \in G (A \cap B =
\emptyset)$. Let $x$ be an arbitrary element in $\cup F$. Then there
is some element $f$ in $F$ such that $x \in F$. From $\forall A \in F
\forall B \in G (A \cap B = \emptyset)$, it follows $f \cap B =
\emptyset$. Since $x \in f$, it follows that $x \notin B$. Since $x
\notin B$, it follows that $x \notin \cup G$. Since $x$ is arbitrary
and $x \in \cup F$ and $x \notin \cup G$, it follows that $(\cup F)
\cap (\cup G) = \emptyset$. Therefore, if $\forall A \in F \forall B
\in G (A \cap B = \emptyset)$ then $(\cup F) \cap (\cup G)$.

\section{Problem 20}

Solution (a)

Let $x$ be an arbitrary element in $(\cup F) \setminus (\cup G)$, so
$x \in \cup F$ and $x \notin \cup G$. From $x \in \cup F$, we can
conclude that there is some $A$ in $F$ such that $x \in A$. From $x
\notin \cup G$, we can conclude that for all elements $B$ in $G$, $x
\notin B$. $A \notin G$ because if it was in $G$, it will lead to
contradiction. Therefore, $A \in F \setminus G$. From $x \in F$ and $A
\in F \setminus G$ it follows that $x \in \cup (F \setminus G)$. Since
$x$ is arbitrary it follows that $(\cup F) \setminus (\cup G)
\subseteq \cup (F \setminus G)$.

Solution (b)

Problem is with this line: Since $x \in A$ and $A \notin G$, $x \notin
\cup G$.

Example: $x = 1 \land A = \{1\} \land G = \{\{1,2\}\} \land x
\in \cup G$.

Solution (c)

( $\Rightarrow$ ) Suppose $\cup(F \setminus G) \subseteq (\cup F)
\setminus (\cup G)$. Let $A$ be an arbitrary element in $F \setminus
G$ and $B$ be an arbitrary element in $G$. Let $x$ be an arbitrary
element in $A$. From $A \in F \setminus G$ and $x \in A$, it follows
that $x \in \cup (F \setminus G)$. Since $x \in \cup(F \setminus G)$
and from $\cup(F \setminus G) \subseteq (\cup F) \setminus (\cup G)$,
it follows that $x \in (\cup F) \setminus (\cup G)$. So $x \in \cup F$
and $x \notin \cup G$. Since $x \notin \cup G$ and $B \in G$, it
follows that $x \notin B$. So, $x \in A$ and $x \notin B$. Since $x$
is arbitrary $A \cap B = \emptyset$.

( $\Leftarrow$ ) Suppose $\forall A \in (F \setminus G) \forall B \in
G(A \cap B = \emptyset)$. Let $x$ be an arbitrary element in $\cup(F
\setminus G)$. Then there is some element $Y$ in $F \setminus G$ such
that $x \in Y$. From $Y \in F \setminus G$, it follows that $Y \in F$
and $Y \notin G$. Since $x \in Y$ and $Y \in F$ it follows that $x \in
\cup F$. We will prove the contradiction. Let us assume $x \in \cup
G$. Since $x \in \cup G$, it follows that there is some element $g$ in
$G$ such that $x \in g$. But this contradicts the fact that for all
element $B$ in $G$ if $x$ is in $A$ and $x \notin B$, so $x \notin \cup
G$. Since $x$ is arbitrary, we can conclude $\cup (F \setminus G)
\subseteq (\cup F) \setminus (\cup G)$.

Solution (d)
\begin{align*}
F = \{\{1,2\}\} \\
G = \{\{2\}\}  
\end{align*}

\end{document}



