%%%%%%%%%%%%%%%%%%%%%%%%%%%%%%%%%%%%%%%%%
% Author: Sibi <sibi@psibi.in>
%%%%%%%%%%%%%%%%%%%%%%%%%%%%%%%%%%%%%%%%%
\documentclass{article}
\usepackage{graphicx}
\usepackage{verbatim}
\usepackage{amsmath}
\usepackage{amsfonts}
\usepackage{amssymb}
\usepackage{tabularx}
\setlength\parskip{\baselineskip}
\begin{document}
\title{Chapter 3 (Section 3.4)}
\author{Sibi}
\date{\today}
\maketitle
\newpage

\section{Problem 1}

( $\Rightarrow$ ) Suppose $\forall x ( P(x) \land Q(x))$. Let y be
arbitrary. Then since $\forall x (P(x) \land Q(x))$, $P(y) \land Q(y)$
and so in particular $P(y)$. Since y is arbitrary, this shows that
$\forall x P(x)$. Similarly, $\forall x Q(x)$ for arbitrary y. Thus,
$\forall x P(x) \land \forall x Q(x)$.

( $\Leftarrow$ ) Suppose $\forall x P(x) \land \forall x Q(x)$. Let y
be arbitrary. Then since $\forall x P(x)$, $P(y)$ and similarly since
$\forall x Q(x)$, $Q(y)$. Thus $P(y) \land Q(y)$ and since y is
arbitrary, it follows that $\forall x(P(x) \land Q(x))$.

\end{document}

