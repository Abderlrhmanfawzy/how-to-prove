%%%%%%%%%%%%%%%%%%%%%%%%%%%%%%%%%%%%%%%%%
% Author: Sibi <sibi@psibi.in>
%%%%%%%%%%%%%%%%%%%%%%%%%%%%%%%%%%%%%%%%%
\documentclass{article}
\usepackage{graphicx}
\usepackage{verbatim}
\usepackage{amsmath}
\usepackage{amsfonts}
\usepackage{amssymb}
\setlength\parskip{\baselineskip}
\begin{document}
\title{Chapter 3 (Section 3.1)}
\author{Sibi}
\date{\today}
\maketitle
\newpage

\section{Problem 1}
Consider the following theorem. (This theorem was proven in the
introduction.)

Theorem. Suppose n is an integer larger than 1 and n is not prime.
Then $2^n - 1$  is not prime.

(a) Identify the hypotheses and conclusion of the theorem. Are the
hypotheses true when n = 6? What does the theorem tell you in this
instance? Is it right?

(b) What can you conclude from the theorem in the case n = 15? Check
directly that this conclusion is correct.

(c) What can you conclude from the theorem in the case n = 11?

Solution (a)

Hypotheses:
\begin{itemize}
\item n is an integer
\item $n > 1$
\item n is not prime
\end{itemize}

Conclusion:
$2^n - 1$ is not prime.

Hypotheses are true when $n = 6$. The theorem in the instance says
that $2^n - 1 = 65$ for $n=6$ is not a prime number. Yes, it is
correct.

Solution (b)

When, $n=15, 2^n - 1 = 32767/$. So, I guess that is not an prime
number. 7 is one of the factors.

Solution (c)

When, $n=11, 2^n - 1 = 2047$. So 2047 is not a prime number. And hey
23 is one of it's factor.

\section{Problem 2}


\end{document}

