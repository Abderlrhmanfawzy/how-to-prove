%%%%%%%%%%%%%%%%%%%%%%%%%%%%%%%%%%%%%%%%%
% Author: Sibi <sibi@psibi.in>
%%%%%%%%%%%%%%%%%%%%%%%%%%%%%%%%%%%%%%%%%
\documentclass{article}
\usepackage{graphicx}
\usepackage{verbatim}
\usepackage{amsmath}
\usepackage{amsfonts}
\usepackage{amssymb}
\usepackage{tabularx}
\setlength\parskip{\baselineskip}
\begin{document}
\title{Chapter 4 (Section 4.1)}
\author{Sibi}
\date{\today}
\maketitle
\newpage

\section{Problem 1}

Solution (a)

$\{(x,y) \in P \times P \mid x \text{ is a parent of } y \} = \{(Prabakaran, Sibi) ..
(Prabakaran, Madhu)\}$ 

Solution (b)

$\{(x,y) \in C \times U \mid \text{Someone lives in } x \text{ and
    attends } y \}$

\section{Problem 2}

Solution (a)

$\{(x,y) \in P \times C \mid x \text{ lives in } y \} = \{(Sibi, Chennai) .. \}$

Solution (b)

$\{(x,y) \in C \times N \mid \text{Population of } x \text{ is } y \} = \{(Chennai, 3434343)\}$

\section{Problem 3}

Solution (a)

$\{(0,-2), (1,-2), ..\}$

Solution (b)

$\{(1,0), (2,0), (3,0), ..\}$

Solution (c)

$\{(0,-2), (1,-2), ..\}$

Solution (d)

$\{(2,0), (0, -2), ..\}$

\section{Problem 4}

\begin{align*}
A = \{1,2,3\} \\
B = \{1,4\} \\
C = \{3,4\} \\
D = \{5\} 
\end{align*}

Solution(1)

\begin{align*}
B \cap C = \{4\} \\
A \times (B \cap C) = \{(1,4),(2,4),(3,4)\} \\ \\
A \times B = \{(1,1),(1,4),(2,1),(2,4),(3,1),(3,4)\} \\
A \times C = \{(1,3),(1,4),(2,3),(2,4),(3,3),(3,4)\} \\
(A \times B) \cap (A \times C) = \{(1,4),(2,4),(3,4)\} 
\end{align*}

Solution(2)

\begin{align*}
B \cup C = \{1,3,4\} \\
A \times (B \cup C) =
\{(1,1),(1,4),(1,3),(2,1),(2,4),(2,3),(3,1),(3,4),(3,3)\} \\
(A \times B) \cup (A \times C) =
\{(1,1),(1,4),(2,1),(2,4),(3,1),(3,4),(1,3),(2,3),(3,3)\} \\
\end{align*}

Solution(3)

\begin{align*}
A \times B = \{(1,1),(1,4),(2,1),(2,4),(3,1),(3,4)\} \\
C \times D = \{(3,5),(4,5)\} \\ \\
(A \times B) \cap (C \times D) = \emptyset \\
A \cap C = \{3\} \\
B \cap D = \emptyset \\
(A \cap C) \times (B \cap D) = \emptyset 
\end{align*}

Solution(4)

\begin{align*}
A \times B = \{(1,1),(1,4),(2,1),(2,4),(3,1),(3,4)\} \\
C \times D = \{(3,5),(4,5)\} \\
(A \times B) \cup (C \times D) =
\{(1,1),(1,4),(2,1),(2,4),(3,1),(3,4),(3,5),(4,5)\} \\ \\
A \cup C = \{1,2,3,4\} \\
B \cup D = \{1,4,5\}\\
(A \cup C) \times (B \cup D) = \{(1,1),(1,4),(1,5),(2,1),(2,4),(2,5),(3,1),(3,4),(3,5),(4,1),(4,4),(4,5)\}
\end{align*}

\section{Problem 5}

Solution(a)

($ \Leftarrow $) Let $p$ be an arbitrary element of $A \times (B \cup C)$. Then by the
definition of cartesian product, $p = (x,y)$ for some $x \in A$ and $y
\in B \cup C$. Let us consider the cases:

Case 1. $y \in B$. Since $x \in A$ and $y \in B$, it follows that $p
\in A \times B$. So, $p \in (A \times B) \cup (A \times C)$.

Case 2. $y \in C$. Since $x \in A$, it follows that $p \in A \times
C$. So, $p \in (A \times B) \cup (A \times C)$.

($ \Rightarrow $) Let $p$ be an arbitrary element of $(A \times B)
\cup (A \times C)$. Let us consider the cases:

Case 1. $p \in A \times B$ Then there is some element $x$ and $y$ such
that $p = (x,y)$ and $x \in A$ and $y \in B$. From $y \in B$, it
follows that $y \in B \cup C$. So, $p \in A \times (B \cup C)$.

Case 2. $p \in A \times C$ Then by definition of cartesian product, $p
= (x,y)$ for some $x \in A$ and $y \in C$. From $y \in C$, it follows
that $y \in B \cup C$. So, $p \in A \times (B \cup C)$.

Solution(b)

($ \Leftarrow $) Let $p$ be an arbitrary element in $(A \times B) \cap
(C \times D)$. Then it follows that $p \in A \times B$ and $p \in C
\times D$. By definition of cartesian product, $p = (x,y)$ for some $x
\in A$ and $y \in B$. Similarly, $x \in C$ and $y \in D$. It follows
that $x \in (A \cap C)$ and $y \in (B \cap D)$. So, $(x,y) \in (A \cap
C) \times (B \cap D)$. Since $p$ is an arbitrary element, $(A \times
B) \cap (C \times D) \subseteq (A \cap C) \times (B \cap D)$.

($ \Rightarrow $) Let $p$ be an arbitrary element in $(A \cap C)
\times (B \cap D)$. By the definition of cartesian product it follows
that $p = (x,y)$ for some $x \in A \cap C$ and $y \in (B \cap D)$.
From $x \in A \cap C$, we can conclude that $x \in A$ and $x \in C$.
Similarly, $y \in B$ and $y \in D$. So, $(x,y) \in (A \times B) \cap
(C \times D)$. Since p is arbitrary, $(A \cap C) \times (B \cap D)
\subseteq (A \times B) \cap (C \times D)$.

\section{Problem 6}

The cases are not exhaustive. What about when $x \in A$ and $y \in D$,
huh ?

\section{Problem 7}

$ m * n$

\section{Problem 8}

($ \Leftarrow $) Let $p$ be an arbitrary element in $A \times (B
\setminus C)$. From the definition of cartesian product it follows
that $p = (x,y)$ for some $x \in A$ and $y \in B \setminus C$. From $y
\in B \setminus C$, it follows that $y \in B$ and $y \notin C$.
Therefore $(x,y) \in (A \times B)$. Similarly, $(x,y) \in A \times C$.
Since $p$ is arbitrary, we can conclude that $A \times (B \setminus C)
\subseteq (A \times B) \setminus (A \times C)$.

($ \Rightarrow $) Let $p$ be an arbitrary element in $(A \times B)
\setminus (A \times C)$. It follows that $p \in (A \times B)$ and $p
\notin (A \times C)$. From the definition of cartesian product it
follows that $p = (x,y)$ for some $x \in A$ and $y \in B$. Similarly
$(x,y) \notin (A \times C)$. But we know that $x \in A$, so $y \notin
C$. Therefore, $y \in B \setminus C$. So, $(x,y) \in A \times (B
\setminus C)$. Since $p$ is arbitrary, we can conclude that $A \times
(A \setminus C) = (A \times B) \setminus (A \times C)$.

\section{Problem 9}

($ \Leftarrow $) Let $p$ be an arbitrary element in $(A \times B)
\setminus (C \times D)$. It follows that $p \in A \times B$ and $p
\notin (C \times D)$. From the definition of cartesian product, it
follows that $p = (x,y)$ for some $x \in A$ and $y \in B$. Similarly,
$(x,y) \notin (C \times D)$ which means either $x \notin c$ or $y
\notin D$. Let us consider the cases separately:

Case 1. $y \notin D$. We already know that $x \in A$ and $y \in B$.
So, $y \in B \setminus D$. Therefore $(x,y) \in (A \times (B \setminus
D))$. So, $p \in (A \times (B \setminus D)) \cup ((A \setminus C) \times B)$.

Case 2. $x \notin C$. From earlier, $x \in A$ and $y\in B$. It follows
that $x \in A \setminus C$. Therefore, $(x,y) \in ((A \setminus c)
\times B)$. So, $p \in (A \times (B \setminus D)) \cup ((A \setminus
C) \times B)$.

($ \Rightarrow $) Let $p$ be an arbitrary element in $(A \times (B
\setminus D)) \cup ((A \setminus C) \times B)$. Let us consider the
cases separately:

Case 1. $p \in A \times (B \setminus D)$. From the definition of
cartesian product, it follows that $p = (x,y)$ for some $x \in A$ and
$y \in B \setminus D$. Therefore $y \in B$ and $y \notin D$. So, $p
\in (A \times B)$ and since $ y \notin D$, it follows that $(x,y)
\notin C \times D$. So, $p \in (A \times B) \setminus (C \times D)$.

Case 2. $p \in ((A \setminus C) \times B)$ From the definition of
cartesian product, it follows that $p = (x,y)$ for some $x \in A
\setminus C$ and $y \in B$. From $x \in A \setminus C$, it follows
that $x \in A$ and $x \notin C$. So, $p \in (A \times B)$ and $p
\notin C \times D$ since $x \notin C$. So, $p \in (A \times B)
\setminus (C \times D).

\end{document}




