%%%%%%%%%%%%%%%%%%%%%%%%%%%%%%%%%%%%%%%%%
% Author: Sibi <sibi@psibi.in>
%%%%%%%%%%%%%%%%%%%%%%%%%%%%%%%%%%%%%%%%%
\documentclass{article}
\usepackage{graphicx}
\usepackage{verbatim}
\usepackage{amsmath}
\usepackage{amsfonts}
\usepackage{amssymb}
\usepackage{tabularx}
\setlength\parskip{\baselineskip}
\begin{document}
\title{Chapter 4 (Section 4.3)}
\author{Sibi}
\date{\today}
\maketitle
\newpage

\section{Problem 1}
\subsection{Solution (a)}

\begin{itemize}
\item Reflexive
\item Transitive
\item Anti-symmetric (And hence partial order)
\item Not total order because $(a,c) \lor (c,a) \notin R$
\end{itemize}

\subsection{Solution (b)}
\begin{itemize}
\item Reflexive 
\item Transitive
\item Not Anti-symmetric. (Example: $(2,-2)$)
\end{itemize}

\subsection{Solution (c)}
\begin{itemize}
\item Reflexive
\item Transitive
\item Anti symmetric
\item Not total order. (Example: $(-2,2)$)
\end{itemize}

\section{Problem 2}
\subsection{Solution (a)}
\begin{itemize}
\item Reflexive
\item Transitive
\item Anti-symmetric
\item Not total order. Example: $(go, haskell)$
\end{itemize}

\subsection{Solution (b)}
\begin{itemize}
\item Reflexive
\item Transitive
\item Not anti-symmetric (Example: $(aba, aa)$)
\end{itemize}

\subsection{Solution (c)}
\begin{itemize}
\item Reflexive
\item Transitive
\item If there is two country with same population, then it is not anti-symmetric.
\end{itemize}

\section{Problem 3}
\subsection{Solution (a)}
\begin{itemize}
\item Smallest element = 2
\item Minimal element = 2
\item Largest = 4
\item Maximal =  4
\item Upper bound = Nothing
\item Lower bound =  \{1,2\}
\item glb = 2
\end{itemize}

\subsection{Solution (b)}
\begin{itemize}
\item Smallest element = 1
\item Minimal element = 1
\item Largest = No number
\item Maximal = No number
\item Upper bound = $\{2, ... R\}$
\item Lower bound = $\{... 1\}$
\item glb = 1
\item lub = 2
\end{itemize}

\subsection{Solution (c)}
\begin{itemize}
\item Smallest element = $\emptyset$
\item Minimal element = $\emptyset$
\item Largest = Nothing
\item Maximal =  $\{x \in \mathcal{P}(\mathbb{N} \mid \text{x has
    exactly 5 elements}\}$
\item Upper bound = $\{emptyset\}$
\item Lower bound = No upper bound
\item glb = $\emptyset$
\end{itemize}

\section{Problem 4}

($\Rightarrow$) Suppose $R$ is anti-symmetric and symmetric. Let
$(a,b)$ be an arbitrary element in $R$. Since $R$ is symmetric it
follows that $(b,a) \in R$. Now from anti-symmetric property it
follows that $a = b$. Since $a = b$, it follows that $(a,b) \in i_A$.
Since $(a,b)$ was arbitrary, it follows that $R \subseteq i_A$.

($\Leftarrow$) Suppose $R \subseteq i_A$. Let $a,b$ be arbitrary
element in $A$ such that $(a,b) \in R$. From $R \subseteq i_A$, it
follows that $a = b$. Therefore $R$ is symmetric. Also, since
$(a,b) \in R$ and $(b,a) \in R$ and $a = b$, it follows that $R$ is
anti-symmetric.

\section{Problem 5}
Suppose $R$ is partial order on $A$ and $B \subseteq A$.

\subsection{Proof for reflexivity}

Let $b$ be arbitrary element in B. From $B \subseteq A$, it follows
that $b \in A$. Since we know that $R$ is partial order on $A$ it
follows that $(b,b) \in R$. Also, $(b,b) \in B \times B$. Therefore
$(b,b) \in R \cap (B \times B)$. Since $b$ is arbitrary, $R \cap (B
\times B)$ is reflexive.

\subsection{Proof for transitivity}

Let $a,b$ and $c$ be arbitrary element in $B$ such that $(a,b) \in R
\cap B \times B$ and $(b,c) \in R \cap B \times B$. Since $a$ and $c$
is an element of $B$, it follows that $(a,c) \in B \times B$. We know
that $R$ is partial order on $A$ and $(a,b) \in R$ and $(b,c) \in R$.
By applying transitive property on $R$ it follows that $(a,c) \in R$.
Therefore $(a,c) \in R \cap (B \times B)$. Hence $R \cap (B \times B)$
is transitive.

\subsection{Proof for anti-symmetric}
Let $a,b$ be arbitrary element in $B$ such that $(a,b) \in R \cap B
\times B$ and $(b,a) \in R \cap B \times B$. Since $R$ is
anti-symmetric it follows that $a = b$. Hence $R \cap B \times B$ is
anti symmetric.

\section{Problem 6}
\subsection{Solution (a)}
Suppose $R_1$ and $R_2$ are partial orders on $A$.

\subsubsection{Proof for reflexivity}

Let $a$ be an arbitrary element in $A$. Since $R_1$ and $R_2$ are
partial orders on $A$ it follows that $(a,a) \in R_1$ and $(a,a) \in
R_2$. So, $(a,a) \in R_1 \cap R_2$. Since $a$ was arbitrary, we can
conclude that $R_1 \cap R_2$ is reflexive.

\subsubsection{Proof for transitivity}

Let $a,b$ and $c$ be an arbitrary element in $A$ such that $(a,b) \in
R_1 \cap R_2$ and $(b,c) \in R_1 \cap R_2$. Since $R_1$ and $R_2$ are
partial orders it follows that $(a,c) \in R_1$ and $(a,c) \in R_2$.
So, $(a,c) \in R_1 \cap R_2$. Therefore, $R_1 \cap R_2$ is transitive.

\subsubsection{Proof for anti-symmetric}
Let $a,b$ be an arbitrary element in $A$ such that $(a,b) \in R_1 \cap
R_2$ and $(b,a) \in R_1 \cap R_2$. Since $R_1$ and $R_2$ are anti
symmetric it follows that $a = b$. Therefore $R_1 \cap R_2$ is anti
symmetric.

\subsection{Solution (b)}

Counterexample:

\begin{align*}
  R_1 = \{(1,2), (1,1), (2,2)\} \\
  R_2 = \{(2,1), (1,1), (2,2)\} \\
\end{align*}

$R_1 \cup R_2$ isn't anti-symmetric because $1 \neq 2$.

\section{Problem 7}
Suppose $R_1$ is a partial order on $A_1$. Suppose $R_2$ is a partial
order on $A_2$ and $A_1 \cap A_2 = \emptyset$.

\subsection{Solution (a)}

\subsubsection{Proof of reflexivity}
Let $a$ be an arbitrary element in $A_1 \cup A_2$. Let us consider the
cases separately:

Case 1. $a \in A_1$ Since $R_1$ is partial order on $A_1$ it follows
$(a,a) \in R_1$. So $(a,a) \in R_1 \cup R_2$.
Case 2. $a \in A_2$ Since $R_2$ is partial order on $A_2$ it follows
that $(a,a) \in R_2$. So $(a,a) \in R_1 \cup R_2$.

\subsubsection{Proof of transitivity}
Let $a,b$ and $c$ be an arbitrary element in $A_1 \cup A_2$ such that
$(a,b) \in R_1 \cup R_2$ and $(b,c) \in R_1 \cup R_2$. Let us consider
the cases:

Case 1. $(a,b) \in R_1$. Now this itself has two cases with it: $(b,c)
\in R_1$ and $(b,c) \in R_2$. Now when $(b,c) \in R_1$, then since
$R_1$ is partial order it follows that $(a,c) \in R_1 \cup R_2$. Now
when $(b,c) \in R_2$, this means that $(b,c) \notin A_1$ since $A_1
\cap A_2 = \emptyset $. In this case it is vacuously true.

Case 2. $(a,b) \in R_2$. Now this itself has two cases within it.
$(b,c) \in R_1$ or $(b,c) \in R_2$. If $(b,c) \in R_1$, then $(b,c)
\notin R_2$ since $A_1 \cap A_2 = \emptyset$. So it is vacuously true.
On the other hand if $(b,c) \in R_2$, then since $R_2$ is partial
order we can conclude that $(a,c) \in R_1 \cup R_2$.

\subsubsection{Proof of Anti-symmetric}
Let $a$ and $b$ be arbitrary element in $A_1 \cup A_2$ such that
$(a,b) \in R_1 \cup R_2$ and $(b,a) \in R_1 \cup R_2$. Let us consider
the cases separately:

Case 1. $(a,b) \in R_1$. Now let's consider the sub-cases:
\begin{itemize}
\item $(b,a) \in R_1$. Since $R_1$ is partial order, it follows that
  $a = b$.
\item $(b,a) \in R_2$. From $A_1 \cap A_2 = \emptyset$, it follows
  that $(b,a) \notin R_1$. So, it is vacuously true.
\end{itemize}

Case 2. $(a,b) \in R_2$. Now let's consider the sub-cases:
\begin{itemize}
\item $(b,a) \in R_1$ From $A_1 \cap A_2$ it follows that $(b,a)
  \notin R_2$. So it is vacuously true.
\item $(b,a) \in R_2$ Since $R_2$ is partial order it follows that $b
  = a$.
\end{itemize}

\subsection{Solution (b)}
\subsubsection{Proof of reflexivity}
Let $a$ be an arbitrary element in $A_1 \cup A_2$. From $A_1 \cap A_2
= \emptyset$ it follows that either $a \in A_1$ or $A \in A_2$ but not
both. Since $R_1$ and $R_2$ are both partial order it follows that
$(a,a) \in R_1 \cup R_2$. Since $a$ is arbitrary, we can conclude that
$R_1 \cup R_2 \cup (A_1 \times A_2)$ is reflexive.

\subsubsection{Proof of transitivity}
Let $a,b$ and $c$ be an arbitrary element in $A_1 \cup A_2$ such that
$(a,b) \in R_1 \cup R_2 \cup (A_1 \times A_2)$ and $(b,c) \in R_1 \cup
R_2 \cup (A_1 \times A_2)$. Let us consider the cases separately.

Case 1. $(a,b) \in R_1$. Now let's consider the sub cases:
\begin{itemize}
\item $(b,c) \in R_1$ Since $R_1$ is partial order it follows that
  $(a,c) \in R$. So $(a,c) \in R_1 \cup R_2 \cup (A_1 \times A_2)$.
\item $(b,c) \in R_2$ From $A_1 \cap A_2 = \emptyset$, it follows that
  $(b,c) \notin R_1$. So it is vacuously true.
\item $(b,c)\in A_1 \times A_2$ It follows that $b \in A_1$ and $c \in
  A_2$. Since $c \in A_2$ from $A_1 \cap A_2 = \emptyset$, we can
  conclude that $c \notin A_1$. So, $(b,c) \notin R_1$ since $R_1$ is
  a relation on $A_1$. So it is vacuously true.
\end{itemize}
Prove case $2$ similarly.

\subsubsection{Proof of Anti symmetric}
Let $a,b$ be arbitrary element in $A_1 \cup A_2$ such that $(a,b) \in
R_1 \cup R_2 \cup (A_1 \times A_2)$ and $(b,c) \in R_1 \cup R_2 \cup
(A_1 \times A_2)$.

Case 1. $(a,b) \in R_1$. Now let's consider the sub-cases:
\begin{itemize}
\item $(b,a) \in R_1$. Since $R_1$ is partial order on $A_1$ it
  follows that $a = b$.
\item $(b,a) \in R_2$ Now $(b,a) \notin R_1$ since $A_1 \cap A_2 =
  \emptyset$. So it is vacuously true.
\item $(b,a) \in A_1 \times A_2$. Since $a \in A_2$, it follows that
  $(b,a) \notin R_1$. So it is vacuously true.
\end{itemize}
Proof case $2$ similarly.

\subsection{Solution (c)}
\subsubsection{Part (i)}
\begin{align*}
  A_1 = \{1\} \\
  A_2 = \{2\} \\
  R_1 = \{(1,1)\} \\
  R_2 = \{(2,2)\} \\
  (1,2) \lor (2,1) \notin R_1 \cup R_2
\end{align*}

\subsubsection{Part (ii)}
Let $a$ and $b$ be arbitrary element in $A_1 \cup A_2$. Let us
consider the cases separately:

Case 1. $a \in A_1$ It has two sub cases:
\begin{itemize}
\item $b \in A_1$ Then since $R_1$ is total order it follows that
  $(a,b) \lor (b,a) \in R_1$. So, we can conclude that $(a,b) \lor
  (b,a) \in R_1 \cup R_2 \cup (A_1 \times A_2)$
\item $b \in A_2$ Then it follows that $(a,b) \in A_1 \times A_2$. So
  $(a,b) \lor (b,a) \in R_1 \cup R_2 \cup A_1 \times A_2$.
\end{itemize}

Case 2 can be proved similarly.

\section{Problem 8}
\subsection{Proof of Reflexivity}
Let $(a,b)$ be arbitrary element in $A \times B$. It follows that $a
\in A$ and $b \in B$. Since $R$ is reflexive on $A$ and $S$ is
reflexive on $B$ it follows that $(a,a) \in R$ and $(b,b) \in S$. From
$(a,a) \in R$ and $(b,b) \in S$, it follows that $((a,b),(a,b)) \in T$
from the definition of $T$. Since $(a,b)$ was arbitrary it follows
that $T$ is reflexive.

\subsection{Proof of transitivity}
Let $((a,b),(a',b'))$ and $((a',b'),(c,d))$ be arbitrary element of
$(A \times B) \times (A \times B)$ such that $((a,b),(a',b')) \in T$
and $((a',b'),(c,d)) \in T$. It follows that $(a,b) \in A \times B$
and $(a',b') \in A \times B$ such that $(a,a') \in R$ and $(b,b') \in
S$. Similarly $(a',c) \in R$ and $(b',d) \in S$. Since $R$ is partial
order it follows that $(a,c) \in R$ and $(b,d) \in S$. So,
$((a,b),(c,d)) \in T$. So, $T$ is transitive.

\subsection{Proof of Anti-symmetric}
Let $((),())$ be arbitrary element of $(A \times B) \times (A \times
B)$ such that $((a,b),(a',b')) \land ((a',b'),(a,b)) \in T$. It
follows that $(a,a') \in R$ and $(b,b') \in S$. Similarly $(a',a) \in
R$ and $(b',b) \in S$. Since $R$ is partial order and hence
anti-symmetric it follows that $a' = a$ and $b' = b$. Therefore
$((a,b),(a',b')) = ((a',b'),(a,b))$. So $T$ is anti-symmetric.

\subsection{Is T total order?}
Let $(a,b)$ and $(a',b')$ be arbitrary element of $A \times B$. It
follows that $(a,a') \in R$ and $(b,b') \in S$. Since $R$ is a total
order, it follows that $(a,a') \lor (a',a) \in R$. Similarly, $(b,b')
\lor (b',b) \in S$. So, $((a,b),(a',b')) \lor ((a',b),(a,b)) \lor
((a,b'),(a',b)) \lor ((a',b),(a,b')) \in T$. Now that doesn't make $T$
as total order!

\section{Problem 9}
\subsection{Proof of reflexivity}
Let $(a,b)$ be an arbitrary element in $A \times B$. Then it follows
that $a \in A$ and $b \in B$. Since $R$ is partial order on $A$ it
follows that $(a,a) \in R$. Similarly $(b,b) \in S$. From the
definition of $L$, it follows that $((a,b),(a,b)) \in L$. Since
$(a,b)$ is arbitrary, $L$ is reflexive.

\subsection{Proof of transitivity}
Let $(a,b),(a',b')$ and $(c,d)$ be arbitrary element in $A \times B$
such that $((a,b),(a',b')) \in L$ and $((a',b'),(c,d)) \in L$. From
the definition of $L$ it follows that $a = a'$ and $a' = c$. So, $a =
c$. Also from the definition of $L$, $(a,c) \in R$. Also we know that
$(b,b') \in S$ and $(b',d) \in S$. Since $S$ is partial order on $B$,
it follows that $(b,d) \in S$. Therefore, $((a,b),(c,d)) \in L$. Hence
$L$ is transitive.

\subsection{Proof of anti-symmetric}
Let $(a,b)$ and $(a',b')$ be arbitrary element in $A \times B$ such
that $((a,b),(a',b')) \in L$ and $((a',b'),(a,b)) \in L$. From the
definition of $L$, it follows that $a = a'$. Also, $(b,b') \in S$ and
$(b',b) \in S$. Since $S$ is anti-symmetric it follows that $b = b'$.
So, we can conclude that $((a,b),(a',b')) = ((a',b'),(a,b))$. Hence
$L$ is anti-symmetric.

\subsection{Is $L$ total order?}
Let $(a,b)$ and $(a',b')$ be arbitrary element of $A \times B$. Then
it follows that $a \in A$ and $a' in A$. Similarly $b \in B$ and $b'
\in B$. Since $R$ is total order, it follows that $((a,a') \lor
(a',a))$. Similarly, $((b,b') \lor (b',b))$. Now if $a \neq a'$, then
$L$ will not hold. Hence it is not total order.

\section{Problem 10}
Suppose $x$ and $y$ be arbitrary element in $A$.
$\Rightarrow$ Suppose $xRy$. Let $a$ be arbitrary element in $P_x$.
From $a \in P_x$ we know that $aRx$. Since $X$ is partial order on $A$
and $aRx$ and $xRy$, it follows that $aRy$. So, $a \in P_y$. Since $a$
is an arbitrary element it follows that $P_x \subseteq P_y$.

$\Leftarrow$ Suppose $P_x \subseteq P_y$. Now clearly $x \in P_x$
since $R$ is partial order and $(x,x) \in R$. From $P_x \subseteq
P_y$, it follows that $x \in P_y$. From the definition of $P_y$, then
$xRy$ is true.

\section{Problem 11}
\begin{align*}
  D = \{(x,y) \in Z^{+} \times Z^{+} \mid x \text{divides} y \} \\
  B = \{x \in Z \mid x > 1\} \\
  \text{Smallest element:} \forall x \in B(x R b) \\
  \text{There is no smallest element in B.} \\
  \text{Minimal element:} \neg x \in B(xRb \land x \neq b) \\
  Prime numbers.
\end{align*}

\section{Problem 12}
First, let's get some things straight: \\
Let $S = \{(X,Y) \in P(R) \times P(R) \mid X \subseteq Y\}$ \\
Let $F = \{X \subseteq R \mid X \neq \emptyset \land \forall x \forall
y ((x \in X \land x < y) \implies y \in X)\}$ \\

To show that it has no minimal element, we have to prove $\exists x
\in B(xSb \land x \neq b)$.

Suppose $C = \{y \in R \mid y \geq 1\}$. Now let $D = \{y \in R \mid y
\geq 0\}$. Now $D \in F$ and $(C,D) \in S$ and $C \neq D$. So $F$
doesn't have any minimal element.

\section{Problem 13}
Suppose $R$ is a partial order on $A$.
\subsection{Proof of reflexivity}
Let $a$ be an arbitrary element in $A$. Since $R$ is partial order, it
follows that $(a,a) \in R$. So, $(a,a) \in R^{-1}$.
\subsection{Proof of transitive}
Let $a,b$ and $c$ be arbitrary element in $A$ such that $(a,b) \in
R^{-1}$ and $(b,c) \in R^{-1}$. It follows that $(b,a) \in R$ and
$(c,b) \in R$. Since $R$ is partial order, it follows that $(c,a) \in
R$. So, $(a,c) \in R^{-1}$. Since $a$ and $c$ are arbitrary element we
can conclude that $R^{-1}$ is transitive.
\subsection{Proof of anti-symmetric}
Let $a$ and $b$ be arbitrary element in $A$ such that $(a,b) \in
R^{-1}$ and $(b,a) \in R^{-1}$. It follows that $(b,a) \in R$ and
$(a,b) \in R$. Since $R$ is partial order it follows that $a = b$.
Since $a$ and $b$ are arbitrary element, $R^{-1}$ is anti-symmetric.
\subsection{Proof of total order}
Let $a$ and $b$ be arbitrary element in $A$. Since $R$ is total order,
we know that $(a,b) \in R$ or $(b,a) \in R$. It follows that $(b,a)
\in R^{-1}$ or $(a,b) \in R^{-1}$. Since $a$ and $b$ are arbitrary, we
can conclude that $R^{-1}$ is a total order.

\section{Problem 14}
\subsection{Solution (a)}
Suppose $R$ is a partial order on $A$ and $B \subseteq A$ and $b \in
B$.
$\Rightarrow$ Suppose $b$ is the R-largest element of $B$. Let $x$ be
an arbitrary element in $B$. Since $b$ is the largest element, it
follows that $xRb$. So, $(b,x) \in R^{-1}$. Since $x$ was arbitrary,
$\forall x \in B(b R^{-1} x)$. Therefore $b$ is the $R$ smallest
element of $B$.
$\Leftarrow$ Suppose $b$ is the $R^{-1}$ smallest element of $R$. Let
$x$ be an arbitrary element of $B$. Since $b$ is the $R^{-1}$ smallest
element of $B$, it follows that $(b,x) \in R^{-1}$. So, $(x,b) \in R$.
Since $x$ was arbitrary, we get $\forall x \in B(xRb)$. Therefore $b$
is the R-largest element of $B$.

\subsection{Solution (b)}
$\Rightarrow$ Suppose $b$ is an R-maximal element of $B$. Let $x$ be
an arbitrary element in $B$ such that $xR^{-1}b$.. Since $b$ is an
R-maximal element of $B$, it follows that $bRx \implies x = b$. From
$xR^{-1}b$, it follows that $(b,x) \in R$. Then from modulus pollens,
$x = b$. Hence $b$ is an $R^{-1}$ minimal element of $B$.
$\Leftarrow$ Suppose $b$ is an $R^{-1}$ minimal element of $B$. Let
$x$ be an arbitrary element in $B$ such that $bRx$. Since $b$ is an
$R^{-1}$ minimal element of $B$, it follows that $xR^{1}b \implies x =
b$. Then from modulus pollens, $x = b$. Hence $b$ is an R-maximal
element of B.

\section{Problem 15}
Suppose $R_1$ and $R_2$ are partial order on $A$, $R_1 \subseteq R_2$,
$B \subseteq A$ and $b \in B$.
\subsection{Solution (a)}
Suppose $b$ is the $R_1$ smallest element of $B$. Let $x$ be an
arbitrary element of $B$. Since $b$ is the $R_1$ smallest element of
$B$, it follows that $(b,x) \in R_1$. From $R_1 \subseteq R_2$, it
follows that $(b,x) \in R_2$. Since $x$ is arbitrary we have $\forall
x \in B(bR_2x)$. Therefore $b$ is the $R_2$ smallest element of $B$.
\subsection{Solution (b)}
Suppose $b$ is an $R_2$ minimal element of $B$. Let $x$ be arbitrary
element of $B$ and suppose $xR_1b$. Since $b$ is an $R_2$ minimal
element, it follows that $xR_2b \implies x = b$. From $(x,b) \in R_1$
and $R_1 \subseteq R_2$ it follows that $(x,b) \in R_2$. Applying
modulus pollens, we get $x = b$. Since $x$ was arbitrary in $B$, we
can conclude that $b$ is an $R_1$ minimal element of $B$.

\section{Problem 16}
Suppose $R$ is a partial order on $A$, $B \subseteq A$ and $b \in B$.
Suppose $b$ is the largest element of $B$. Let $x$ be an arbitrary of
$B$ and $bRx$. Since $b$ is the largest element of $B$, it follows that
$xRb$. Since $R$ is partial order on $A$, we can conclude from $bRx$
and $xRb$ that $x = b$ from the anti-symmetric property. Since $x$ was
arbitrary we can conclude that $b$ is the maximal element of $B$.\\
To see that it's the only maximal element, suppose $c$ is also a
maximal element. Since $b$ is the largest element of $B$, it follows
$cRb$. Since $c$ is the maximal element, we have $\forall x \in B(cRx
\implies c = x)$. Applying universal instantiation, we get $cRb
\implies c = b$. By modulus pollens, we get $c = b$. Hence $b$ is the
only largest element of $B$.

\section{Problem 17}
do it

\section{Problem 18}
do it

\section{Problem 19}
\subsection{Solution (a)}
Proof is wrong. Universal generalization doesn't distribute over
disjunction.

\subsection{Solution (b)}
Counterexample:
\begin{align*}
  R = \{(x,y) \in R \times R \mid x \leq y\} \\
  B = \{1,2,3\}
\end{align*}
Smallest element of $B = 1$. \\
Largest element of $B = 3$. \\
$2$ is neither smallest nor largest element of $B$.
  
\end{document}

