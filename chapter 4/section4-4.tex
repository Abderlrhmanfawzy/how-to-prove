%%%%%%%%%%%%%%%%%%%%%%%%%%%%%%%%%%%%%%%%%
% Author: Sibi <sibi@psibi.in>
%%%%%%%%%%%%%%%%%%%%%%%%%%%%%%%%%%%%%%%%%
\documentclass{article}
\usepackage{graphicx}
\usepackage{verbatim}
\usepackage{amsmath}
\usepackage{amsfonts}
\usepackage{amssymb}
\usepackage{tabularx}
\setlength\parskip{\baselineskip}
\begin{document}
\title{Chapter 4 (Section 4.3)}
\author{Sibi}
\date{\today}
\maketitle
\newpage

\section{Problem 1}
\subsection{Solution (a)}

\begin{itemize}
\item Reflexive
\item Transitive
\item Anti-symmetric (And hence partial order)
\item Not total order because $(a,c) \lor (c,a) \notin R$
\end{itemize}

\subsection{Solution (b)}
\begin{itemize}
\item Reflexive 
\item Transitive
\item Not Anti-symmetric. (Example: $(2,-2)$)
\end{itemize}

\subsection{Solution (c)}
\begin{itemize}
\item Reflexive
\item Transitive
\item Anti symmetric
\item Not total order. (Example: $(-2,2)$)
\end{itemize}

\section{Problem 2}
\subsection{Solution (a)}
\begin{itemize}
\item Reflexive
\item Transitive
\item Anti-symmetric
\item Not total order. Example: $(go, haskell)$
\end{itemize}

\subsection{Solution (b)}
\begin{itemize}
\item Reflexive
\item Transitive
\item Not anti-symmetric (Example: $(aba, aa)$)
\end{itemize}

\subsection{Solution (c)}
\begin{itemize}
\item Reflexive
\item Transitive
\item If there is two country with same population, then it is not anti-symmetric.
\end{itemize}

\section{Problem 3}
\subsection{Solution (a)}
\begin{itemize}
\item Smallest element = 2
\item Minimal element = 2
\item Largest = 4
\item Maximal =  4
\item Upper bound = Nothing
\item Lower bound =  \{1,2\}
\item glb = 2
\end{itemize}

\subsection{Solution (b)}
\begin{itemize}
\item Smallest element = 1
\item Minimal element = 1
\item Largest = No number
\item Maximal = No number
\item Upper bound = $\{2, ... R\}$
\item Lower bound = $\{... 1\}$
\item glb = 1
\item lub = 2
\end{itemize}

\subsection{Solution (c)}
\begin{itemize}
\item Smallest element = $\emptyset$
\item Minimal element = $\emptyset$
\item Largest = Nothing
\item Maximal =  $\{x \in \mathcal{P}(\mathbb{N} \mid \text{x has
    exactly 5 elements}\}$
\item Upper bound = $\{emptyset\}$
\item Lower bound = No upper bound
\item glb = $\emptyset$
\end{itemize}

\section{Problem 4}

($\Rightarrow$) Suppose $R$ is anti-symmetric and symmetric. Let
$(a,b)$ be an arbitrary element in $R$. Since $R$ is symmetric it
follows that $(b,a) \in R$. Now from anti-symmetric property it
follows that $a = b$. Since $a = b$, it follows that $(a,b) \in i_A$.
Since $(a,b)$ was arbitrary, it follows that $R \subseteq i_A$.

($\Leftarrow$) Suppose $R \subseteq i_A$. Let $a,b$ be arbitrary
element in $A$ such that $(a,b) \in R$. From $R \subseteq i_A$, it
follows that $a = b$. Therefore $R$ is symmetric. Also, since
$(a,b) \in R$ and $(b,a) \in R$ and $a = b$, it follows that $R$ is
anti-symmetric.

\section{Problem 5}
Suppose $R$ is partial order on $A$ and $B \subseteq A$.

\subsection{Proof for reflexivity}

Let $b$ be arbitrary element in B. From $B \subseteq A$, it follows
that $b \in A$. Since we know that $R$ is partial order on $A$ it
follows that $(b,b) \in R$. Also, $(b,b) \in B \times B$. Therefore
$(b,b) \in R \cap (B \times B)$. Since $b$ is arbitrary, $R \cap (B
\times B)$ is reflexive.

\subsection{Proof for transitivity}

Let $a,b$ and $c$ be arbitrary element in $B$ such that $(a,b) \in R
\cap B \times B$ and $(b,c) \in R \cap B \times B$. Since $a$ and $c$
is an element of $B$, it follows that $(a,c) \in B \times B$. We know
that $R$ is partial order on $A$ and $(a,b) \in R$ and $(b,c) \in R$.
By applying transitive property on $R$ it follows that $(a,c) \in R$.
Therefore $(a,c) \in R \cap (B \times B)$. Hence $R \cap (B \times B)$
is transitive.

\subsection{Proof for anti-symmetric}
Let $a,b$ be arbitrary element in $B$ such that $(a,b) \in R \cap B
\times B$ and $(b,a) \in R \cap B \times B$. Since $R$ is
anti-symmetric it follows that $a = b$. Hence $R \cap B \times B$ is
anti symmetric.

\end{document}

