%%%%%%%%%%%%%%%%%%%%%%%%%%%%%%%%%%%%%%%%%
% Author: Sibi <sibi@psibi.in>
%%%%%%%%%%%%%%%%%%%%%%%%%%%%%%%%%%%%%%%%%
\documentclass{article}
\usepackage{graphicx}
\usepackage{verbatim}
\usepackage{amsmath}
\usepackage{amsfonts}
\usepackage{amssymb}
\usepackage{tabularx}
\setlength\parskip{\baselineskip}
\begin{document}
\title{Chapter 4 (Section 4.3)}
\author{Sibi}
\date{\today}
\maketitle
\newpage

\section{Problem 1}
\subsection{Solution (a)}

\begin{itemize}
\item Reflexive
\item Transitive
\item Anti-symmetric (And hence partial order)
\item Not total order because $(a,c) \lor (c,a) \notin R$
\end{itemize}

\subsection{Solution (b)}
\begin{itemize}
\item Reflexive 
\item Transitive
\item Not Anti-symmetric. (Example: $(2,-2)$)
\end{itemize}

\subsection{Solution (c)}
\begin{itemize}
\item Reflexive
\item Transitive
\item Anti symmetric
\item Not total order. (Example: $(-2,2)$)
\end{itemize}

\section{Problem 2}
\subsection{Solution (a)}
\begin{itemize}
\item Reflexive
\item Transitive
\item Anti-symmetric
\item Not total order. Example: $(go, haskell)$
\end{itemize}

\subsection{Solution (b)}
\begin{itemize}
\item Reflexive
\item Transitive
\item Not anti-symmetric (Example: $(aba, aa)$)
\end{itemize}

\subsection{Solution (c)}
\begin{itemize}
\item Reflexive
\item Transitive
\item If there is two country with same population, then it is not anti-symmetric.
\end{itemize}

\section{Problem 3}
\subsection{Solution (a)}
\begin{itemize}
\item Smallest element = 2
\item Minimal element = 2
\item Largest = 4
\item Maximal =  4
\item Upper bound = Nothing
\item Lower bound =  \{1,2\}
\item glb = 2
\end{itemize}

\subsection{Solution (b)}
\begin{itemize}
\item Smallest element = 1
\item Minimal element = 1
\item Largest = No number
\item Maximal = No number
\item Upper bound = $\{2, ... R\}$
\item Lower bound = $\{... 1\}$
\item glb = 1
\item lub = 2
\end{itemize}

\subsection{Solution (c)}
\begin{itemize}
\item Smallest element = $\emptyset$
\item Minimal element = $\emptyset$
\item Largest = Nothing
\item Maximal =  $\{x \in \mathcal{P}(\mathbb{N} \mid \text{x has
    exactly 5 elements}\}$
\item Upper bound = $\{emptyset\}$
\item Lower bound = No upper bound
\item glb = $\emptyset$
\end{itemize}

\section{Problem 4}

($\Rightarrow$) Suppose $R$ is anti-symmetric and symmetric. Let
$(a,b)$ be an arbitrary element in $R$. Since $R$ is symmetric it
follows that $(b,a) \in R$. Now from anti-symmetric property it
follows that $a = b$. Since $a = b$, it follows that $(a,b) \in i_A$.
Since $(a,b)$ was arbitrary, it follows that $R \subseteq i_A$.

($\Leftarrow$) Suppose $R \subseteq i_A$. Let $a,b$ be arbitrary
element in $A$ such that $(a,b) \in R$. From $R \subseteq i_A$, it
follows that $a = b$. Therefore $R$ is symmetric. Also, since
$(a,b) \in R$ and $(b,a) \in R$ and $a = b$, it follows that $R$ is
anti-symmetric.

\section{Problem 5}
Suppose $R$ is partial order on $A$ and $B \subseteq A$.

\subsection{Proof for reflexivity}

Let $b$ be arbitrary element in B. From $B \subseteq A$, it follows
that $b \in A$. Since we know that $R$ is partial order on $A$ it
follows that $(b,b) \in R$. Also, $(b,b) \in B \times B$. Therefore
$(b,b) \in R \cap (B \times B)$. Since $b$ is arbitrary, $R \cap (B
\times B)$ is reflexive.

\subsection{Proof for transitivity}

Let $a,b$ and $c$ be arbitrary element in $B$ such that $(a,b) \in R
\cap B \times B$ and $(b,c) \in R \cap B \times B$. Since $a$ and $c$
is an element of $B$, it follows that $(a,c) \in B \times B$. We know
that $R$ is partial order on $A$ and $(a,b) \in R$ and $(b,c) \in R$.
By applying transitive property on $R$ it follows that $(a,c) \in R$.
Therefore $(a,c) \in R \cap (B \times B)$. Hence $R \cap (B \times B)$
is transitive.

\subsection{Proof for anti-symmetric}
Let $a,b$ be arbitrary element in $B$ such that $(a,b) \in R \cap B
\times B$ and $(b,a) \in R \cap B \times B$. Since $R$ is
anti-symmetric it follows that $a = b$. Hence $R \cap B \times B$ is
anti symmetric.

\section{Problem 6}
\subsection{Solution (a)}
Suppose $R_1$ and $R_2$ are partial orders on $A$.

\subsubsection{Proof for reflexivity}

Let $a$ be an arbitrary element in $A$. Since $R_1$ and $R_2$ are
partial orders on $A$ it follows that $(a,a) \in R_1$ and $(a,a) \in
R_2$. So, $(a,a) \in R_1 \cap R_2$. Since $a$ was arbitrary, we can
conclude that $R_1 \cap R_2$ is reflexive.

\subsubsection{Proof for transitivity}

Let $a,b$ and $c$ be an arbitrary element in $A$ such that $(a,b) \in
R_1 \cap R_2$ and $(b,c) \in R_1 \cap R_2$. Since $R_1$ and $R_2$ are
partial orders it follows that $(a,c) \in R_1$ and $(a,c) \in R_2$.
So, $(a,c) \in R_1 \cap R_2$. Therefore, $R_1 \cap R_2$ is transitive.

\subsubsection{Proof for anti-symmetric}
Let $a,b$ be an arbitrary element in $A$ such that $(a,b) \in R_1 \cap
R_2$ and $(b,a) \in R_1 \cap R_2$. Since $R_1$ and $R_2$ are anti
symmetric it follows that $a = b$. Therefore $R_1 \cap R_2$ is anti
symmetric.

\subsection{Solution (b)}

Counterexample:

\begin{align*}
  R_1 = \{(1,2), (1,1), (2,2)\} \\
  R_2 = \{(2,1), (1,1), (2,2)\} \\
\end{align*}

$R_1 \cup R_2$ isn't anti-symmetric because $1 \neq 2$.

\section{Problem 7}
Suppose $R_1$ is a partial order on $A_1$. Suppose $R_2$ is a partial
order on $A_2$ and $A_1 \cap A_2 = \emptyset$.

\subsection{Solution (a)}

\subsubsection{Proof of reflexivity}
Let $a$ be an arbitrary element in $A_1 \cup A_2$. Let us consider the
cases separately:

Case 1. $a \in A_1$ Since $R_1$ is partial order on $A_1$ it follows
$(a,a) \in R_1$. So $(a,a) \in R_1 \cup R_2$.
Case 2. $a \in A_2$ Since $R_2$ is partial order on $A_2$ it follows
that $(a,a) \in R_2$. So $(a,a) \in R_1 \cup R_2$.

\subsubsection{Proof of transitivity}
Let $a,b$ and $c$ be an arbitrary element in $A_1 \cup A_2$ such that
$(a,b) \in R_1 \cup R_2$ and $(b,c) \in R_1 \cup R_2$. Let us consider
the cases:

Case 1. $(a,b) \in R_1$. Now this itself has two cases with it: $(b,c)
\in R_1$ and $(b,c) \in R_2$. Now when $(b,c) \in R_1$, then since
$R_1$ is partial order it follows that $(a,c) \in R_1 \cup R_2$. Now
when $(b,c) \in R_2$, this means that $(b,c) \notin A_1$ since $A_1
\cap A_2 = \emptyset $. In this case it is vacuously true.

Case 2. $(a,b) \in R_2$. Now this itself has two cases within it.
$(b,c) \in R_1$ or $(b,c) \in R_2$. If $(b,c) \in R_1$, then $(b,c)
\notin R_2$ since $A_1 \cap A_2 = \emptyset$. So it is vacuously true.
On the other hand if $(b,c) \in R_2$, then since $R_2$ is partial
order we can conclude that $(a,c) \in R_1 \cup R_2$.

\subsubsection{Proof of Anti-symmetric}
Let $a$ and $b$ be arbitrary element in $A_1 \cup A_2$ such that
$(a,b) \in R_1 \cup R_2$ and $(b,a) \in R_1 \cup R_2$. Let us consider
the cases separately:

Case 1. $(a,b) \in R_1$. Now let's consider the sub-cases:
\begin{itemize}
\item $(b,a) \in R_1$. Since $R_1$ is partial order, it follows that
  $a = b$.
\item $(b,a) \in R_2$. From $A_1 \cap A_2 = \emptyset$, it follows
  that $(b,a) \notin R_1$. So, it is vacuously true.
\end{itemize}

Case 2. $(a,b) \in R_2$. Now let's consider the sub-cases:
\begin{itemize}
\item $(b,a) \in R_1$ From $A_1 \cap A_2$ it follows that $(b,a)
  \notin R_2$. So it is vacuously true.
\item $(b,a) \in R_2$ Since $R_2$ is partial order it follows that $b
  = a$.
\end{itemize}

\subsection{Solution (b)}
\subsubsection{Proof of reflexivity}
Let $a$ be an arbitrary element in $A_1 \cup A_2$. From $A_1 \cap A_2
= \emptyset$ it follows that either $a \in A_1$ or $A \in A_2$ but not
both. Since $R_1$ and $R_2$ are both partial order it follows that
$(a,a) \in R_1 \cup R_2$. Since $a$ is arbitrary, we can conclude that
$R_1 \cup R_2 \cup (A_1 \times A_2)$ is reflexive.

\subsubsection{Proof of transitivity}
Let $a,b$ and $c$ be an arbitrary element in $A_1 \cup A_2$ such that
$(a,b) \in R_1 \cup R_2 \cup (A_1 \times A_2)$ and $(b,c) \in R_1 \cup
R_2 \cup (A_1 \times A_2)$. Let us consider the cases separately.

Case 1. $(a,b) \in R_1$. Now let's consider the sub cases:
\begin{itemize}
\item $(b,c) \in R_1$ Since $R_1$ is partial order it follows that
  $(a,c) \in R$. So $(a,c) \in R_1 \cup R_2 \cup (A_1 \times A_2)$.
\item $(b,c) \in R_2$ From $A_1 \cap A_2 = \emptyset$, it follows that
  $(b,c) \notin R_1$. So it is vacuously true.
\item $(b,c)\in A_1 \times A_2$ It follows that $b \in A_1$ and $c \in
  A_2$. Since $c \in A_2$ from $A_1 \cap A_2 = \emptyset$, we can
  conclude that $c \notin A_1$. So, $(b,c) \notin R_1$ since $R_1$ is
  a relation on $A_1$. So it is vacuously true.
\end{itemize}
Prove case $2$ similarly.

\subsubsection{Proof of Anti symmetric}
Let $a,b$ be arbitrary element in $A_1 \cup A_2$ such that $(a,b) \in
R_1 \cup R_2 \cup (A_1 \times A_2)$ and $(b,c) \in R_1 \cup R_2 \cup
(A_1 \times A_2)$.

Case 1. $(a,b) \in R_1$. Now let's consider the sub-cases:
\begin{itemize}
\item $(b,a) \in R_1$. Since $R_1$ is partial order on $A_1$ it
  follows that $a = b$.
\item $(b,a) \in R_2$ Now $(b,a) \notin R_1$ since $A_1 \cap A_2 =
  \emptyset$. So it is vacuously true.
\item $(b,a) \in A_1 \times A_2$. Since $a \in A_2$, it follows that
  $(b,a) \notin R_1$. So it is vacuously true.
\end{itemize}
Proof case $2$ similarly.

\subsection{Solution (c)}
\subsubsection{Part (i)}
\begin{align*}
  A_1 = \{1\} \\
  A_2 = \{2\} \\
  R_1 = \{(1,1)\} \\
  R_2 = \{(2,2)\} \\
  (1,2) \lor (2,1) \notin R_1 \cup R_2
\end{align*}

\subsubsection{Part (ii)}
Let $a$ and $b$ be arbitrary element in $A_1 \cup A_2$. Let us
consider the cases separately:

Case 1. $a \in A_1$ It has two sub cases:
\begin{itemize}
\item $b \in A_1$ Then since $R_1$ is total order it follows that
  $(a,b) \lor (b,a) \in R_1$. So, we can conclude that $(a,b) \lor
  (b,a) \in R_1 \cup R_2 \cup (A_1 \times A_2)$
\item $b \in A_2$ Then it follows that $(a,b) \in A_1 \times A_2$. So
  $(a,b) \lor (b,a) \in R_1 \cup R_2 \cup A_1 \times A_2$.
\end{itemize}

Case 2 can be proved similarly.
\end{document}

