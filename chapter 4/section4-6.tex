%%%%%%%%%%%%%%%%%%%%%%%%%%%%%%%%%%%%%%%%%
% Author: Sibi <sibi@psibi.in>
%%%%%%%%%%%%%%%%%%%%%%%%%%%%%%%%%%%%%%%%%
\documentclass{article}
\usepackage{graphicx}
\usepackage{verbatim}
\usepackage{amsmath}
\usepackage{amsfonts}
\usepackage{amssymb}
\usepackage{tabularx}
\setlength\parskip{\baselineskip}
\begin{document}
\title{Chapter 4 (Section 4.6)}
\author{Sibi}
\date{\today}
\maketitle
\newpage

\section{Problem 1}

\begin{align*}
  \{\{1\},\{2\},\{3\}\}
  \{\{1,2\},\{3\}\}
  \{\{1,3\},\{2\}\}
  \{\{2,3\},\{1\}\}
  \{\{1,2,3\}\}
\end{align*}

\section{Problem 2}

Identity relation and $A \times A$

\section{Problem 3}

\subsection{Solution (a)}
$R = \{(x,y) \in W \times W \mid the words x and y start with the same
letter \}$

\begin{itemize}
\item Reflexive: Let $x$ be an arbitrary element in $W$. Then $(x,x)
  \in R$ since both start from the same letter as they are same.
\item Symmetric: Let $x,y$ be an arbitrary element in $W$ such that
  $xRy$. Now we know that $x$ and $y$ start with the same. So, $yRx$.
\item Transitive: Let $x,y$ and $z$ be arbitrary element in $W$ such
  that $xRy$ and $yRz$. Both $x$ and $y$ start with the same letter.
  Similarly, $y$ and $z$ start with the same letter. So, $xRz$.
\end{itemize}
\end{document}


