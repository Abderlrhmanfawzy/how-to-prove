%%%%%%%%%%%%%%%%%%%%%%%%%%%%%%%%%%%%%%%%%
% Author: Sibi <sibi@psibi.in>
%%%%%%%%%%%%%%%%%%%%%%%%%%%%%%%%%%%%%%%%%
\documentclass{article}
\usepackage{graphicx}
\usepackage{verbatim}
\usepackage{amsmath}
\usepackage{amsfonts}
\usepackage{amssymb}
\usepackage{tabularx}
\setlength\parskip{\baselineskip}
\begin{document}
\title{Chapter 4 (Section 4.6)}
\author{Sibi}
\date{\today}
\maketitle
\newpage

\section{Problem 1}

\begin{align*}
  \{\{1\},\{2\},\{3\}\}
  \{\{1,2\},\{3\}\}
  \{\{1,3\},\{2\}\}
  \{\{2,3\},\{1\}\}
  \{\{1,2,3\}\}
\end{align*}

\section{Problem 2}

Identity relation and $A \times A$

\section{Problem 3}

\subsection{Solution (a)}
$R = \{(x,y) \in W \times W \mid \text{the words x and y start with the same
letter} \}$

\begin{itemize}
\item Reflexive: Let $x$ be an arbitrary element in $W$. Then $(x,x)
  \in R$ since both start from the same letter as they are same.
\item Symmetric: Let $x,y$ be an arbitrary element in $W$ such that
  $xRy$. Now we know that $x$ and $y$ start with the same. So, $yRx$.
\item Transitive: Let $x,y$ and $z$ be arbitrary element in $W$ such

  that $xRy$ and $yRz$. Both $x$ and $y$ start with the same letter.
  Similarly, $y$ and $z$ start with the same letter. So, $xRz$.
\end{itemize}

\subsection{Solution (b)}
Counterexample for transitive:
\begin{align*}
  ("hello","hask") \in S, ("hask", "kitten") \in S \\
  ("hello", kitten) \notin S
\end{align*}

\subsection{Solution (c)}
$T = \{(x,y) \in W \times W \mid \text{the words x and y have the same
  number of letters}\}$

\begin{itemize}
\item Reflexive: Let $x$ be an arbitrary element in $W$. It follows
  that $xTx$ since it has the same number of letters.
\item Symmetric: Let $x$ and $y$ be arbitrary element in $W$ such that
  $xTy$. Now from $xTy$, we get to know that $x$ and $y$ have the same
  number of letters. So, $yTx$.
\item Transitive: Let $x,y$ and $z$ be arbitrary element in $T$ such
  that $xTy$ and $yTz$. Now we know $x$ and $y$ have same number of
  letters. Similarly from $yTz$, we know that $y$ and $z$ have the
  same number of letters. So, $x$ and $z$ has the same number of
  letters. So, $xRz$.
\end{itemize}

\section{Problem 4}
\subsection{Solution (a)}
Counterexample for Symmetric property:

\begin{align*}
  (10,5) \in R \\
  (5,10) \notin R \text{ since} -5 \notin \mathbb{N}
\end{align*}

\subsection{Solution (b)}
\begin{itemize}
\item Reflexive. Let $x$ be an arbitrary element in $\mathbb{R}$. Since
  $x - x = 0$ and $0 \in \mathbb{Q}$, it follows that $(x,x) \in S$.
\item Symmetric. Let $x, y$ be an arbitrary element in $\mathbb{R}$
  such that $xSy$. It follows that $x - y \in \mathbb{Q}$. Now since
  $y - x \in \mathbb{Q}$, it follows that $ySx$.
\item Transitive. Let $x,y$ and $z$ be arbitrary element in
  $\mathbb{R}$ such that $xSy$ and $ySz$. It follows that $x - y$ and
  $y - z$ are in $\mathbb{Q}$. Adding two rational numbers gives us a
  rational number, so $xSz$.
\end{itemize}

Equivalence classes is the partition of real numbers whose difference
is a rational number

\subsection{Solution (c)}
It is an equivalence relation.

\section{Problem 5}
$P = $ Set of all people \\
$B = \{(p,q) \in P \times P \mid \text{p and q have same birthday}\}$ \\
$P_d = { p \in P \mid  \text{p has birthday on day d}}$ \\

From the definition of $P$ modulo $B$, we get $P/R = \{X \subseteq P
\mid \exists x \in P(X = [x]_R)\}$. Now $P_d$ would be an equivalent
class if there exists at least one person who was born on day $d$. So,
$P/R = \{P_d \mid d \in D\}$.

\section{Problem 6}
$S = \{(s,t) \in T \times T \mid \text{the triangles s and t are similar}$

\begin{itemize}
\item Reflexive. Let $x$ be an arbitrary element in $T$. Since a
  triangle is similar with itself, $S$ is reflexive.
\item Symmetric. Let $x,y$ be an arbitrary element in $T$ such that
  $xSy$. We know that $x$ and $y$ are similar. So, $ySx$.
\item Transitive. Let $x,y$ and $z$ be arbitrary element in $T$ such
  that $xSy$ and $ySz$. We know that $x$ and $y$ are similar to each
  other. Similarly $y$ and $z$ are similar to each other. So. $x$ and
  $z$ are similar to each other. So, $xSz$. Since $x,y$ and $z$ are
  arbitrary we can conclude that $S$ is transitive.
\end{itemize}

\section{Problem 7}
\begin{itemize}
\item Symmetric. Let $x,y$ be arbitrary elements in $A$ such that
  $xRy$. So $(x,y) \in R$. Since $R = \cup_{X \in F}(X \times X)$, it
  follows that for some $X \in F$, $(x,y) \in X \times X$. So, $x \in
  X$ and $y \in X$. But then $(y,x) \in X \times X$. So, $(y,x) \in
  \cup_{X \in F}(X \times X)$. Since $x$ and $y$ are arbitrary, it
  follows that $R$ is symmetric.
\item Transitive. Let $x,y$ and $z$ be arbitrary elements in $A$ such
  that $xRy$ and $yRz$. Since $R = \cup_{X \in F}(X \times X)$, it
  follows that for some $X \in F$, $(x,y) \in X \times X$. Similarly
  for some $Y \in F$, $(y,z) \in Y \times Y$. So, $y \in X$ and $y \in
  Y$. But from the definition of partition, we know that $F$ is
  pairwise disjoint. So, $X=Y$ which leads us to $(x,z) \in \cup_{X
    \in F}(X \times X)$. Since $x,y$ and $z$ are arbitrary, it follows
  that $R$ is transitive.
\end{itemize}

\section{Problem 8}
\begin{itemize}
\item $(\Rightarrow)$ Let $(x,y)$ be an arbitrary element in $R$ such
  that $x \in A$ and $y \in A$. From $xRy$, it follows that they are
  part of the same equivalence class, so $x \in [x]_R$ and
  $y \in [y]_R$. From $A/R = A/S$, it follows that $[x]_R \in A/R$ and
  $[y]_R \in A/S$. Since they are from the same equivalence class,
  $(x,y) \in S$. Since $x$ and $y$ are arbitrary, it follows that
  $R \subseteq S$.
\item $(\Leftarrow)$. Follows the same reasoning in other direction.
\end{itemize}

\section{Problem 9}
\begin{itemize}
\item $(\Rightarrow)$ Let $(x,y)$ be an arbitrary element in $S$ such
  that $x \in A$ and $y \in B$. From $S=\cup_{X \in F}(X \times X)$,
  it follows that there is some element $X in F$ such that $x \in X$
  and $y \in X$. Now $F = A/R$, so $X$ is an equivalence class of $R$
  on $A$. So, $X = [x]_R = [y]_R$. Hence $(x,y) \in R$, so $S
  \subseteq R$.
\item $(\Leftarrow)$. Follow the same reasoning in other direction
\end{itemize}
\end{document}


