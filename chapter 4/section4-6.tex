%%%%%%%%%%%%%%%%%%%%%%%%%%%%%%%%%%%%%%%%%
% Author: Sibi <sibi@psibi.in>
%%%%%%%%%%%%%%%%%%%%%%%%%%%%%%%%%%%%%%%%%
\documentclass{article}
\usepackage{graphicx}
\usepackage{verbatim}
\usepackage{amsmath}
\usepackage{amsfonts}
\usepackage{amssymb}
\usepackage{tabularx}
\usepackage{mathtools}
\setlength\parskip{\baselineskip}
\begin{document}
\title{Chapter 4 (Section 4.6)}
\author{Sibi}
\date{\today}
\maketitle

% See here: http://tex.stackexchange.com/a/43009/69223
\DeclarePairedDelimiter\abs{\lvert}{\rvert}%
\DeclarePairedDelimiter\norm{\lVert}{\rVert}%

% Swap the definition of \abs* and \norm*, so that \abs
% and \norm resizes the size of the brackets, and the 
% starred version does not.
\makeatletter
\let\oldabs\abs
\def\abs{\@ifstar{\oldabs}{\oldabs*}}
%
\let\oldnorm\norm
\def\norm{\@ifstar{\oldnorm}{\oldnorm*}}
\makeatother
\newpage

\section{Problem 1}

\begin{align*}
  \{\{1\},\{2\},\{3\}\}
  \{\{1,2\},\{3\}\}
  \{\{1,3\},\{2\}\}
  \{\{2,3\},\{1\}\}
  \{\{1,2,3\}\}
\end{align*}

\section{Problem 2}

Identity relation and $A \times A$

\section{Problem 3}

\subsection{Solution (a)}
$R = \{(x,y) \in W \times W \mid \text{the words x and y start with the same
letter} \}$

\begin{itemize}
\item Reflexive: Let $x$ be an arbitrary element in $W$. Then $(x,x)
  \in R$ since both start from the same letter as they are same.
\item Symmetric: Let $x,y$ be an arbitrary element in $W$ such that
  $xRy$. Now we know that $x$ and $y$ start with the same. So, $yRx$.
\item Transitive: Let $x,y$ and $z$ be arbitrary element in $W$ such

  that $xRy$ and $yRz$. Both $x$ and $y$ start with the same letter.
  Similarly, $y$ and $z$ start with the same letter. So, $xRz$.
\end{itemize}

\subsection{Solution (b)}
Counterexample for transitive:
\begin{align*}
  ("hello","hask") \in S, ("hask", "kitten") \in S \\
  ("hello", kitten) \notin S
\end{align*}

\subsection{Solution (c)}
$T = \{(x,y) \in W \times W \mid \text{the words x and y have the same
  number of letters}\}$

\begin{itemize}
\item Reflexive: Let $x$ be an arbitrary element in $W$. It follows
  that $xTx$ since it has the same number of letters.
\item Symmetric: Let $x$ and $y$ be arbitrary element in $W$ such that
  $xTy$. Now from $xTy$, we get to know that $x$ and $y$ have the same
  number of letters. So, $yTx$.
\item Transitive: Let $x,y$ and $z$ be arbitrary element in $T$ such
  that $xTy$ and $yTz$. Now we know $x$ and $y$ have same number of
  letters. Similarly from $yTz$, we know that $y$ and $z$ have the
  same number of letters. So, $x$ and $z$ has the same number of
  letters. So, $xRz$.
\end{itemize}

\section{Problem 4}
\subsection{Solution (a)}
Counterexample for Symmetric property:

\begin{align*}
  (10,5) \in R \\
  (5,10) \notin R \text{ since} -5 \notin \mathbb{N}
\end{align*}

\subsection{Solution (b)}
\begin{itemize}
\item Reflexive. Let $x$ be an arbitrary element in $\mathbb{R}$. Since
  $x - x = 0$ and $0 \in \mathbb{Q}$, it follows that $(x,x) \in S$.
\item Symmetric. Let $x, y$ be an arbitrary element in $\mathbb{R}$
  such that $xSy$. It follows that $x - y \in \mathbb{Q}$. Now since
  $y - x \in \mathbb{Q}$, it follows that $ySx$.
\item Transitive. Let $x,y$ and $z$ be arbitrary element in
  $\mathbb{R}$ such that $xSy$ and $ySz$. It follows that $x - y$ and
  $y - z$ are in $\mathbb{Q}$. Adding two rational numbers gives us a
  rational number, so $xSz$.
\end{itemize}

Equivalence classes is the partition of real numbers whose difference
is a rational number

\subsection{Solution (c)}
It is an equivalence relation.

\section{Problem 5}
$P = $ Set of all people \\
$B = \{(p,q) \in P \times P \mid \text{p and q have same birthday}\}$ \\
$P_d = { p \in P \mid  \text{p has birthday on day d}}$ \\

From the definition of $P$ modulo $B$, we get $P/R = \{X \subseteq P
\mid \exists x \in P(X = [x]_R)\}$. Now $P_d$ would be an equivalent
class if there exists at least one person who was born on day $d$. So,
$P/R = \{P_d \mid d \in D\}$.

\section{Problem 6}
$S = \{(s,t) \in T \times T \mid \text{the triangles s and t are similar}$

\begin{itemize}
\item Reflexive. Let $x$ be an arbitrary element in $T$. Since a
  triangle is similar with itself, $S$ is reflexive.
\item Symmetric. Let $x,y$ be an arbitrary element in $T$ such that
  $xSy$. We know that $x$ and $y$ are similar. So, $ySx$.
\item Transitive. Let $x,y$ and $z$ be arbitrary element in $T$ such
  that $xSy$ and $ySz$. We know that $x$ and $y$ are similar to each
  other. Similarly $y$ and $z$ are similar to each other. So. $x$ and
  $z$ are similar to each other. So, $xSz$. Since $x,y$ and $z$ are
  arbitrary we can conclude that $S$ is transitive.
\end{itemize}

\section{Problem 7}
\begin{itemize}
\item Symmetric. Let $x,y$ be arbitrary elements in $A$ such that
  $xRy$. So $(x,y) \in R$. Since $R = \cup_{X \in F}(X \times X)$, it
  follows that for some $X \in F$, $(x,y) \in X \times X$. So, $x \in
  X$ and $y \in X$. But then $(y,x) \in X \times X$. So, $(y,x) \in
  \cup_{X \in F}(X \times X)$. Since $x$ and $y$ are arbitrary, it
  follows that $R$ is symmetric.
\item Transitive. Let $x,y$ and $z$ be arbitrary elements in $A$ such
  that $xRy$ and $yRz$. Since $R = \cup_{X \in F}(X \times X)$, it
  follows that for some $X \in F$, $(x,y) \in X \times X$. Similarly
  for some $Y \in F$, $(y,z) \in Y \times Y$. So, $y \in X$ and $y \in
  Y$. But from the definition of partition, we know that $F$ is
  pairwise disjoint. So, $X=Y$ which leads us to $(x,z) \in \cup_{X
    \in F}(X \times X)$. Since $x,y$ and $z$ are arbitrary, it follows
  that $R$ is transitive.
\end{itemize}

\section{Problem 8}
\begin{itemize}
\item $(\Rightarrow)$ Let $(x,y)$ be an arbitrary element in $R$ such
  that $x \in A$ and $y \in A$. From $xRy$, it follows that they are
  part of the same equivalence class, so $x \in [x]_R$ and
  $y \in [y]_R$. From $A/R = A/S$, it follows that $[x]_R \in A/R$ and
  $[y]_R \in A/S$. Since they are from the same equivalence class,
  $(x,y) \in S$. Since $x$ and $y$ are arbitrary, it follows that
  $R \subseteq S$.
\item $(\Leftarrow)$. Follows the same reasoning in other direction.
\end{itemize}

\section{Problem 9}
\begin{itemize}
\item $(\Rightarrow)$ Let $(x,y)$ be an arbitrary element in $S$ such
  that $x \in A$ and $y \in B$. From $S=\cup_{X \in F}(X \times X)$,
  it follows that there is some element $X in F$ such that $x \in X$
  and $y \in X$. Now $F = A/R$, so $X$ is an equivalence class of $R$
  on $A$. So, $X = [x]_R = [y]_R$. Hence $(x,y) \in R$, so $S
  \subseteq R$.
\item $(\Leftarrow)$. Follow the same reasoning in other direction
\end{itemize}

\section{Problem 10} 
$C_m = \{(x,y) \in \mathbb{Z} \times \mathbb{Z} \mid x \equiv y (\text{ mod m})\}$
\subsection{Solution (a)}
Reflexive \\ \\
Let $x$ be an arbitrary element in $Z$. $(x,x) \in C_m$ for $k = 0$.
Since $x$ was arbitrary, it follows that $C_m$ is reflexive.

\noindent
Symmetric \\ \\
Let $x,y$ be an arbitrary element in $Z$ such that $(x,y) \in C_m$. It
follows that $x \equiv y (\text{ mod m })$ or $\exists k \in Z (x - y
= km)$. Since $-k \in Z$, it follows that $y - x = (-k)m$. From the
definition of $C_m$, it follows that $(y,x) \in C_m$. Since $x$ and
$y$ was arbitrary, it follows that $C_m$ is symmetric.

\subsection{Solution (b)}
$C_2 = \{(x,y) \in \mathbb{Z} \times \mathbb{Z} \mid \abs{x-y} \text{ is even}\}$
$C_3 = \{(x,y) \in \mathbb{Z} \times \mathbb{Z} \mid \abs{x-y} \text{
  is divisible by 3}\}$

$C_2$ has two equivalence class and $C_2$ has three equivalence
classes. $C_m$ is likely to have $m$ equivalence classes.

\section{Problem 11}
Let $n$ be an arbitrary integer. We know that every integer is either
even or odd.
\\
Case 1. $n$ is even. Suppose $k = \frac{n^2}{4}$. $k$ is an integer
since $n$ is even and the smallest non-zero even number squared by
itself is divisible by 4. It follows that $n^2 = k4$. So, $n^2 \equiv
0 (\text{ mod 4})$.
\\
Case 2. $n$ is odd. Suppose $k = \frac{n^2 - 1}{4}$. $k$ is an integer
since $n^2 - 1$ is an even number having a factor of $4$. It follows
that $n^2 - 1 = 4k$. So, $n^2 = 1 (\text{ mod 4})$

\section{Problem 12}
Suppose $a \equiv c (\text{ mod m})$ and $b \equiv d (\text{ mod m})$. Then it follows
that $a - c = k_1m$ and $b - d = k_2m$ for some integer $k_1$ and
$k_2$. Summing them up we get, $(a + b) - (c + d) = m(k_1 + k_2)$.
Since $k_1 + k_2$ is an integer, from the definition of congruent it
follows that $(a + b) \equiv c + d (\text{ mod m})$. \\ \\
Proof of $ab \equiv cd (\text{ mod m})$ \\
We know that $ a = c + mk_1$. It follows that $ab = bc + bmk_1$.
Similarly we know that $b = d + mk_2$. Then it follows that $cd = cb -
cmk_2$. Subtracting them we get, $ab - cd = m(bk_1 + ck_2)$. Since
$bk_1 + ck_2$ is an integer, from the definition of congruent it
follows that $ab \equiv cd (\text{ mod m})$

\section{Problem 13}
\subsection{Solution (a)}
Proof of Reflexive \\ 
Let $b$ be an arbitrary element on $B$. Since $B \subseteq A$, it
follows that $b \in A$. Since $R$ is reflexive on $A$, it follows that
$(b,b) \in R$. From the definition of cartesian product, it follows
that $(b,b) \in B \times B$. So, $(b,b) \in R \cap (B \times B)$.
Since $b$ was an arbitrary element of $B$, it follows that
$R \cap (B \times B)$ is reflexive. \\ \\
Proof of Symmetric \\ 
Let $a,b$ be arbitrary element on $B$ such that
$(a,b) \in R \cap (B \times B)$. Since $R$ is symmetric, it follows
that $(b,a) \in R$. From the definition of cartesian product, it
follows that $(b,a) \in B \times B$. So,
$(b,a) \in R \cap (B \times B)$. Since $a$ and $b$ are an arbitrary
element of $B$, it follows that $R \cap (B \times B)$ is symmetric. \\\\
Proof of Transitive \\ 
Let $a,b$ and $c$ be arbitrary element in $B$ such that $(a,b) \in R
\cap (B \times B)$ and $(b,c) \in R \cap (B \times B)$. Since $R$ is
transitive it follows that $(a,c) \in R$. From the definition of
cartesian product, it follows that $(a,c) \in B \times B$. So, $(a,c)
\in R \cap (B \times B)$. Since $a,b$ and $c$ are arbitrary we can
conclude that $R \cap (B \times B)$ is transitive.

\subsection{Solution (b)}
Let x be an arbitrary element in $B$. \\
($\Rightarrow$) Suppose $y \in [x]_S$. From the definition of
equivalence class, it follows that $ySx$. We know that $S = R \cap (B
\times B)$, so $(y,x) \in R \cap (B \times B)$. From $(y,x) \in B
\times B$, it follows that $y \in B$. Similarly from $yRx$ and from
the definition of equivalence class, it follows that $y \in [x]_R$.
So, $y \in [x]_R \cap B$. \\

\noindent
($\Leftarrow$) Suppose $y \in [x]_R \cap B$. From the definition of
equivalence class, it follows that $yRx$. Also, we know that $y \in B$
and $x \in B$. So, $(y,x) \in B \times B$. From this we can conclude
that $(y,x) \in R \cap (B \times B)$ or $(y,x) \in S$. Using the
definition of equivalence class, we get $y \in [x]_S$.
\\
Since $y$ was arbitrary it follows that $[x]_S = [x]_R \cap B$.

\section{Problem 14}
\subsection{Solution (a)}
Proof for reflexive

Let $X$ be an arbitrary element on $\mathbb{P}(A)$. We know that $X
\Delta X = \emptyset$. And since null set is part of all set, $X \Delta
X \subseteq B$. From the definition of $R$, it follows that $(X,X) \in
R$. Since $X$ is arbitrary it follows that $R$ is reflexive.

Proof for reflexive

Let $X,Y$ be an arbitrary element on $\mathbb{P}(A)$ such that $(X,Y)
\in R$. Since $(X,Y) \in R$, it follows that $X \Delta Y \subseteq B$.
So, $(X \setminus Y) \cup (Y \setminus X) \subseteq B$. This can be
also written as $(Y \setminus X) \subseteq B \cup (X \setminus Y)$.
So, $Y \Delta X \subseteq B$. From the definition of $R$, it follows
that $(Y,X) \in R$. Since $X$ and $Y$ are arbitrary it follows that
$R$ is symmetric.

Proof for transitive

Let $X,Y$ and $Z$ be arbitrary element of $\mathbb{P}(A)$ such that
$(X,Y) \in R$ and $(Y,Z) \in R$. From $(X,Y) \in R$, it follows that
$(X \setminus Y) \cup (Y \setminus X) \subseteq B$. Similarly, $(Y
\setminus Z) \cup (Z \setminus Y) \subseteq B$. Let $a$ be an
arbitrary element in $(X \setminus Z) \cup (Z \setminus X)$.
Case 1. $a \in X \setminus Z$
It follows that $a \in X$ and $a \notin Z$.Combining union, we get $(X
\setminus Y) \cup (Y \setminus X) \cup (Y \setminus Z) \cup (Z
\setminus Y) \subseteq B$. Now we know $a \in X$, so $a \notin (Y
\setminus X)$. Similarly from $a \notin Z$, it follows that $a \notin
(Z \setminus Y)$. Simplifying it we get, $(X \setminus Y) \cup (Y
\setminus Z) \subseteq B$. From the law of excluded middle, $a \in Y$
or $a \notin Y$. If $a \in Y$, then $a \in Y \setminus Z$ so $a \in
B$. If $a \notin Y$, then $a \in X \setminus Y$, so $a \in R$.
Case 2. $a \in Z \setminus X$
It follows that $a \in Z$ and $a \notin X$. Combining union, we get $(X
\setminus Y) \cup (Y \setminus X) \cup (Y \setminus Z) \cup (Z
\setminus Y) \subseteq B$. Now we know $a \in Z$, so $a \notin Y
\setminus Z$. Also from $a \notin X$, we know that $a \notin X
\setminus Y$. Simplifying it, we get $(Y \setminus X) \cup (Z
\setminus Y) \subseteq R$. From the law of excluded middle, $a \in Y$
or $a \notin Y$. If $a \in Y$, then $a \in Y /setminus X$ so $a \in
B$. If $a \notin Y$, then $a \in Z / setminus Y$, so $a \in R$.

\section{Solution 15}
Suppose $F$ is a partion of $A$ and $G$ is a partition of $B$ and $A$
and $B$ are disjoint.
\subsection{Proof of $\cup(F \cup G) = A \cup B$}
We know that $\cup F = A$ and $\cup G = B$. Taking union of both,
$(\cup F) \cup (\cup G) = A \cup B$. From chapter 3, exercise 15 we
know that $(\cup F) \cup (\cup G) = \cup(F \cup G)$. Hence we can
conclude that $\cup(F \cup G) = A \cup B$.

\subsection{Proof of $F \cup G$ is pairwise disjoint}
Let $X$ and $Y$ be an arbitrary element in $F \cup G$. Suppose $x \neq
Y$.
\\ Case 1: $X \in F \land Y \in F$. Since $F$ is a partition of $A$, we
know that $F$ is pairwise disjoint. So, $X \cap Y = \emptyset$.
\\ Case 2: $X \in F \land Y \in G$. Since $\cup F = A$, it follows that
$X \subseteq A$. Similarly, $Y \subseteq B$. Since $A \cap B =
\emptyset$, we can conclude that $X \cap Y = \emptyset$. (Prove that
lemma if you are not yet convinced. (Hint: Use contradiction in that proof)).
\\ Case 3: $X \in G \land Y \in G$ Similar to case 1.
\\ Case 4: $X \in G \land Y \in F$ Similar to case 2.

\subsection{$\forall X \in (F \cup G)(X \neq \emptyset)$}
Let $X$ be an arbitrary element in $F \cup G$. Let us consider the
cases:
\\ Case 1: $X \in F$. Since $F$ is partition of $A$, it follows that $x
\neq \emptyset$.
\\ Case 2: $X \in G$. Similar as case 1.

\section{Solution 16}
Suppose $R$ is an equivalence relation on $A$. $S$ is an equivalence
relation on $B$ and $A \cap B = \emptyset$.
\subsection{Solution (a)}

Proof for Reflexive

Let $x$ be an arbitrary element on $A \cup B$. Let us consider the
cases:
\\ Case 1. $x \in A$. Since $R$ is an equivalence relation on $A$, it
follows that $(x,x) \in R$. So, $(x,x) \in R \cup S$.
\\ Case 2. $x \in B$. Since $S$ is an equivalence relation on $B$, it
follows that $(x,x) \in S$. So, $(x,x) \in R \cup S$.

\noindent
Proof of Symmetric

Let $x,y$ be an arbitrary element in $A \cup B$ such that $(x,y) \in R
\cup S$. Let us consider the cases:
\\ Case 1. $(x,y) \in R$. Since $R$ is symmetric it follows that
$(y,x) \in R$. So, $(y,x) \in R \cup S$.
\\ Case 2. $(x,y) \in S$. Similar to case 1.

\noindent
Proof of Transitive

Let $x,y$ and $z$ be an arbitrary element in $A \cup B$ such that
$(x,y) \in R \cup S$ and $(y,z) \in R \cup S$. Let us consider the
cases:
\\ Case 1. $(x,y) \in R \land (y,z) \in R$. Since $R$ is transitive,
it follows that $(x,z) \in R$. So, $(x,z) \in R \cup S$.
\\ Case 2. $(x,y) \in R \land (y,z) \in S$. Now we know that $R$ is an
equivalence relation on $A$. So, $x \in A$ and $y \in A$. Similarly $y
\in B$ and $z \in B$. So, $y \in A \cap B$. But we know that $A \cap
B$ is disjoint. So this case is not possible.
\\ Case 3. $(x,y) \in S \land (y,z) \in S$ Similar to Case 1.
\\ Case 4. $(x,y) \in S \land (y,z) \in R$ Similar to Case 2.

\subsection{Solution (b)}
Proof of $\forall x \in A ([x]_{R \cup S} = [x]_R)$

Let $x$ be an arbitrary element in $A$
\\ $(Rightarrow)$ Suppose $y \in [x]_{R \cup S}$. From the definition
of equivalence class it follows that $(y,x) \in R \cup S$. Let us
consider the case:
\\ Case 1. $(y,x) \in R$ Since $(y,x) \in R$, then from the definition of
equivalence classes it follows that $y \in [x]_R$.
\\ Case 2. $(y,x) \in S$. We know that $x \in A$. Since $S$ is an
equivalence relation on $B$, it follows that $y \in B$ and $x \in B$.
So, $x \in A \cap B$. But we know that $A$ and $B$ are disjoint. So
this case is not possible.

$(Leftarrow)$ Suppose $y \in [x]_R$. From the definition of
equivalence classes it follows that $(y,x) \in R$. So, $(y,x) \in R
\cup S$. Now, we know that $R \cup S$ is an equivalence relation on $A
\cup B$. So, $y \in [x]_{R \cup S}$.

\subsection{Solution (c)}
$(\Rightarrow)$ Let $x$ be an arbitrary element on $(A \cup B)
/ (R \cup S)$. It follows that $x \in A \cup B$ and $x \notin R \cup
S$. Let us consider the cases:
\\ Case 1. $x \in A \land x \notin R$. Then it follows that $x \in
A \setminus R$. So, $x \in (A \setminus R) \cup (B \setminus S)$.
\\ Case 2. $x \in A \land x \notin S$. $R$ is a relation on $A$. We
know that $x \in A$. From the definition of relation, it follows that
$x \notin R$ since $x$ is not an ordered pair. So, $x \in (A \setminus
R) \cup (B \setminus S)$.
\\ Case 3. $x \in B \land x \notin R$. Same as case 2.
\\ Case 4. $x \in B \land x \notin S$. Same as case 1.

$(\Leftarrow)$ Let $x$ be an arbitrary element in $(A \setminus R)
\cup (B \setminus S)$. Let us consider the cases:
\\ Case 1. $x \in A \setminus R$. It follows that $x \in A$ and $x
\notin R$. Since $x \in A$, it follows that $x \in A \cup B$. Since $x
\notin R$, it follows that $x \notin (R \cup S)$. So $x \in (A \cup B)
\setminus (R \cup S)$.
\\ Case 2. $x \in B \setminus S$. It follows that $x \in B$ and $x
\notin S$. Since $x \in B$, it follows that $x \in A \cup B$. Since $x
\notin S$, it follows that $x \notin (R \cup S)$. So, $x \in (A \cup
B) \setminus (R \cup S)$.

\section{Solution 17}
\subsection{Proof of $\cup (F.G) = A$}
$(\Rightarrow)$ Let $x$ be an arbitrary element of $\cup (F.G)$. Then
it follows that there is some $Z \in P(A)$ such that $Z \neq
\emptyset$, $\exists X \in F \exists Y \in G(Z = X \cap Y)$ and $x
\in Z$. Since $Z \in P(A)$, it follows that $Z \subseteq A$. Since $x
\in Z$, it follows that $x \in A$.

$(\Leftarrow)$ Let $x$ be an arbitrary element of $A$. Since $F$ is a
partition of $A$ it follows that that $x \in [x]$. Similarly for $G$,
$x \in [x]$. So there is some $X \in F$ and $Y \in G$ such that
$x \in (X \cap Y)$. From the definition of $F.G$ it follows that there
is some $Z \in F.G$ such that $Z = X \cap Y$ and $Z = \emptyset$ since
$x \in Z$. So $x \in \cup(F.G)$.

\subsection{Proof of F.G is pairwise disjoint}
Let $A,B$ be an arbitrary element of $F.G$ such that $A \neq B$. Let
us prove by contradiction. Suppose $a \in A \cap B$. It follows that
$a \in A$ and $a \in B$. From $A \in F.G$, it follows that there is
some $X_1 \in F$ and $Y_1 \in G$ such that $A = X_1 \cap Y_1$. From $a
\in A$, it follows that $a \in X_1$ and $a \in Y_1$. Now from $B \in
F.G$, it follows that there is some $X_2 \in F$ and $Y_2 \in G$ such
that $B = X_2 \cap Y_2$. We know that $a \in X_1$ and $a \in X_2$, so
$[a] = X_1$ and $[a] = X_2$. So, $X_1 = X_2$. Similarly, $Y_1 = Y_2$.
So $A = B$. But this contradicts the fact that $A \neq B$. So, $a
\notin A \cap B$. Since a was an arbitrary element, it follows that $A
\cap B = \emptyset$.

\subsection{Proof of $\forall X \in F.G(X \neq \emptyset)$}
Suppose $X \in F.G$. Then from the definition of $F.G$, it follows
that $X \neq \emptyset$.

\section{Solution 18}
Let us compare each elements:
\begin{align*}
  \mathbb{R}^{-} \cap \mathbb{Z} = \mathbb{Z}^{-1} \\
  \mathbb{R}^{-} \cap \mathbb{R} \setminus \mathbb{Z} = \mathbb{R}^{-1} \setminus \mathbb{Z} \\
  \mathbb{R}^{+} \cap \mathbb{Z} = \mathbb{Z}^{+} \\
  \mathbb{R}^{+} \cap \mathbb{R} \setminus \mathbb{Z} = \mathbb{R}^{+} \setminus \mathbb{Z} \\
  \{0\} \cap \mathbb{Z} = \{0\} \\
  \{0\} \cap \mathbb{R} \setminus \mathbb{Z} = \emptyset 
\end{align*}

The elements are: $\{\mathbb{Z}^{-1}, \mathbb{R}^{-1} \setminus
\mathbb{Z}, \mathbb{Z}^{+}, \mathbb{R}^{+}
\setminus \mathbb{Z}, \{0\}\}$

\section{Solution 19}
\subsection{Solution (a)}
Proof of Reflexive

Let $x$ be an arbitrary elements in $A$. Since $R$ is equivalence
relation on $A$, it follows that $(x,x) \in R$. Similarly, $(x,x) \in
R \cap S$. Since $x$ is an arbitrary elements, we can conclude that
$T$ is reflexive.

Proof of Symmetric

Let $x,y$ be an arbitrary element on $A$ such that $(x,y) \in T$. It
follows that $(x,y) \in R$ and $(x,y) \in S$. Since $R$ is symmetric,
it follows that $(y,x) \in R$. Similarly $(y,x) \in S$. So, $(y,x)\in
R \cap S$. Since $x$ and $y$ are arbitrary , we can conclude that $T$
is symmetric.

Proof of Transitive

Let $x,y$ and $z$ be arbitrary element on $A$ such that $(x,y)\in T$
and $(y,z) \in T$. It follows that $(x,y) \in R$ and $(y,z) \in R$.
Since $R$ is transitive, it follows that $(x,z) \in R$. For similar
reasons, $(x,z) \in S$. So, $(x,z) \in R \cap S$. Since $x$ and $z$
are arbitrary, it follows that $T$ is transitive.

\subsection{Solution (b)}
Suppose $x \in A$
$(\Rightarrow)$ Suppose $y \in [x]_T$. Then it follows that $yTx$. We
know that $T = R \cap S$. So, $(y,x) \in R \land (y,x) \in S$. From
the definition of equivalence classes, it follows that $y \in [x]_R
\land y \in [x]_S$.

$(\Leftarrow)$ Suppose $y \in [x]_R \cap [x]_S$. Then it follows that
$y \in [x]_R \cap y \in [x]_S$. So, $(y,x) \in R \land (y,x) \in S$.
Combining them, $(x,y) \in R \cap S$. So, $y \in [x]_{R \cap S}$.

\subsection{Solution (c)}
$(\Rightarrow)$ Suppose $X \in A/T$. Since $A/T$ is a partition it
follows that $X \neq \emptyset$. Now for some $x \in A$, it follows
that $X = [x]_T = [x]_R \cap [x]_S$. Since $[x]_R \in A \setminus R$
and $[x]_S \in A \setminus S$, $X \in (A/R).(A/S)$.

\section{Solution 20}
\subsection{Proof of $\cup(F \otimes G) = A \times B$}

($\Rightarrow$) Let $(x,y)$ be arbitrary element of
$\cup(F \otimes G)$. Then it follows that there is some
$Z \in F \otimes G$ such that $(x,y) \in Z$,
$\exists Y \in F \exists Y \in G (Z = X \times Y)$. So
$(x,y) \in X \times Y$. Since $x \in X$ and $X \in F$, $x \in \cup F$.
Therefore $x \in A$. Similarly, $y \in B$. So, $(x,y) \in A \times B$.

$(\Leftarrow)$ Let $(x,y)$ be an arbitrary of $A \times B$. Since $F$
is a partition of $A$ and $x \in A$, it follows that there is some $X
\in F$ such that $x \in X$. Similarly there is some $Y \in G$ such
that $y \in Y$. So, $(x,y) \in X \times Y$. It follows that there is
some $Z \in F \otimes G$ such that $(x,y) \in Z$. Hence, $(x,y) \in
\cup(F \otimes G)$.

\subsection{Proof of $F \otimes G$ is pairwise disjoint}
Let $X,Y$ be arbitrary element in $F \otimes G$ such that $X \neq Y$.
Let us prove by contradiction. Suppose there exists some element $x,y$
such that $(x,y) \in X \cap Y$. Since $X \in F \otimes G$, it follows
that there is some $X_1 \in F$ and $Y_1 \in G$ such that $X = X_1
\times Y_1$. Similarly for some $X_2 \in F$ and $Y_2 \in G$ it follows
that $Y = X_1 \times Y_2$. Now, $(x,y) \in (X_1 \times Y_1) \cap (X_2
\times Y_2)$. Since $F$ is a partition on $A$, $x \in A$, $x \in X_1$
and $x \in X_2$, it follows that $X_1 = X_2$ from the definition of
equivalence class. Similarly $Y_1 = Y_2$. Since $X = X_1 \times Y_1$
and $Y = X_2 \times Y_2$, it follows that $X = Y$. But this
contradicts the fact that $X \neq Y$. So, $X \cap Y = \emptyset$.

\subsection{Proof of $\forall X \in F \otimes G(X \neq \emptyset)$}
Suppose $X \in F \otimes G$. Then there exists some $X_1 \in F$ and
$Y_1 \in G$ such that $X = X_1 \times Y_1$. Since $F$ is a partition
on $A$, $X_1 \neq \emptyset$. Similarly $Y_1 \neq \emptyset$.
Therefore $X_1 \times Y_1 \neq \emptyset$. $So X \neq \emptyset$.

\section{Solution 21}
\begin{align*}
  F \otimes F = \{Z \in \mathbb(R \times R) \mid \exists X \in F
  \exists Y \in F(Z = X \times Y)\} \\
  F \otimes F = \{\mathbb{R}^- \times \mathbb{R}^+, \mathbb{R}^+
  \times \mathbb{R}^-, \mathbb{R}^- \times \{0\}, \mathbb{R}^{+}
  \times \{0\}, \{0\} \times \mathbb{R}^-, \{0\} \times \mathbb{R}^+,
  \mathbb{R}^+ \times \mathbb{R}^+, \mathbb{R}^- \times \mathbb{R}^-,
  \{0\} \times \{0\} \}
\end{align*}

For the next part of the answer, just try to visualize the graph in a
plane and you will figure that out!

\section{Solution 22}
\subsection{Solution (a)}
Proof of $T$ is reflexive

Suppose $(a,b) \in A \times B$ for arbitrary $a \in A$ and $b \in B$.
It follows that $((a,b), (a,b)) \in (A \times B) \times (A \times B)$.
Since $a$ and $b$ are arbitrary it follows that $T$ is reflexive.

\noindent
Proof of $T$ is symmetric

Suppose $((a,b), (a',b')) \in T$. We know that $(a',b') \in A \times
B$ and $(a,b) \in A \times B$. Also, $aRa'$ and $bRb'$. Since $R$ is
symmetric it follows that $a'Ra$. Similarly $b'Rb$. So,
$((a',b'),(a,b)) \in T$. Since $a,b,a'$ and $b'$ are arbitrary it
follows that $T$is symmetric.

\noindent
Proof of $T$ is transitive Suppose $((a,b),(a',b')) \in T$ and
$((a',b'),(c,d)) \in T$. It follows that $(a,a') \in R$,
$(b,b') \in S$, $(a',c) \in R$ and $(b',d) \in S$. From $(a,a') \in R$
and $(a',c) \in R$, it follows that $(a,c) \in R$. Similarly,
$(b,d) \in S$. So, $((a,b),(c,d)) \in T$. Since $a,a',b',c$ and $d$
are arbitrary it follows that $T$ is transitive.

\subsection{Solution (b)}
Suppose $a \in A$ and $b \in B$. Suppose $a' \in A$ and $b' \in B$.
$(\Rightarrow)$ Suppose $(a',b') \in [(a,b)]_T$. It follows that
$((a',b'),(a,b)) \in T$. So, $(a',a) \in R$ and $(b',b) \in S$. It
follows that $a' \in [a]_R$ and $b' \in [b]_S$. Therefore $(a',b') \in
[a]_R \times [b]_R$. Since $a'$ and $b'$ are arbitrary it follows that
$[(a,b)]_T \subseteq [a]_R \times [b]_S$.

$(\Leftarrow)$ Suppose $(a',b') \in [a]_R \times [b]_R$. Then it
follows that $a' \in [a]_R$ and $b' \in [b]_S$. So, $(a',a) \in R$ and
$(b',b) \in S$. From the definition of $T$, it follows that
$((a',b'),(a,b)) \in T$. So, $(a',b') \in [(a,b)]_T$. Since $a'$ and
$b'$ are arbitrary it follows that
$[a]_R \times [b]_R \subseteq [(a,b)]$.

\subsection{Solution (c)}
$(\Rightarrow)$ Suppose $X \in (A \times B)/T$. Since $(A \times B)/T$
is a partition it follows that $X \neq \emptyset$. There is some
$(a,b) \in X$. Since $X \in (A \times B)/T$ and $\cup (A \times B)/T =
A \times B$, it follows that $(a,b) \in A \times B$. So, $a \in A$ and
$b \in B$. Since $R$ is a equivalence relation on $A$, $a \in [a]_R$.
Similarly, $b \in [b]_S$. So, $(a,b) \in [a]_R \times [b]_R$. From the
definition of $\otimes$, it follows that $[a]_R \times [b]_S \in (A/R)
\otimes (B/S)$. From (b), it follows that $[(a,b)]_T \in (A/R) \otimes
(B/S)$. So, $X \in (A/R) \otimes (B/R)$.

$(\Leftarrow)$ Suppose $X \in (A/R) \otimes (B/S)$. It follows that
there is some $X_1 \in (A/R)$ and $Y_1 \in (B/S)$ such that $X=X_1
\times Y_1$. Both $X_1$ and $Y_1$ cannot be $\emptyset$. So, suppose
$x \in X_1$ and $y \in Y_1$. Since $X_1 \in F$ and $\cup F = A$, it
follows that $x \in A$. Similarly $y \in B$. So, $(x,y) \in A \times
B$. It follows that $[(x,y)]_T \in (A \times B)/T$. From (b), $[x]_R
\times [y]_S \subseteq (A \times B)/T$. So, $X_1 \times Y_1 \in (A
\times B)/T$. So, $X_1 \times Y_1 \in (A \times B)/T$. Hence $X \in (A
\times B)/T$.

\section{Solution 23}
\subsection{Solution (a)}
Let $T = \{(X,Y) \in A/S \times A/S \mid \forall x \in X \forall y \in Y(xRy)\}$
\begin{align*}
  [x]_ST[y]_s \\
  \iff \forall a \in [x]_S \forall b \in [y]_S (aRb) \\
  \iff \forall a \in A \forall b \in A (a \in [x]_S \land b \in [y]_S \land aRb) \\
  \iff \forall a \in A \forall b \in A (aSx \land bSy \land aRb) \\
  \iff xRy          \text{(Since R is compatible with S)}
\end{align*}

\subsection{Solution (b)}
Let us formulate the set $T$:

$T = \{([x]_s, [y]_s) \in A/S \times A/S \mid \forall x \in A \forall
y \in A(xRy)\}$ \\
$T = \{(X,Y) \in A/S \times A/S \mid \forall x \in A \forall y \in
A(xRy \land x \in X \land y \in Y)\}$.

Suppose $x,y,x'$ and $y' \in A$. Suppose $xSx'$ and $ySy'$. It follows
that $[x]_S = [x']_S$ and $[y]_S = [y']_S$. So, $xRy$ iff
$[x]_ST[y]_S$ iff $[x']_ST[y']_S$ iff $x'Ry'$.

\section{Chapter 24}
\subsection{Solution (a)}
Proof of Reflexive \\

Let $a$ be an arbitrary element in $A$. Since $R$ and $R^{-1}$ are
reflexive on $A$, it follows that $(a,a) \in R \cap R^{-1}$. Since $a$
is arbitrary, we can conclude that $S$ is reflexive.

Proof of Symmetric \\

Let $a,b$ be an arbitrary element in $A$ such that $(a,b) \in S$. It
follows that $(a,b) \in R \land (a,b) \in R^{-1}$. Since $(a,b) \in
R^{-1}$, it follows that $(b,a) \in R$. Similarly, $(b,a) \in R^{-1}$.
So, $(b,a) \in R \cap R^{-1}$. Since $a$ and $b$ are arbitrary, we can
conclude that $S$ is symmetric.

Proof of Transitive \\

Let $a,b$ and $c$ be an arbitrary element on $A$ such that $(a,b) \in
S$ and $(b,c) \in S$. From $(a,b) \in S$, it follows that $(a,b) \in R
\land (a,b) \in R^{-1}$. Similarly, $(b,c) \in S$ and $(b,c) \in
R^{-1}$. Since $R$ and $R^{-1}$ are transitive, it follows that $(a,c)
\in R \cap R^{-1}$. Since $a,b$ and $c$ are arbitrary, $S$ is
transitive.

\subsection{Solution (b)}
From 23 (a), it follows that if $R$ is compatible with $S$, then we
can prove that there is a unique relation $T$. Suppose $x,y,x'$ and
$y'$ in A. Suppose $xSx'$ and $ySy'$.

$(\Rightarrow)$ Suppose $xRy$. From $xSx'$, it follows that $xRx'
\land xR^{-1}x'$. Similarly, $yRy' \land yR^{-1}y'$. From $xR^{-1}x'$,
it follows that $x'Rx$. From $xRy$ and $yRy'$, it follows that $xRy'$.
From $x'Rx$ and $xRy'$, it follows that $x'Ry'$.

$(\Leftarrow)$ Suppose $x'Ry'$. We know that $yR^{-1}y'$. So, $y'Ry$.
From transitive property, it follows that $x'Ry$. We also know,
$xRx'$. Again from the transitive property it follows that $xRy$.

\subsection{Solution (c)}

$T = \{(X,Y) \in A/S \times A/S \mid \exists x \in X \exists y \in Y (xRy)\}$

Reflexive \\
Let $X$ be an arbitrary element on $A/S$. Then it follows that $X \neq
\emptyset$. Since $A/S$ is a partition. So there is some $x$ such that
$x \in X$. Now since $X \subseteq A$, it follows that $x \in A$. Since
$R$is reflexive on $A$, it follows that $xRx$. From the definition of
$T$, it follows that $(X,X) \in T$. Since $X$ is arbitrary, $T$ is
reflexive.

Transitive \\
Let $x,y,z$ be an arbitrary element on $A$ such that $[x]_ST[y]_S$ and
$[y]_ST[z]_S$. It follows that $xRy$ and $yRz$. Since $R$ is
transitive, it follows that $xRz$. So, $[x]_ST[y]_S$. Since $x,y$ and
$z$ are arbitrary it follows that $T$ is transitive.

Anti symmetric \\
Let $x$ and $y$ be arbitrary element on $A$ such that $[x]_ST[y]_S$
and $[y]_ST[x]_S$. So $xRy$ and $yRx$. Then $xRy$ and $xR^{-1}y$. So
$xSy$. So, $[x]_S = [y]_S$. Hence $T$ is anti symmetric.

\section{Solution 25}
\subsection{Solution (a)}
Reflexive \\
Let $X$ be an arbitrary element in $A$. It follows that $(X,Y) \in A
\times A$ since both $X$ and $Y$ has same number of elements. Since
$X$ is arbitrary, we can conclude that $R$ is reflexive.

Transitive \\
Let $X,Y$ and $Z$ be an arbitrary elements in $A$ such that $(X,Y) \in
R$ and $(Y,Z) \in R$. It follows that $Y$ has at least as many
elements as $X$. Similarly, $Z$ has at least as many elements as $Y$.
So, $(X,Z) \in R$ since $Z$ has at least as many elements as $X$.
Hence $R$ is transitive.

\subsection{Solution (b)}
$S = R \cap R^{-1} = \{(X,Y) \in A \times A \mid \text{ X and Y has same
  number of elements}\}$ \\
$A/S = \{\{\emptyset\}, \{set with single elements\}, ... \}$ \\
Number of elements is 101.

\section{Solution 26}
\subsection{Solution (a)}
Reflexive \\

Lemma. A partition on $A$ refines itself.
Suppose $F$ be a partition on $A$. Let $X$ be an arbitrary elements in
$F$. Suppose $Y \in F$ such that $Y = X$ then it follows that $X
\subseteq Y$. So, $F$ refines itself.

Let $X$ be an arbitrary element in $P$. Now we know that $X$ refines
$X$. So, $(X,X) \in R$. Since $X$ is arbitrary, it follows that $R$ is
reflexive.

Transitive \\
Let $F,G,H$ be arbitrary element in $P$ such that $(F,G) \in R$ and
$(G,H) \in R$. Since $F$ refines $G$, it follows that $\forall X \in F
\exists Y \in G(X \subseteq Y)$. Similarly, $\forall A \in G \exists B
\in H(A \subseteq B)$. Let $X$ be an arbitrary element in $F$ and $Y$
be an arbitrary element in $G$, it follows that $X \subseteq Y$.
Universal instantiation gives us $Y \subseteq B$ for some $B \in H$.
From $X \subseteq Y$ and $Y \subseteq B$, it follows that $X \subseteq
B$. So, $F$ refines $H$. Since $F,G$ and $H$ are arbitrary, we can
conclude that $R$ is transitive.

\end{document}

