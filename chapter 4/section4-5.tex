%%%%%%%%%%%%%%%%%%%%%%%%%%%%%%%%%%%%%%%%%
% Author: Sibi <sibi@psibi.in>
%%%%%%%%%%%%%%%%%%%%%%%%%%%%%%%%%%%%%%%%%
\documentclass{article}
\usepackage{graphicx}
\usepackage{verbatim}
\usepackage{amsmath}
\usepackage{amsfonts}
\usepackage{amssymb}
\usepackage{tabularx}
\setlength\parskip{\baselineskip}
\begin{document}
\title{Chapter 4 (Section 4.3)}
\author{Sibi}
\date{\today}
\maketitle
\newpage

\section{Problem 1}
\subsection{Solution (a)}

Reflexive closure of $R = \{(b,b), (c,c), (a,a), (a,b), (b,c),
(c,b)\}$ \\
Symmetric closure of $R = (a,a), (a,b), (b,a), (b,c), (c,b)$ \\
Transitive closure of $R = (a,a), (a,b), (a,c), (a,b), (b,c), (c,b),
(b,b)$

\subsection{Solution (b)}

Reflexive closure of $R = \{(x,y \in R \times R \mid x < y \lor x =
y)\}$ \\
$= \{(x,y) \in R \times R \mid x \leq y\}$ \\ \\
Symmetric closure of $R = \{(x,y) \in R \times R \mid x < y \lor y <
x\}$ \\ \\
Transitive closure of $R = R$

\subsection{Solution (c)}

Reflexive and symmetric closure of $D_r = D_r$
Transitive closure of $D_r = \mathbb{R} \times \mathbb{R}$

\section{Problem 3}
\subsection{Solution (a)}

Suppose $R$ is asymmetric. Let $x$ and $y$ be an arbitrary element in
$A$ such that $xRy$ and $yRx$. Since $R$ is asymmetric, from $xRy$ it
follows that $(y,x) \notin R$. But this contradicts with out
assumption that $yRx$. So, it is vacuously true.

\subsection{Solution (b)}
Suppose $R$ is a strict partial order on $A$. Let $x$ and $y$ be
arbitrary element in $A$ such that $(x,y) \in R$. Let us try to prove
by contradiction. Suppose $(y,x) \in R$. Then from transitive property
of $R$ it follows that $(x,x) \in R$. But this contradicts the fact
that $R$ is irreflexive. So $(y,x) \notin R$.

\section{Problem 4}
\subsection{Solution (a)}
Suppose $R$ is a strict partial order on $A$. Let $S$ be the reflexive
closure on $R$. That means $S = R \cup i_A$.
\subsubsection{Proof of reflexive}
By the definition of reflexive closure, $S$ is reflexive.

\subsubsection{Proof of transitive}
Let $x,y$ and $z$ be arbitrary elements in $A$ such that $(x,y) \in S$
and $(y,z) \in S$. We have to prove that $(x,z) \in S$. Now we know
that $S = R \cup i_A$. Let us consider the cases separately:
Case 1. $(x,y) \in R$ Now this itself has two sub cases within it.
\begin{itemize}
\item $y,z \in R$ We know that $R$ is transitive, so $x,z \in R$.
  So $(x,z) \in R \cup i_A$.
\item $y,z \in i_A$ From the property of identity set it follows that
  $y = z$. Substituting it in $(x,y) \in R$, we get $(x,z) \in R$. So
  $(x,z) \in R \cup i_A$.
\end{itemize}
Case 2. $(x,y) \in i_A$ From the property of identity set it follows
that $x = y$. Substituting that in $(y,z) \in S$ gives $(x,z) \in S$.

\subsubsection{Proof of Anti-symmetric}
Let $x$ and $y$ be arbitrary elements in $A$ such that $(x,y) \in S$
and $(y,x) \in S$. We have to prove that $x = y$. Now we know
that $S = R \cup i_A$. Let us consider the cases separately:
Case 1. $(x,y) \in R$ Now this itself has two sub cases within it.
\begin{itemize}
\item $y,x in R$. Since $R$ is transitive, it follows that $(x,x) \in
  R$. But we know that $R$ is irreflexive, so this contradicts the
  fact that $(x,x) \notin R$. So $(y,x) \notin R$.
\item $y,x \in i_A$ From the property of identity set it follows that
  $y = x$. 
\end{itemize}
Case 2. $(x,y) \in i_A$. From the property of identity set, it follows
that $x = y$

\subsection{Solution (b)}
Suppose $R$ is total strict order. Let $x$ and $y$ be arbitrary
elements in $A$. We have to prove that $xSy \lor ySx$. We know that R
satisfies trichotomy, so $xRy \lor yRx \lor x = y$. Let us consider
the cases separately:
\begin{itemize}
\item Case 1. $xRy$ Since $S = R \cup i_A$, it follows that $(x,y) \in
  S$. So $xSy \lor ySx$.
\item Case 2. $yRx$ Similar to the case 1 proof.
\item Case 3. $x = y$ So, $(x,y) \in i_A$. It follows that $(x,y) \in
  R \cup i_A$. So $xSy \lor ySx$.
\end{itemize}

\section{Problem 5}
Suppose $R$ is a relation on $A$. Let $S = R \setminus i_A$.
\subsection{Solution (a)}
Let
$F = \{T \subset A \times A \mid T \subseteq R and T is
irreflexive\}$. We have to prove things here.
\subsubsection{Proof $S \in F$}
Let $x,y$ be arbitrary elements in $A$ such that $(x,y) \in S$. From
$S = R \setminus i_A$, it follows that $(x,y) \in R$. So, $S \subseteq
R$. It also follows that $(x,y) \notin i_A$. So $\forall x \forall y
(x,y) \notin R$. That can be re-stated as $\forall x (x,x) \notin R$.
So $R$ is irreflexive. So, we can conclude that $S \in F$.
\subsubsection{S is the largest element in F}
Let $T$ be an arbitrary element in $F$. We have to prove that $T
\subseteq S$. Let $(x,y)$ be arbitrary element in $T$. Since $T
\subseteq R$, it follows that $(x,y) \in R$. Also since $T$ is
irreflexive, it follows that $x \neq y$. So $(x,y) \notin i_A$. Hence
$(x,y) \in R \setminus i_A$. So, $T \subseteq S$.

\subsection{Solution (b)}
Suppose $R$ is a partial order on $A$.
\subsubsection{Proof that S is irreflexive}
Let $x$ be an arbitrary element in $A$. Since $R$ is reflexive it
follows that $(x,x) \in R$. From the identity property, it follows
that $(x,x) \in i_A$. So, it can be concluded that $(x,x) \notin R
\setminus i_A$. Therefore $S$ is irreflexive.
\subsubsection{Proof that S is transitive}
Let $x,y$ and $z$ be arbitrary elements in $A$ such that $(x,y) \in S$
and $(y,z) \in S$. We have to prove that $xSz$. We know that
$S = R \setminus i_A$. Then we have $(x,y) \in R \land (x,y) \in i_A$
and $(y,z) \in R \land (y,z) \in i_A$. From
$(x,y) \in R \land (y,z) \in R$, it follows that $(x,z) \in R$ since
$R$ is transitive. Now there can be two cases here. If
$(x,z) \notin i_A$ then $xSz$. But if $(x,z) \in i_A$, then $x = z$.
From $xRy$ and $yRx$, it follows that $x = y$ since
$R is antisymmetric$. But we know that $(x,y) \in S$ and so this
contradicts with the fact that $(x,y) \notin i_A$. Therefore $x \neq
z$. Hence $xSz$.

\section{Problem 6}
\subsection{Solution (a)}
$S = \{(p,q) \in P \times P \mid q is the descendant of P \}$
\subsection{Solution (b)}
$S^{-1} = \{(a,b) \in P \times P \mid a is the descendant of b\}$
$S \circ S^{-1} = \{(p,q) \in P \times P \mid \exists z \in P((p,z)
\in S^{-1} \land (z,q) \in S\}$ \\
${(p,q) \in P \times P \mid p is the descendant of z and q is the
  descendant of z}$ \\
${(p,q) \in P \times P \mid p and q are descendant of some same man}$

\section{Problem 7}
\subsection{Solution (a)}
Let $S$ be the reflexive closure of $R$.
$\Rightarrow$ Suppose $R$ is reflexive. From clause $1$ of reflexive
closure it follows that $R \subseteq S$. From clause 3, it follows
that $S \subseteq R$. So, $S = R$.
$\Leftarrow$ Suppose $R = S$. By clause 2, $R$ is reflexive.

\subsection{Solution (b)}
Yes, it holds.

\section{Problem 8}
\begin{align*}
  a \in Dom(S) iff \exists b \in B((a,b) \in S) \\
  iff \exists b \in B((a,b) \in R \lor (a,b) \in R^{-1}) \\
  iff \exists b \in B((b,a) \in R^{-1} \lor (b,a) \in R) \\
  iff \exists b \in B((b,a) \in R^{-1} \lor R) \\
  iff \exists b \in B((b,a) \in S) \\
  iff a \in Ran(S)
\end{align*}

\begin{align*}
  a \in Dom(S) iff \exists b \in B((a,b) \in S) \\
  iff \exists b \in B((a,b) \in R \lor R^{-1}) \\
  iff \exists b \in B((a,b) \in R \lor (a,b) \in R^{-1}) \\
  iff \exists b \in B((a,b) \in R \lor (b,a) \in R) \\
  iff \exists b \in B((a,b) \in R) \lor \exists b \in B((b,a) \in R) \\
  iff a \in Ran(R) \cup Dom(R)
\end{align*}

\section{Problem 9}
Let $T = \{(x,y) \in S \mid x \in Dom(R) \land y \in Ran(R)\}$ \\

\subsection{Proof of $R \subseteq T$}
Let $a,b$ be arbitrary element in $A$ such that $(a,b) \in R$. Since
$a \in Dom(R)$ and $b \in Ran(R)$ it follows that $(a,b) \in T$. Since
$a$ and $b$ are arbitrary, it follows that $R \subseteq T$.

\subsection{Proof of transitivity of T}
Let $a,b$ and $c$ be arbitrary element in $A$ such that $(a,b) \in T$
and $(b,c) \in T$. So, it follows that $(a,b) \in S$ and $(b,c) \in
S$. Since $S$ is transitive closure, it follows that $(a,c) \in S$.
So, $(a,c) \in T$. Hence T is transitive.

Proof. Since $T$ is transitive and $R \subseteq T$ it follows that $S
\subseteq T$. Since $S$ is transitive closure of $R$, we know that $R
\subseteq S$. So $Dom(R) \subseteq Dom(S)$.
Let $a,b$ be arbitrary elements in $A$ such that $(a,b) \in S$ . Since
$S \subseteq T$, it follows that $a \in Dom(R)$ and $b \in Dom(R)$.
Since $a$ and $b$ were arbitrary it follows that $S \subseteq R$. So,
$Dom(S) \subseteq Dom(R)$. Therefore $Dom(S) = Dom(R)$.

\section{Problem 10}
\subsection{Solution (a)}
We know that $A \times A$ is symmetric. Also $R \subseteq A \times A$.
So from the definition of $F$, it follows that $A \times A \in F$.
Therefore $F \neq \emptyset$.

\subsection{Solution (b)}
\subsubsection{Proof of  $R \subseteq \cap F$}
Let $a,b$ be arbitrary elements in $A$ such that $(a,b) \in R$. Let
$T$ be arbitrary elements of $F$. Then by the definition of $F$, $R
\subseteq T$, so $(a,b) \in T$. Since $T$ was arbitrary this shows
that $\forall T \in F((a,b) \in T)$. So $(a,b) \in \cap F$. Thus $R
\subseteq \cap F$.

\subsubsection{Proof of $\cap F$ is symmetric}
Let $a,b$ be arbitrary element in $A$ such that $(a,b) \in S$. Let $T$
be arbitrary element in $F$. Then by the definition of $F$, $T$ is
symmetric. So $(a,b) \in S = (a,b) \in \cap F$. So $(a,b) \in T$.
Since $T$ is symmetric, $(b,a) \in T$. Now since $F$ is arbitrary,
$(b,a) \in \cap F$. Therefore $\cap F$ is symmetric.

\subsection{$\cap F$ is smallest element in F}
Let $T$ be arbitrary element in $F$. We have to prove that $\cap F
\subseteq T$. Let $a,b$ be arbitrary element in $A$ such that $(a,b)
\in \cap F$. Since $T$ is arbitrary it follows that $\forall T \in
F((a,b) \in T)$. So, $(a,b) \in T$. Q.E.D.

\section{Problem 11}
Suppose $R_1$ and $R_2$ are relations on $A$ and $R_1 \subseteq R_2$.
\subsection{Solution (a)}
Let $S_1$ and $S_2$ be reflexive closure of $R_1$ and $R_2$. Then it
follows that $R_1 \subseteq S_1$ and $R_2 \subseteq S_2$. Let $a,b$ be
arbitrary element in $A$ such that $(a,b) \in S_1$. There can be two
cases now:

Case 1. $(a,b) \in R_1$. Since $R_1 \subseteq R_2$, it follows that
$(a,b) \in R_2$. From $R_2 \subseteq S_2$, it follows that $S_1
\subseteq S_2$.
Case 2. $(a,b) \notin R_1$ Then it follows that $(a,b)$ was built as
part of reflexive closure, so $a = b$. Now $R_2$ is a relation on $A$.
So, reflexive closure of $R_2$ will have $(a,b)$. So $(a,b) \in S_2$.

\subsection{Transitive closure}
Let $S_1$ and $S_2$ be transitive closure of $R_1$ and $R_2$. Since
the transitive closure of a relation is the smallest transitive
relation containing the original relation, we know $R_1 \subseteq R_2$
and $R_2 \subseteq S_2$. So, $R_1 \subseteq S_2$. $S_2$ is transitive
and contains $R_1$, so it must contain the transitive closure of $R_1$
ie $S_1$. Since the transitive closure is the smallest transitive
relation containing $R_1$. So $S_1 \subseteq S_2$.

\subsubsection{Symmetric closure}
Let $S_1$ and $S_2$ be symmetric closure of $R_1$ and $R_2$. Let $a,b$
be arbitrary element in $A$ such that $(a,b) \in S_1$. We know that
$R_1 \subseteq S_1$. Let us consider the cases:
Case 1. $(a,b) \in R_1$ Since $R_1 \subseteq R_2$, it follows that
$(a,b) \in R_2$. From $R_2 \subseteq S_2$, it follows that $(a,b) \in
S_2$.
Case 2. $(a,b) \notin R_1$ This means $(b,a) \in R_1$. Since $R_1
\subseteq R_2$, it follows that $(b,a) \in R_2$. Now symmetric closure
of $R_2$ will have $(a,b)$. So $(a,b) \in S_2$.

\section{Problem 12}
\subsection{Solution (a)}
$S_1 = R_1 \cup i_A$
$S_2 = R_2 \cup i_A$
$S_1 \cup S_2 = R_1 \cup i_A \cup R_2 \cup i_A$
$R_1 \cup R_2 \cup i_A$
$R \cup i_A$
$S$

\subsection{Solution (b)}
$S_1 = R_1 \cup (R_1)^{-1}$
$S_2 = R_2 \cup (R_2)^{-1}$
\begin{align*}
  S_1 \cup S_2 = R_1 \cup R_2 \cup (R_1)^{-1} \cup (R_2)^{-1} \\
  = R \cup R^{-1} \\
  = S
\end{align*}

\section{Problem 13}
Suppose $R_1$ and $R_2$ are relations on $A$. Let $R = R_1 \cap R_2$.

\subsection{Solution (a)}
Let $S_1, S_2$ and $S$ be the reflexive clsoure of $R_1, R_2$ and $R$.

We know that $R \subseteq R_1$and $R \subseteq R_2$. From exercise 11,
it follows that $S \subseteq S_1 $ and $S \subseteq S_2$. Let $(a,b)$
be arbitrary element in $S$. Then it follows that $(a,b) \in S_1 \land
S_2$. So, $S \subseteq S_1 \cap S_2$.

\subsection{Solution (b)}
We know that $R \subseteq R_1$ and $R \subseteq R_2$. From exercise
11, it follows that $S \subseteq S_1$ and $S \subseteq S_2$. So, $S
\subseteq S_1 \cap S_2$.

\subsection{Solution (c)}
Similar proof.

\section{Solution 14}
\begin{align*}
  R_1 = \{(1,2),(2,5),(2,9)\} \\
  R_2 = \{(1,2)\} \\
  S_1 = \{(1,2), (2,5), (1,5), (1,9)\} \\
  S_2 = \{(1,2)\} \\
  R = R_1 \setminus R_2 = \{(2,5),(2,9)\} \\
  S_1 \setminus S_2 = \{(2,5), (1,5), (1,9)\} \\
  S = \{(2,5), (2,9)\} \\
  S_1 \setminus S_2 \nsubseteq S \land S \nsubseteq S_1 \setminus S_2 \\
\end{align*}

\section{Solution 15}
Suppose $R$ is a relation on $A$. Let $S = R \cup i_A \cup R^{-1}$. We
know that $S$ is symmetric and reflexive. Let $F = \{ T \subseteq A
\times A \mid T \text{is reflexive and symmetric and} R \subseteq
T\}$. Let $T$ be an arbitrary element in $F$. We have to prove that $S
\subseteq T$. Let $(a,b)$ be arbitrary element in $S$. Now let us
consider the cases:

Case 1. $(a,b) \in R$. Since $R \subseteq T$, it follows that
$(a,b)\in T$. So $S \subseteq T$.
Case 2. $(a,b) \notin R$. Then it means either $(a,b) \in i_A$ ir
$(a,b) \in R^{-1}$. If $(a,b) \in i_A$ then $a = b$. Since $T$ is
reflexive closure, $(a,b) \in T$. If $(a,b) \in R^{-1}$, then $(b,a)
\in R$. Since $R \subseteq T$, it follows that $(b,a) \in T$. Now $T$
is symmetric closure, so $(a,b) \in T$. So, we can conclude that $S
\subseteq T$.

\section{Solution 16}
Suppose $R$ is a relation on $A$ and $S$ be the reflexive closure of
$R$.
\subsection{Solution (a)}
Suppose $R$ is symmetric. Let $a,b$ be arbitrary elements in $A$ such
that $(a,b) \in S$. We know that $R \subseteq S$. Let us consider the
cases:
Case 1. $(a,b) \in S$. Since $R$ is symmetric, it follows that $(b,a)
\in R \land (b,a) \in S$.
Case 2. $(a,b) \notin R$ Then it means $a = b$. And $(b,a) \in S$.j

\subsection{Solution (b)}
Suppose $R$ is transitive. Let $a,b,c$ be arbitrary element in $A$
such that $(a,b) \in S$ and $(b,c) \in S$. We know that $R \subseteq
S$. Let us consider the cases:

Case 1. $(a,b) \in R \land (b,c) \in R$. Since $R$ is transitive it
follows that $(a,c) \in R$. From $R \subseteq S$, it follows that
$(a,c) \in S$.
Case 2. $(a,b) \in R \land (b,c) \notin R$. Then it follows that $b =
c$. So, $(a,c) \in R$. From $R \subseteq S$, it follows that $(a,c)
\in S$.
Case 3. $(a,b) \notin R \land (b,c) \in R$. Then it follows that $a =
b$. So, $(a,c) \in R$. From $R \subseteq S$, it follows that $(a,c)
\in S$.
Case 4. $(a,b) \notin R \land (b,c) \notin R$. Then it follows that $a
= b$ and $b = c$. Since $S$ is reflexive closure of $R$, it follows
that $(a,c) \in S$.
\end{document}


