%%%%%%%%%%%%%%%%%%%%%%%%%%%%%%%%%%%%%%%%%
% Author: Sibi <sibi@psibi.in>
%%%%%%%%%%%%%%%%%%%%%%%%%%%%%%%%%%%%%%%%%
\documentclass{article}
\usepackage{graphicx}
\usepackage{verbatim}
\usepackage{amsmath}
\usepackage{amsfonts}
\usepackage{amssymb}
\usepackage{tabularx}
\setlength\parskip{\baselineskip}
\begin{document}
\title{Chapter 4 (Section 4.3)}
\author{Sibi}
\date{\today}
\maketitle
\newpage

\section{Problem 1}
\subsection{Solution (a)}

Reflexive closure of $R = \{(b,b), (c,c), (a,a), (a,b), (b,c),
(c,b)\}$ \\
Symmetric closure of $R = (a,a), (a,b), (b,a), (b,c), (c,b)$ \\
Transitive closure of $R = (a,a), (a,b), (a,c), (a,b), (b,c), (c,b),
(b,b)$

\subsection{Solution (b)}

Reflexive closure of $R = \{(x,y \in R \times R \mid x < y \lor x =
y)\}$ \\
$= \{(x,y) \in R \times R \mid x \leq y\}$ \\ \\
Symmetric closure of $R = \{(x,y) \in R \times R \mid x < y \lor y <
x\}$ \\ \\
Transitive closure of $R = R$

\subsection{Solution (c)}

Reflexive and symmetric closure of $D_r = D_r$
Transitive closure of $D_r = \mathbb{R} \times \mathbb{R}$

\section{Problem 3}
\subsection{Solution (a)}

Suppose $R$ is asymmetric. Let $x$ and $y$ be an arbitrary element in
$A$ such that $xRy$ and $yRx$. Since $R$ is asymmetric, from $xRy$ it
follows that $(y,x) \notin R$. But this contradicts with out
assumption that $yRx$. So, it is vacuously true.

\subsection{Solution (b)}
Suppose $R$ is a strict partial order on $A$. Let $x$ and $y$ be
arbitrary element in $A$ such that $(x,y) \in R$. Let us try to prove
by contradiction. Suppose $(y,x) \in R$. Then from transitive property
of $R$ it follows that $(x,x) \in R$. But this contradicts the fact
that $R$ is irreflexive. So $(y,x) \notin R$.

\section{Problem 4}
\subsection{Solution (a)}
Suppose $R$ is a strict partial order on $A$. Let $S$ be the reflexive
closure on $R$. That means $S = R \cup i_A$.
\subsubsection{Proof of reflexive}
By the definition of reflexive closure, $S$ is reflexive.

\subsubsection{Proof of transitive}
Let $x,y$ and $z$ be arbitrary elements in $A$ such that $(x,y) \in S$
and $(y,z) \in S$. We have to prove that $(x,z) \in S$. Now we know
that $S = R \cup i_A$. Let us consider the cases separately:
Case 1. $(x,y) \in R$ Now this itself has two sub cases within it.
\begin{itemize}
\item $y,z \in R$ We know that $R$ is transitive, so $x,z \in R$.
  So $(x,z) \in R \cup i_A$.
\item $y,z \in i_A$ From the property of identity set it follows that
  $y = z$. Substituting it in $(x,y) \in R$, we get $(x,z) \in R$. So
  $(x,z) \in R \cup i_A$.
\end{itemize}
Case 2. $(x,y) \in i_A$ From the property of identity set it follows
that $x = y$. Substituting that in $(y,z) \in S$ gives $(x,z) \in S$.

\subsubsection{Proof of Anti-symmetric}
Let $x$ and $y$ be arbitrary elements in $A$ such that $(x,y) \in S$
and $(y,x) \in S$. We have to prove that $x = y$. Now we know
that $S = R \cup i_A$. Let us consider the cases separately:
Case 1. $(x,y) \in R$ Now this itself has two sub cases within it.
\begin{itemize}
\item $y,x in R$. Since $R$ is transitive, it follows that $(x,x) \in
  R$. But we know that $R$ is irreflexive, so this contradicts the
  fact that $(x,x) \notin R$. So $(y,x) \notin R$.
\item $y,x \in i_A$ From the property of identity set it follows that
  $y = x$. 
\end{itemize}
Case 2. $(x,y) \in i_A$. From the property of identity set, it follows
that $x = y$

\subsection{Solution (b)}
Suppose $R$ is total strict order. Let $x$ and $y$ be arbitrary
elements in $A$. We have to prove that $xSy \lor ySx$. We know that R
satisfies trichotomy, so $xRy \lor yRx \lor x = y$. Let us consider
the cases separately:
\begin{itemize}
\item Case 1. $xRy$ Since $S = R \cup i_A$, it follows that $(x,y) \in
  S$. So $xSy \lor ySx$.
\item Case 2. $yRx$ Similar to the case 1 proof.
\item Case 3. $x = y$ So, $(x,y) \in i_A$. It follows that $(x,y) \in
  R \cup i_A$. So $xSy \lor ySx$.
\end{itemize}

\section{Problem 5}
Suppose $R$ is a relation on $A$. Let $S = R \setminus i_A$.
\subsection{Solution (a)}
Let
$F = \{T \subset A \times A \mid T \subseteq R and T is
irreflexive\}$. We have to prove things here.
\subsubsection{Proof $S \in F$}
Let $x,y$ be arbitrary elements in $A$ such that $(x,y) \in S$. From
$S = R \setminus i_A$, it follows that $(x,y) \in R$. So, $S \subseteq
R$. It also follows that $(x,y) \notin i_A$. So $\forall x \forall y
(x,y) \notin R$. That can be re-stated as $\forall x (x,x) \notin R$.
So $R$ is irreflexive. So, we can conclude that $S \in F$.
\subsubsection{S is the largest element in F}
Let $T$ be an arbitrary element in $F$. We have to prove that $T
\subseteq S$. Let $(x,y)$ be arbitrary element in $T$. Since $T
\subseteq R$, it follows that $(x,y) \in R$. Also since $T$ is
irreflexive, it follows that $x \neq y$. So $(x,y) \notin i_A$. Hence
$(x,y) \in R \setminus i_A$. So, $T \subseteq S$.
\end{document}


